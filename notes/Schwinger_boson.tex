#LyX file created by tex2lyx 2.3
\lyxformat 544
\begin_document
\begin_header
\save_transient_properties true
\origin /Users/Hao/Desktop/Notes/basic_condensed_matter_theory /path_integral_cmp/
\textclass article
\begin_preamble
\usepackage{amsthm}\usepackage{graphicx}
\usepackage{physics}
\usepackage{bm}
\usepackage{appendix}
\numberwithin{equation}{subsection}

\newcommand*{\rom}[1]{\expandafter\@slowromancap\romannumeral #1@}

\title{My Notes -- Spin path integral and Schwinger bosons}
\author{Hao Zhang}
\date{May 2018}


\end_preamble
\use_default_options false
\maintain_unincluded_children false
\language english
\language_package none
\inputencoding utf8
\fontencoding default
\font_roman "default" "default"
\font_sans "default" "default"
\font_typewriter "default" "default"
\font_math "auto" "auto"
\font_default_family default
\use_non_tex_fonts false
\font_sc false
\font_osf false
\font_sf_scale 100 100
\font_tt_scale 100 100
\use_microtype false
\use_dash_ligatures true
\graphics default
\default_output_format default
\output_sync 0
\bibtex_command default
\index_command default
\paperfontsize default
\spacing single
\use_hyperref false
\papersize default
\use_geometry false
\use_package amsmath 2
\use_package amssymb 2
\use_package cancel 0
\use_package esint 1
\use_package mathdots 0
\use_package mathtools 2
\use_package mhchem 0
\use_package stackrel 0
\use_package stmaryrd 0
\use_package undertilde 0
\cite_engine basic
\cite_engine_type default
\biblio_style plain
\use_bibtopic false
\use_indices false
\paperorientation portrait
\suppress_date false
\justification true
\use_refstyle 0
\use_minted 0
\index Index
\shortcut idx
\color #008000
\end_index
\secnumdepth 3
\tocdepth 3
\paragraph_separation indent
\paragraph_indentation default
\is_math_indent 0
\math_numbering_side default
\quotes_style english
\dynamic_quotes 0
\papercolumns 1
\papersides 1
\paperpagestyle default
\tracking_changes false
\output_changes false
\html_math_output 0
\html_css_as_file 0
\html_be_strict false
\end_header

\begin_body

\begin_layout Standard

\begin_inset ERT
status collapsed

\begin_layout Plain Layout

\backslash
maketitle
\end_layout

\end_inset


\end_layout

\begin_layout Section
Basic tools
\end_layout

\begin_layout Subsection
Spin representation
\end_layout

\begin_layout Subsubsection
Bosonization
\end_layout

\begin_layout Standard
The broken symmetry phases of a spin system can be described by the Holstein-Primakoff (HP) bosons,
\end_layout

\begin_layout Standard

\begin_inset Formula \begin{align}
    S^+ & = (\sqrt{2S - n_b})b \nonumber \\
    S^- & = b^{\dagger}(\sqrt{2S - n_b}) \nonumber \\
    S^z & = S - n_b
\end{align}
\end_inset


\end_layout

\begin_layout Standard
The symmetric phases can be described by the Schwinger bosons (SB)
\end_layout

\begin_layout Standard

\begin_inset Formula \begin{align}
    S^+ & = a^{\dagger}b \nonumber \\
    S^- & = b^{\dagger}a\nonumber \\
    S^z & = \frac{1}{2}(a^{\dagger}a-b^{\dagger}b)
    \label{eq:Schwingerepresentation}
\end{align}
\end_inset


\end_layout

\begin_layout Standard
It's easy to find the correspondence between the two types of bosons
\end_layout

\begin_layout Standard

\begin_inset Formula \begin{align}
      b \longleftrightarrow & b \nonumber \\
      a \longleftrightarrow & (\sqrt{2S - n_b})
\end{align}
\end_inset


\end_layout

\begin_layout Standard
The Fock space of Schwinger bosons consists of both physical and non-physical states. The physical states form a subspace with definite magnitude of 
\begin_inset Formula $S$
\end_inset


\end_layout

\begin_layout Standard

\begin_inset Formula \begin{equation}
    P_S = \{\ket{n_a,n_b} : n_a + n_b = 2S \}
\end{equation}
\end_inset


\end_layout

\begin_layout Standard
Then
\end_layout

\begin_layout Standard

\begin_inset Formula \begin{equation}
    \bm{S}^2P_S = S(S+1)P_S
\end{equation}
\end_inset


\end_layout

\begin_layout Standard
The spin states are given by
\end_layout

\begin_layout Standard

\begin_inset Formula \begin{equation}
    \ket{S,m} = \frac{(a^{\dagger})^{S+m}}{\sqrt{(S+m)!}}\frac{(b^{\dagger})^{S-m}}{\sqrt{(S-m)!}}\ket{0}
\end{equation}
\end_inset


\end_layout

\begin_layout Standard
where 
\begin_inset Formula $\ket{0}$
\end_inset

 is the vacuum of Schwinger bosons. In the Fock space representation,
\end_layout

\begin_layout Standard

\begin_inset Formula \begin{equation}
    \ket{S+m,S-m} = \frac{(a^{\dagger})^{S+m}}{\sqrt{(S+m)!}}\frac{(b^{\dagger})^{S-m}}{\sqrt{(S-m)!}}\ket{0}
    \label{eq:statesandSchwinger}
\end{equation}
\end_inset


\end_layout

\begin_layout Standard
It's easy to find the correspondence: 
\begin_inset Formula $(S+m)+(S-m) = 2S$
\end_inset

, then the state represents a physical state; 
\begin_inset Formula $\frac{1}{2}((S+m) - (S-m)) = m$
\end_inset

, the eigenvalue of 
\begin_inset Formula $S^z$
\end_inset

.
\end_layout

\begin_layout Subsubsection
Spin rotation
\end_layout

\begin_layout Standard
The basic ingredients of the second quantization formalism are the creation / annihilation operators and the Fock states enforced by the commutation/anti-commutation relations of those operators. More complicated operators can be constructed from the basic ingredients. We can classify the commonly used operators into two categories: normal bilinear operators and linear operators. The 
\shape italic
bilinear
\shape default
 operators are defined as
\end_layout

\begin_layout Standard

\begin_inset Formula \begin{equation}
    \hat{A} = \sum_{ij}a_i^{\dagger}A_{ij}a_j = \bm{a}^{\dagger}\cdot\bm{A}\cdot\bm{a}
\end{equation}
\end_inset


\end_layout

\begin_layout Standard
where 
\begin_inset Formula $A_{ij}$
\end_inset

 is the matrix representation in the single-particle basis 
\begin_inset Formula $\phi_i$
\end_inset

 generated by the creation/annihilation operators. Then it's easy to see that the commutation relation between two bilinear operators is
\end_layout

\begin_layout Standard

\begin_inset Formula \begin{equation}
    \comm{\hat{A}}{\hat{B}} = \bm{a}^{\dagger}\cdot\comm{A}{B}\cdot\bm{a}
\end{equation}
\end_inset


\end_layout

\begin_layout Standard
This relation holds for both bosons and fermions. The result of 
\begin_inset Formula $\comm{A}{B}$
\end_inset

 should be another matrix. We immediately conclude that all bilinear operators form a general 
\shape italic
Lie algebra
\shape default
. The simplest example is the spin operators, which are the generators of Lie algebra 
\begin_inset Formula $\mathfrak{su}(2)$
\end_inset

. 
\begin_inset Newline newline
\end_inset


\end_layout

\begin_layout Standard
The 
\shape italic
linear
\shape default
 operators are defined as
\end_layout

\begin_layout Standard

\begin_inset Formula \begin{equation}
    \hat{\bm{v}}^{\dagger} = \sum_i v_ia_i^{\dagger} = \bm{v}\cdot\bm{a}^{\dagger}
\end{equation}
\end_inset


\end_layout

\begin_layout Standard
The rotation of 
\begin_inset Formula $\hat{\bm{v}}^{\dagger}$
\end_inset

 (generalized vector) in the Fock space is generated by the Lie algebra generators 
\begin_inset Formula $\hat{A}$
\end_inset

:
\end_layout

\begin_layout Standard

\begin_inset Formula \begin{equation}
   \hat{\bm{v}'}^{\dagger} = e^{i\theta\hat{A}}\hat{\bm{v}}^{\dagger}e^{-i\theta\hat{A}}
\end{equation}
\end_inset


\end_layout

\begin_layout Standard
The commutation relation between a 
\shape italic
bilinear
\shape default
 operator and s 
\shape italic
linear
\shape default
 operator is
\end_layout

\begin_layout Standard

\begin_inset Formula \begin{equation}
    \comm{\hat{A}}{\hat{\bm{v}}^{\dagger}} = (A\bm{v})\cdot\bm{a}^{\dagger}
\end{equation}
\end_inset


\end_layout

\begin_layout Standard
If 
\begin_inset Formula $\bm{v}$
\end_inset

 happens to be an eigenvector of 
\begin_inset Formula $A$
\end_inset

 with eigenvalue 
\begin_inset Formula $v$
\end_inset

, then 
\begin_inset Formula $\bm{\hat{v}}^{\dagger}$
\end_inset

 is an 
\shape italic
eigenoperator
\shape default
 of 
\begin_inset Formula $\comm{\hat{A}}{.}$
\end_inset

 with eigenvalue 
\begin_inset Formula $v$
\end_inset

. In this case, the rotation can be written as
\end_layout

\begin_layout Standard

\begin_inset Formula \begin{align}
    e^{i\theta\hat{A}}\hat{\bm{v}}^{\dagger}e^{-i\theta\hat{A}} & = \hat{\bm{v}}^{\dagger} + i\theta\comm{\hat{A}}{\hat{\bm{v}}^{\dagger}} + \frac{(i\theta)^2}{2}\comm{\hat{A}}{\comm{\hat{A}}{\hat{\bm{v}}^{\dagger}}} + ... \nonumber \\
    & = e^{iv\theta}\hat{\bm{v}}^{\dagger}
\end{align}
\end_inset


\end_layout

\begin_layout Standard
If the Lie algebra of bilinear operators has 
\begin_inset Formula $\alpha$
\end_inset

 independent generators, the rotation operator is given by
\end_layout

\begin_layout Standard

\begin_inset Formula \begin{equation}
    \hat{U}_{\theta} = e^{i\sum_{\alpha} \theta_{\alpha}\hat{A}_{\alpha}}
\end{equation}
\end_inset


\end_layout

\begin_layout Standard
The corresponding matrix representation is
\end_layout

\begin_layout Standard

\begin_inset Formula \begin{equation}
    U_{\theta} = e^{i\sum_{\alpha} \theta_{\alpha}A_{\alpha}}
\end{equation}
\end_inset


\end_layout

\begin_layout Standard
Then the eigenoperator 
\begin_inset Formula $\hat{\bm{v}}^{\dagger}$
\end_inset

 of 
\begin_inset Formula $\comm{\hat{A}}{.}$
\end_inset

 transforms in a simple manner
\end_layout

\begin_layout Standard

\begin_inset Formula \begin{equation}
    \hat{U}_{\theta}\hat{\bm{v}}^{\dagger}\hat{U}_{\theta}^{-1} = (U_{\theta}\bm{v})\cdot a^{\dagger}
\end{equation}
\end_inset


\end_layout

\begin_layout Standard
Consider the simplest example: the spin operators, which are generators of the 
\begin_inset Formula $\mathfrak{su}(2)$
\end_inset

 algebra, can be written as
\end_layout

\begin_layout Standard

\begin_inset Formula \begin{equation}
    \hat{S}^{\alpha} = \frac{1}{2}\sum_{ss'=1}^2a_s^{\dagger}\sigma^{\alpha}_{ss'}a_{s'} \quad \alpha = x,y,z
\end{equation}
\end_inset


\end_layout

\begin_layout Standard
The general rotation in three-dimension are parametrized by three Euler angles 
\begin_inset Formula $\phi,\theta,\chi$
\end_inset


\end_layout

\begin_layout Standard

\begin_inset Formula \begin{equation}
    \mathcal{R} = e^{i\phi \hat{S^z}}e^{i\theta \hat{S^y}}e^{i\chi \hat{S^z}}
    \label{eq:rotation}
\end{equation}
\end_inset


\end_layout

\begin_layout Standard
First of all, it's easy to show that
\end_layout

\begin_layout Standard

\begin_inset Formula \begin{align}
    \mathcal{R}\hat{S}_z\mathcal{R}^{\dagger} & = e^{i\phi \hat{S^z}}e^{i\theta \hat{S^y}}e^{i\chi \hat{S^z}}\hat{S}_z e^{-i\chi \hat{S^z}}e^{-i\theta \hat{S^y}}e^{-i\phi \hat{S^z}} \nonumber\\
                                              & = e^{i\phi \hat{S^z}}e^{i\theta \hat{S^y}}\hat{S}_ze^{-i\theta \hat{S^y}}e^{-i\phi \hat{S^z}} \nonumber\\
                                              & = \hat{S}_x\sin\theta\cos\phi + \hat{S}_y\sin\theta\sin\phi + \hat{S}_z\cos\theta
                                              \label{eq:szgeneral}
\end{align}
\end_inset


\end_layout

\begin_layout Standard
We are looking for the unitary transform on the Schwinger boson operators
\end_layout

\begin_layout Standard

\begin_inset Formula \begin{align}
    (a^{\dagger})' = \mathcal{R}a^{\dagger}\mathcal{R}^{\dagger} & = ua^{\dagger} + vb^{\dagger} \nonumber\\
    (b^{\dagger})' = \mathcal{R}b^{\dagger}\mathcal{R}^{\dagger} & = -v^*a^{\dagger} + u^*b^{\dagger}
    \label{eq:transofschwinger}
\end{align}
\end_inset


\end_layout

\begin_layout Standard
Notice that the choice of 
\begin_inset Formula $u$
\end_inset

 and 
\begin_inset Formula $v$
\end_inset

 is constrained by the condition for unitary transform: 
\begin_inset Formula $\abs{u}^2 + \abs{v}^2=1$
\end_inset

. On the other hand,
\end_layout

\begin_layout Standard

\begin_inset Formula \begin{equation}
    \mathcal{R}\hat{S}_z\mathcal{R}^{\dagger} = \frac{1}{2}\mathcal{R}(a^{\dagger}a-b^{\dagger}b)\mathcal{R}^{\dagger} = \frac{1}{2}((a^{\dagger})'a'-(b^{\dagger})'b')
    \label{eq:transofSz}
\end{equation}
\end_inset


\end_layout

\begin_layout Standard
Plug 
\begin_inset CommandInset ref
LatexCommand eqref
reference "eq:transofschwinger"
plural "false"
caps "false"
noprefix "false"

\end_inset

 into 
\begin_inset CommandInset ref
LatexCommand eqref
reference "eq:transofSz"
plural "false"
caps "false"
noprefix "false"

\end_inset

, we can obtain
\end_layout

\begin_layout Standard

\begin_inset Formula \begin{align}
   \frac{1}{2}\mathcal{R}(a^{\dagger}a-b^{\dagger}b)\mathcal{R}^{\dagger}  & = \frac{1}{2}(\abs{u}^2 - \abs{v}^2) (a^{\dagger}a-b^{\dagger}b) + uv^*a^{\dagger}b + u^*vb^{\dagger}a \nonumber \\
                  & = \Hat{S}_z(\abs{u}^2 - \abs{v}^2) + uv^* S^+ + u^*vS^- 
                  \label{eq:szlessgeneral}
\end{align}
\end_inset


\end_layout

\begin_layout Standard
Compare the result with 
\begin_inset CommandInset ref
LatexCommand eqref
reference "eq:szgeneral"
plural "false"
caps "false"
noprefix "false"

\end_inset

, we conclude that
\end_layout

\begin_layout Standard

\begin_inset Formula \begin{align}
    u & = \cos(\theta/2)e^{i\phi/2} \nonumber \\
    v & = \sin(\theta/2)e^{-i\phi/2}
    \label{eq:uv}
\end{align}
\end_inset


\end_layout

\begin_layout Subsubsection
Spin coherent states
\end_layout

\begin_layout Standard
The spin coherent states are created by applying the rotation operator 
\begin_inset Formula $\mathcal{R}$
\end_inset

 
\begin_inset CommandInset ref
LatexCommand eqref
reference "eq:rotation"
plural "false"
caps "false"
noprefix "false"

\end_inset

 to the maximally polarized states 
\begin_inset Formula $\ket{S,S}$
\end_inset


\end_layout

\begin_layout Standard

\begin_inset Formula \begin{equation}
    \ket{\hat{\Omega}} = \mathcal{R}(\phi,\theta,\chi)\ket{S,S}
\end{equation}
\end_inset


\end_layout

\begin_layout Standard
where the unit vector
\end_layout

\begin_layout Standard

\begin_inset Formula \begin{equation}
    \hat{\Omega} = (\sin\theta\cos\phi,\sin\theta\sin\phi,\cos\theta)
\end{equation}
\end_inset


\end_layout

\begin_layout Standard
parametrizes the spin coherent states. The freedom to define 
\begin_inset Formula $\chi$
\end_inset

 is the gauge freedom. The two independent angles are defined within the domains
\end_layout

\begin_layout Standard

\begin_inset Formula \begin{equation}
    \{\theta \in [0,\pi] \quad \phi \in [-\pi,\pi)\}
\end{equation}
\end_inset


\end_layout

\begin_layout Standard
We can write down the spin coherent states explicitly
\end_layout

\begin_layout Standard

\begin_inset Formula \begin{align}
   \ket{\hat{\Omega}} & =  e^{i\phi \hat{S^z}}e^{i\theta \hat{S^y}}e^{i\chi \hat{S^z}}\ket{S,S} \nonumber \\
                      & =  e^{i\chi S} e^{i\phi \hat{S^z}}e^{i\theta \hat{S^y}} \ket{S,S} \nonumber \\
                      & =  e^{i\chi S} \frac{(a^{\dagger'})^{2S}}{\sqrt{2S!}}\ket{0}
                      \label{eq:coherentstatesexplicit}
\end{align}
\end_inset


\end_layout

\begin_layout Standard
as a maximally polarized states in the local frame (created by Schwinger bosons defined in 
\begin_inset CommandInset ref
LatexCommand eqref
reference "eq:statesandSchwinger"
plural "false"
caps "false"
noprefix "false"

\end_inset

). Then we can exploit the transform properties of the Schwinger boson operators
\end_layout

\begin_layout Standard

\begin_inset Formula \begin{align}
   \ket{\hat{\Omega}} & = e^{i\chi S} \frac{(u a^{\dagger} + vb^{\dagger})^{2S}}{\sqrt{2S!}}\ket{0} \nonumber \\
                      & = e^{i\chi S} \sqrt{(2S)!}\sum_m \frac{u^{S+m}v^{S-m}}{\sqrt{(S+m)!(S-m)!}}\ket{S,m}
\end{align}
\end_inset


\end_layout

\begin_layout Standard
where I have used the binomial expansion. Next, we can check the overlap of any two coherent states
\end_layout

\begin_layout Standard

\begin_inset Formula \begin{equation}
    \bra{\Hat{\Omega}}\ket{\Hat{\Omega}'} = e^{-iS(\chi-\chi')}(u^*u'+v^*v')^{2S} = \bigg(\frac{1+\Hat{\Omega}\cdot\Hat{\Omega}'}{2}\bigg)^{S}e^{-iS\psi}
\end{equation}
\end_inset


\end_layout

\begin_layout Standard

\begin_inset Formula \begin{equation}
    \psi = 2\tan^{-1}\bigg(\tan\bigg(\frac{\phi-\phi'}{2}\bigg)\frac{\cos((\theta+\theta')/2)}{\cos((\theta-\theta')/2)}\bigg) + \chi-\chi'
\end{equation}
\end_inset


\end_layout

\begin_layout Standard
Then we find that the coherent states are 
\series bold
not
\series default
 orthogonal. In fact, they form a over-complete basis for the space of states of spin 
\begin_inset Formula $S$
\end_inset

. Consider the following integral
\end_layout

\begin_layout Standard

\begin_inset Formula \begin{equation}
    C \int d\Hat{\Omega} \ket{\hat{\Omega}}\bra{\hat{\Omega}} \quad  \text{where} \quad d\Hat{\Omega} = d\theta\sin\theta d\phi
\end{equation}
\end_inset


\end_layout

\begin_layout Standard
where 
\begin_inset Formula $C$
\end_inset

 is a constant. We want to prove that it is a 
\shape italic
resolution of identity
\shape default
. Using the results 
\begin_inset CommandInset ref
LatexCommand eqref
reference "eq:uv"
plural "false"
caps "false"
noprefix "false"

\end_inset

 and 
\begin_inset CommandInset ref
LatexCommand eqref
reference "eq:coherentstatesexplicit"
plural "false"
caps "false"
noprefix "false"

\end_inset

, we can write
\end_layout

\begin_layout Standard

\begin_inset Formula \begin{align}
   C \int d\Hat{\Omega} \ket{\hat{\Omega}}\bra{\hat{\Omega}} & = C(2\pi)\int_{-1}^{1}d\cos\theta\sum_m\bigg(\frac{1+\cos\theta}{2}\bigg)^{S+m}\bigg(\frac{1-\cos\theta}{2}\bigg)^{S-m} \nonumber \\
   & \times \frac{(2S)!}{(S+m)!(S-m)!}\ket{S,m}\bra{S,m}
\end{align}
\end_inset


\end_layout

\begin_layout Standard
We have to deal with integrals like
\end_layout

\begin_layout Standard

\begin_inset Formula \begin{equation}
    I_{S,m} = \frac{1}{2}\int_0^{\pi}d\theta\sin\theta\bigg(\frac{1+\cos\theta}{2}\bigg)^{S+m}\bigg(\frac{1-\cos\theta}{2}\bigg)^{S-m}
\end{equation}
\end_inset


\end_layout

\begin_layout Standard
define a change of variable as
\end_layout

\begin_layout Standard

\begin_inset Formula \begin{equation}
    x = \frac{1+\cos\theta}{2}
    \label{eq:changeofvariable}
\end{equation}
\end_inset


\end_layout

\begin_layout Standard
The integral changes to
\end_layout

\begin_layout Standard

\begin_inset Formula \begin{equation}
   I_{S,m} = \int_0^{1} dx x^{S+m}(1-x)^{S-m} 
   \label{eq:Ism}
\end{equation}
\end_inset


\end_layout

\begin_layout Standard
This integral can be calculated by using the generating function
\end_layout

\begin_layout Standard

\begin_inset Formula \begin{equation}
    f_S(z) = \sum_{n=0}^{2S}\frac{(2S)!}{(2S-n)!n!}I_{S,n-S}z^n
    \label{eq:Ism1}
\end{equation}
\end_inset


\end_layout

\begin_layout Standard
Plug 
\begin_inset CommandInset ref
LatexCommand eqref
reference "eq:Ism"
plural "false"
caps "false"
noprefix "false"

\end_inset

 into 
\begin_inset CommandInset ref
LatexCommand eqref
reference "eq:Ism1"
plural "false"
caps "false"
noprefix "false"

\end_inset

,
\end_layout

\begin_layout Standard

\begin_inset Formula \begin{align}
    f_S(z) & = \sum_{n=0}^{2S}\frac{(2S)!}{(2S-n)!n!}\bigg(\int_0^1 dx x^n (1-x)^{2S-n}\bigg)z^n \nonumber \\
           & = \int_0^1 dx \sum_{n=0}^{2S}\frac{(2S)!}{(2S-n)!n!} (zx)^n (1-x)^{2S-n} \nonumber \\
           & = \int_0^1 dx (zx + 1 -x)^{2S} \nonumber \\
           & = \int_0^1 dx ((z-1)x + 1)^{2S} \nonumber \\
           & = \frac{z^{2S+1}-1}{(2S+1)(z-1)} \nonumber \\
           & = \frac{1}{2S+1}(1+z + ... +z^{2S})
\end{align}
\end_inset


\end_layout

\begin_layout Standard
The integral 
\begin_inset Formula $I_{S,m}$
\end_inset

 is given by the coefficient of 
\begin_inset Formula $z^{S+m}$
\end_inset

 in 
\begin_inset Formula $f_S(z)$
\end_inset

,
\end_layout

\begin_layout Standard

\begin_inset Formula \begin{equation}
    \frac{(2S)!}{(S+m)!(S-m)!}I_{S,m} = \frac{1}{2S+1}
\end{equation}
\end_inset


\end_layout

\begin_layout Standard
Then
\end_layout

\begin_layout Standard

\begin_inset Formula \begin{equation}
    I_{S,m} = \frac{(S+m)!(S-m)!}{(2S+1)!}
\end{equation}
\end_inset


\end_layout

\begin_layout Standard
Back to the discussion of 
\begin_inset CommandInset ref
LatexCommand eqref
reference "eq:resolution"
plural "false"
caps "false"
noprefix "false"

\end_inset

, if we choose
\end_layout

\begin_layout Standard

\begin_inset Formula \[C = \frac{2S+1}{4\pi} \]
\end_inset


\end_layout

\begin_layout Standard
we obtain
\end_layout

\begin_layout Standard

\begin_inset Formula \begin{equation}
     \frac{2S+1}{4\pi}\int d\Hat{\Omega} \ket{\hat{\Omega}}\bra{\hat{\Omega}} = \sum_m \ket{S,m}\bra{S,m} = \mathbb{I}
     \label{eq:resolution}
\end{equation}
\end_inset


\end_layout

\begin_layout Standard
which defines the 
\shape italic
Haar measure
\shape default
 of the 
\begin_inset Formula $SU(2)$
\end_inset

 Lie group
\end_layout

\begin_layout Standard

\begin_inset Formula \begin{equation}
    \frac{2S+1}{4\pi}d\Hat{\Omega} = \frac{2S+1}{4\pi}d\theta\sin\theta d\phi
\end{equation}
\end_inset


\end_layout

\begin_layout Standard
We can prove another useful identity
\end_layout

\begin_layout Standard

\begin_inset Formula \begin{equation}
    \frac{(S+1)(2S+1)}{4\pi}\int d\hat{\Omega}\hat{\Omega}^{\alpha}\ket{\hat{\Omega}}\bra{\Hat{\Omega}} = \hat{S}^{\alpha} \quad \alpha =x,y,z
\end{equation}
\end_inset


\end_layout

\begin_layout Standard
In the space spanned by the eigenvectors of 
\begin_inset Formula $\hat{S}^z$
\end_inset

, the spin operators can be written as
\end_layout

\begin_layout Standard

\begin_inset Formula \begin{align}
    \hat{S}^z & = \sum_m m\ket{S,m}\bra{S,m} \nonumber \\
    \hat{S}^+ & = \sum_m \sqrt{(S-m)(S+m+1)}\ket{S,m+1}\bra{S,m} \nonumber \\
    \hat{S}^- & = \sum_m \sqrt{(S+m)(S-m+1)}\ket{S,m-1}\bra{S,m}
    \label{eq:defofSzplusminus}
\end{align}
\end_inset


\end_layout

\begin_layout Standard
The proof for the 
\begin_inset Formula $z$
\end_inset

-direction is straightforward. The integral written explicitly is
\end_layout

\begin_layout Standard

\begin_inset Formula \begin{align}
    & \frac{(S+1)(2S+1)}{2}\int_{-1}^1 d(\cos\theta) \cos\theta \sum_m\bigg(\frac{1+\cos\theta}{2}\bigg)^{S+m}\bigg(\frac{1-\cos\theta}{2}\bigg)^{S-m} \nonumber \\
   & \times \frac{(2S)!}{(S+m)!(S-m)!}\ket{S,m}\bra{S,m}
   \label{eq:theintegralofSz}
\end{align}
\end_inset


\end_layout

\begin_layout Standard
Using the same change of variable as 
\begin_inset CommandInset ref
LatexCommand eqref
reference "eq:changeofvariable"
plural "false"
caps "false"
noprefix "false"

\end_inset

, the integral changes to
\end_layout

\begin_layout Standard

\begin_inset Formula \begin{equation}
    (S+1)(2S+1)\int_0^1 dx (2x-1) x^{S+m}(1-x)^{S-m}
\end{equation}
\end_inset


\end_layout

\begin_layout Standard
Define a new type of integrals
\end_layout

\begin_layout Standard

\begin_inset Formula \begin{equation}
    \Tilde{I}_{S,m} = \int_0^1 dx x^{S+m+1} (1-x)^{S-m}
\end{equation}
\end_inset


\end_layout

\begin_layout Standard
To calculate this integral, we can introduce another generating function
\end_layout

\begin_layout Standard

\begin_inset Formula \begin{equation}
    \Tilde{f}_S(z) = \sum_{n=-1}^{2S}\frac{(2S+1)!}{(2S-n)!(n+1)!}\Tilde{I}_{S,n-S}z^{n+1}
\end{equation}
\end_inset


\end_layout

\begin_layout Standard
Following the same fashion, the function can be written as
\end_layout

\begin_layout Standard

\begin_inset Formula \begin{align}
    \Tilde{f}_S(z) & = \int_0^1 dx \sum_{n=-1}^{2S}\frac{(2S+1)!}{(2S-n)!(n+1)!}(zx)^{n+1}(1-x)^{2S-n} \nonumber \\
                   & = \int_0^1 dx ((z-1)+1)^{2S+1} \nonumber \\
                   & = \frac{z^{2(S+1)}-1}{2(S+1)(z-1)} \nonumber \\
                   & = \frac{1}{2(S+1)}(1 + z + ... + z^{2S+1})
\end{align}
\end_inset


\end_layout

\begin_layout Standard
The integral 
\begin_inset Formula $\Tilde{I}_{S,m}$
\end_inset

 is given by the coefficient of 
\begin_inset Formula $z^{m+S}$
\end_inset

, i.e. 
\begin_inset Formula $n = m+S$
\end_inset


\end_layout

\begin_layout Standard

\begin_inset Formula \begin{equation}
   \Tilde{I}_{S,m} = \frac{1}{2(S+1)}\frac{(S-m)!(S+m+1)!}{(2S+1)!}
\end{equation}
\end_inset


\end_layout

\begin_layout Standard
Then 
\begin_inset CommandInset ref
LatexCommand eqref
reference "eq:theintegralofSz"
plural "false"
caps "false"
noprefix "false"

\end_inset

 can be written as
\end_layout

\begin_layout Standard

\begin_inset Formula \begin{align}
    & \sum_m \bigg(2(S+1)(2S+1)\frac{1}{2(S+1)}\frac{(S-m)!(S+m+1)!}{(2S+1)!} \nonumber \\
    & \cdot\frac{(2S)!}{(S+m)!(S-m)!} - (S+1)\bigg)\ket{S,m}\bra{S,m} \nonumber \\
    & = \sum_m (S+m+1 -(S+1))\ket{S,m}\bra{S,m} \nonumber \\
    & = \sum_m m \ket{S,m}\bra{S,m}
\end{align}
\end_inset


\end_layout

\begin_layout Standard
Next, I will prove the identity for 
\begin_inset Formula $\hat{S}^+ = \hat{S}^x + i\hat{S}^y$
\end_inset

. I omitted one step when proving for the identity of 
\begin_inset Formula $\hat{S}^z$
\end_inset

. Let's now write them explicitly
\end_layout

\begin_layout Standard

\begin_inset Formula \begin{align}
    \int d\hat{\Omega}\ket{\hat{\Omega}}\bra{\hat{\Omega}} & = \int d\hat{\Omega}(2S)!\sum_{mm'}\frac{(u*)^{S+m'}(v^*)^{S-m'}u^{S+m}v^{S-m}}{\sqrt{(S+m')!(S-m')!(S+m)!(S-m)!}}\ket{S,m'}\bra{S,m} \nonumber \\
    & = \int_0^{2\pi}\int_0^{\pi}d\theta\sin\theta \sum_{mm'}e^{i(m-m')\phi}f(\theta) 
\end{align}
\end_inset


\end_layout

\begin_layout Standard
Where 
\begin_inset Formula $f(\theta)$
\end_inset

 is a function of 
\begin_inset Formula $\theta$
\end_inset

. When computing for 
\begin_inset Formula $\hat{S}^+$
\end_inset

, there will be an extra 
\begin_inset Formula $\cos\phi+i\sin\phi=e^{i\phi}$
\end_inset

, the non-vanishing 
\begin_inset Formula $\phi$
\end_inset

-integral gives us 
\begin_inset Formula $\delta_{m,m-1}$
\end_inset

, which matches the definition of 
\begin_inset CommandInset ref
LatexCommand eqref
reference "eq:defofSzplusminus"
plural "false"
caps "false"
noprefix "false"

\end_inset

, the details of proof are similar to that of 
\begin_inset Formula $\hat{S}^z$
\end_inset

. 
\begin_inset Newline newline
\end_inset


\end_layout

\begin_layout Standard
The above relations are easily extended to system of many spins. Consider a lattice of spins on 
\begin_inset Formula $\mathcal{N}$
\end_inset

 sites, which are labelled by 
\begin_inset Formula $i$
\end_inset

. The many-spin coherent states are products of the single spin states
\end_layout

\begin_layout Standard

\begin_inset Formula \begin{equation}
    \ket{\Hat{\Omega}} = \prod_{i=1}^{\mathcal{N}}\ket{\Hat{\Omega}_i}
\end{equation}
\end_inset


\end_layout

\begin_layout Standard
Their overlap is
\end_layout

\begin_layout Standard

\begin_inset Formula \begin{equation}
    \bra{\Hat{\Omega}}\ket{\Hat{\Omega}'} = \prod_i \bigg(\frac{1+\Hat{\Omega}_i\cdot\Hat{\Omega}'_i}{2}\bigg)^{S}e^{-iS\psi[\Hat{\Omega},\Hat{\Omega}']}
    \label{eq:overlap}
\end{equation}
\end_inset


\end_layout

\begin_layout Standard
The resolution of identity is
\end_layout

\begin_layout Standard

\begin_inset Formula \begin{equation}
     \int\prod_i\bigg(\frac{2S+1}{4\pi}\int d\Hat{\Omega}_i\bigg)\ket{\Hat{\Omega}}\bra{\Hat{\Omega}}=\mathbb{I}
      \label{eq:resolution1}
\end{equation}
\end_inset


\end_layout

\begin_layout Standard
I will present more properties of coherent states without proof (if I have time, I will finish this part). 
\begin_inset Newline newline
\end_inset


\end_layout

\begin_layout Standard
The spin coherent states is an eigenstate of the spin component in the 
\begin_inset Formula $\hat{\Omega}$
\end_inset

 direction
\end_layout

\begin_layout Standard

\begin_inset Formula \begin{equation}
    \hat{\Omega}\cdot\hat{\bm{S}}\ket{\hat{\Omega}} = S\ket{\hat{\Omega}}
\end{equation}
\end_inset


\end_layout

\begin_layout Standard
The expectation value of spin operator in coherent state is given by
\end_layout

\begin_layout Standard

\begin_inset Formula \begin{equation}
    \bra{\hat{\Omega}}\hat{\bm{S}}\ket{\hat{\Omega}} = S \hat{\Omega}
\end{equation}
\end_inset


\end_layout

\begin_layout Standard
The spin correlations of any wave function 
\begin_inset Formula $\Psi$
\end_inset

 can be computed by the integral
\end_layout

\begin_layout Standard

\begin_inset Formula \begin{equation}
    \frac{\bra{\Psi} \hat{\bm{S}}_i \cdot \hat{\bm{S}}_j \ket{\Psi}}{\bra{\Psi}\ket{\Psi}} = \frac{(S+1-\delta_{ij})(S+1)}{Z}\int\prod_i d\hat{\Omega}_i \abs{\Psi[\Hat{\Omega}]}^2 \Hat{\Omega}_i\cdot\Hat{\Omega}_j
\end{equation}
\end_inset


\end_layout

\begin_layout Standard

\begin_inset Formula \begin{equation}
    Z = \int \prod_i d\hat{\Omega}_i \abs{\Psi[\Hat{\Omega}]}^2 
\end{equation}
\end_inset


\end_layout

\begin_layout Standard

\begin_inset Formula \begin{equation}
    \Psi[\Hat{\Omega}] = \bra{\Psi}\ket{\hat{\Omega}}
\end{equation}
\end_inset


\end_layout

\begin_layout Standard
The coherent states actually elucidate the correspondence between classical and quantum spins. In the classical limit 
\begin_inset Formula $S \rightarrow \infty$
\end_inset

, the overlap of spin coherent states vanishes exponentially with 
\begin_inset Formula $S$
\end_inset

, according to 
\begin_inset CommandInset ref
LatexCommand eqref
reference "eq:overlap"
plural "false"
caps "false"
noprefix "false"

\end_inset

. The expectation value of spin operators are functions of unit vector, exactly like classical spins. Quantum effects are associated with the non-orthogonality of spin coherent states, furthermore, 
\begin_inset Formula $\Psi[\Hat{\Omega}]$
\end_inset

 has a finite width in 
\begin_inset Formula $\Hat{\Omega}$
\end_inset

 space, just like Schrödinger wave function. 
\begin_inset Newline newline
\end_inset


\end_layout

\begin_layout Standard
Last but not least, it's important to notice that the spin coherent states can be used to evaluate the trace of any operator. If 
\begin_inset Formula $\ket{n}$
\end_inset

 is any orthonormal basis, then
\end_layout

\begin_layout Standard

\begin_inset Formula \begin{align}
    \tr\mathcal{O} & = \sum_n \bra{n}\mathcal{O}\ket{n} \nonumber \\
                   & = \bigg(\frac{2S+1}{4\pi}\bigg)\int d\hat{\Omega} \sum_n \bra{n}\ket{\hat{\Omega}}\bra{\hat{\Omega}}\mathcal{O}\ket{n} \nonumber \\
                   & = \frac{2S+1}{4\pi}\int d\hat{\Omega} \sum_n \bra{\hat{\Omega}}\mathcal{O}\ket{n}\bra{n}\ket{\hat{\Omega}} \nonumber \\
                   & = \frac{2S+1}{4\pi}\int d\hat{\Omega} \bra{\hat{\Omega}}\mathcal{O} \ket{\hat{\Omega}}
                   \label{eq:trace}
\end{align}
\end_inset


\end_layout

\begin_layout Subsection
Spin response function
\end_layout

\begin_layout Standard
The spin response function describes the dynamics of the spins, in response to an arbitrary weak space-time dependent magnetic field (or source current) 
\begin_inset Formula $j_i^{\alpha}(t)$
\end_inset

 (not the dynamics of the system, controlled by the "observer"). The full Hamiltonian is a 
\series bold
functional
\series default
 of the source
\end_layout

\begin_layout Standard

\begin_inset Formula \begin{equation}
    \mathcal{H}[j] = \mathcal{H}_0 - \sum_{i,\alpha}j_i^{\alpha}S_i^{\alpha}
\end{equation}
\end_inset


\end_layout

\begin_layout Standard
where 
\begin_inset Formula $\alpha = x, y, z$
\end_inset

, and 
\begin_inset Formula $i$
\end_inset

 denotes the position of the source. 
\begin_inset Formula $\mathcal{H}_0(\mu)$
\end_inset

 is the non-interacting Hamiltonian in the grand canonical ensemble. All states in the Hilbert space evolve under the Schödinger equation
\end_layout

\begin_layout Standard

\begin_inset Formula \begin{equation}
    i\frac{\partial}{\partial t}\ket{\psi(t)} = \mathcal{H}[j]\ket{\psi(t)}
\end{equation}
\end_inset


\end_layout

\begin_layout Standard
The solution is given by the evolution operator
\end_layout

\begin_layout Standard

\begin_inset Formula \begin{align}
   \ket{\psi(t)} & = U[j(t)]\ket{\psi(0)} \nonumber \\
   U[j(t)] & = \mathcal{T}_{t'}\exp\bigg(-i\int_0^t dt'\mathcal{H}[j(t')]\bigg)
\end{align}
\end_inset


\end_layout

\begin_layout Standard
where 
\begin_inset Formula $\mathcal{T}_{t'}$
\end_inset

 is the 
\shape italic
time-ordered
\shape default
 operator with respect to 
\begin_inset Formula $t'$
\end_inset

. In most experiments, we study the dynamics by looking at the change in the expectation value (in the ensemble of the states governed by 
\begin_inset Formula $\mathcal{H}_0$
\end_inset

) in the presence of a "probe" (the source). The expectation value an operator evolves under 
\begin_inset Formula $\mathcal{H}[j]$
\end_inset

 is
\end_layout

\begin_layout Standard

\begin_inset Formula \begin{equation}
    \expval{S_i^{\alpha}(t)}_j = Z^{-1}\tr(\rho U^{-1}[j(t)]S_i^{\alpha}U[j(t)])
\end{equation}
\end_inset


\end_layout

\begin_layout Standard

\begin_inset Formula $\rho$
\end_inset

 is the density matrix governed by 
\begin_inset Formula $\mathcal{H}_0$
\end_inset

. The approximation to the first order in 
\begin_inset Formula $j$
\end_inset

 of 
\begin_inset Formula $U[j(t)]$
\end_inset

 is
\end_layout

\begin_layout Standard

\begin_inset Formula \begin{equation}
    U[j(t)] \simeq e^{-iH_0t}(\mathbb{I} + i\int_0^t dt'\sum_{i,\alpha}j_i^{\alpha}(t')S_i^{\alpha}(t'))
\end{equation}
\end_inset


\end_layout

\begin_layout Standard
Then
\end_layout

\begin_layout Standard

\begin_inset Formula \begin{align}
    \expval{S_i^{\alpha}(t)}_j & \simeq  Z^{-1} \tr\bigg(\rho e^{iH_0t}(\mathbb{I} - i\int_0^t dt'\sum_{i',\alpha'}j_{i'}^{\alpha'}(t')S_{i'}^{\alpha'}(t')) \nonumber \\
                               & \cdot S_i^{\alpha}e^{-iH_0t}(\mathbb{I} + i\int_0^t dt'\sum_{i',\alpha'}j_{i'}^{\alpha'}(t')S_{i'}^{\alpha'}(t'))\bigg) \nonumber \\
                               & = \expval{S_i^{\alpha}(0)} + i\int_0^t dt'\sum_{i',\alpha'}j_{i'}^{\alpha'}(t') \expval{\comm{S_i^{\alpha}(t)}{S_{i'}^{\alpha'}(t')}} + \mathcal{O}(j^2)
\end{align}
\end_inset


\end_layout

\begin_layout Standard
where I have assumed that 
\begin_inset Formula $\comm{S_i^{\alpha}}{\mathcal{H}_0} = 0$
\end_inset

. From the above derivation, we can extract the definition of the response function (retarded Green's function) as the kernel of the integral
\end_layout

\begin_layout Standard

\begin_inset Formula \begin{equation}
    R_{ii'}^{\alpha\alpha'}(t-t') = -i\theta(t-t') \expval{\comm{S_i^{\alpha}(t)}{S_{i'}^{\alpha'}(t')}}
\end{equation}
\end_inset


\end_layout

\begin_layout Standard
Then
\end_layout

\begin_layout Standard

\begin_inset Formula \begin{equation}
    \expval{\delta S_i^{\alpha}(t)} = \expval{S_i^{\alpha}(t)}_j - \expval{S_i^{\alpha}(0)} = \sum_{i',\alpha'}\int_0^{\infty}dt'R_{ii'}^{\alpha\alpha'}(t-t')j_{i'}^{\alpha'}(t')
\end{equation}
\end_inset


\end_layout

\begin_layout Standard
It's possible to express the response function in the eigenstates representation (Lehnmann representation) of 
\begin_inset Formula $\mathcal{H}_0$
\end_inset


\end_layout

\begin_layout Standard

\begin_inset Formula \begin{align}
    R_{ii'}^{\alpha\alpha'}(t) & = -i\theta(t)Z^{-1}\sum_{nm}e^{-\beta E_n}\bigg(e^{i(E_n-E_m)t}\bra{n}S_i^{\alpha}\ket{m}\bra{m}S_i^{\alpha'}\ket{n} \nonumber \\
                               & e^{i(E_m-E_n)t}\bra{n}S_{i'}^{\alpha'}\ket{m}\bra{m}S_i^{\alpha}\ket{n}\bigg)
\end{align}
\end_inset


\end_layout

\begin_layout Standard
Assuming the translational invariance of the Hamiltonian, 
\begin_inset Formula $R_{ii'} = R(\bm{x}_i - \bm{x}_i')$
\end_inset

, then we can define the Fourier transform in both space and time as
\end_layout

\begin_layout Standard

\begin_inset Formula \begin{align}
    R(\bm{q},\omega+i0^+) & = \mathcal{N}^{-1}\sum_{ii'}\int_0^{\infty}dte^{-i\bm{q}\cdot(\bm{x}_i-\bm{x}_i')+i(\omega+i0^+)t}R(\bm{x}_i - \bm{x}_i') \nonumber \\
                          & = (\mathcal{N}^{-1}Z)^{-1}\sum_{n,m}\bra{n}S_{\bm{q}}^{\alpha}\ket{m}\bra{m}S_{-\bm{q}}^{\alpha}\ket{n}\frac{e^{-\beta E_n} - e^{-\beta E_m}}{E_n - E_m + \omega + i0^+}
                          \label{eq:spectralresponse}
\end{align}
\end_inset


\end_layout

\begin_layout Standard
where
\end_layout

\begin_layout Standard

\begin_inset Formula \begin{equation}
    S_{\bm{q}}^{\alpha} = \sum_i e^{-i\bm{q}\cdot\bm{x}_i}S_i^{\alpha}
\end{equation}
\end_inset


\end_layout

\begin_layout Standard
Using the fluctuation-dissipation theorem, we can relate the retarded Green's function with the spectral function and then the greater and less correlation function.
\end_layout

\begin_layout Subsection
The imaginary-time generating functional
\end_layout

\begin_layout Standard
The grand canonical generating functional is defined as
\end_layout

\begin_layout Standard

\begin_inset Formula \begin{equation}
    Z[j,\mu] = \tr T_{\tau}\bigg(\exp{-\int_0^{\beta}d\tau\mathcal{H}[j(\tau)]}\bigg)
\end{equation}
\end_inset


\end_layout

\begin_layout Standard
where
\end_layout

\begin_layout Standard

\begin_inset Formula \begin{equation}
    \mathcal{H}[j(\tau)] = \mathcal{H}_0(\mu) - \sum_i j_i^{\alpha}(\tau)S_i^{\alpha}, \quad \tau \in [0,\beta)
\end{equation}
\end_inset


\end_layout

\begin_layout Standard
We assume 
\begin_inset Formula $\mathcal{H}_0$
\end_inset

 is symmetric under rotation, i.e. 
\begin_inset Formula $\comm{S_i^{\alpha}}{\mathcal{H}_0} = 0$
\end_inset

. The 
\shape italic
imaginary-time
\shape default
 Green's function is defined as
\end_layout

\begin_layout Standard

\begin_inset Formula \begin{align}
    \Tilde{R}_{ii'}^{\alpha\alpha'}(\tau,\tau') & \equiv \left.\frac{\delta^2 \ln Z}{\delta j_i^{\alpha}(\tau) \delta j_{i'}^{\alpha'}(\tau')}\right\vert_{j=0} \nonumber \\
                                                & = \frac{1}{Z}\left.\frac{\delta^2}{\delta j_i^{\alpha}(\tau) \delta j_{i'}^{\alpha'}(\tau')}\tr T_{\tau}\bigg(e^{-\beta\mathcal{H}_0}e^{\int_0^{\beta}d\tau \sum_i j_i^{\alpha}(\tau)S_i^{\alpha}}\bigg)\right\vert_{j=0} \nonumber \\
                                                & = \frac{1}{Z}\tr\bigg(e^{-\beta\mathcal{H}_0}T_{\tau}\big(S_i^{\alpha}(\tau)S_{i'}^{\alpha'}(\tau')\big)\bigg)
\end{align}
\end_inset


\end_layout

\begin_layout Standard
where
\end_layout

\begin_layout Standard

\begin_inset Formula \begin{equation}
    S_i^{\alpha}(\tau) = e^{\mathcal{H}_0\tau}S_i^{\alpha}e^{-\mathcal{H}_0\tau}
\end{equation}
\end_inset


\end_layout

\begin_layout Standard
and I have used the assumption 
\begin_inset Formula $\comm{S_i^{\alpha}}{\mathcal{H}_0} = 0$
\end_inset

 to add the imaginary-time dependence. First, we can verify that the 
\shape italic
imaginary-time
\shape default
 Green's function depends only on the imaginary time difference. Set 
\begin_inset Formula $\tau > \tau'$
\end_inset


\end_layout

\begin_layout Standard

\begin_inset Formula \begin{align}
  \tr\bigg(e^{-\beta\mathcal{H}_0} S_i^{\alpha}(\tau)S_{i'}^{\alpha'}(\tau')\bigg) & =  \tr\bigg(e^{-\beta\mathcal{H}_0} e^{\mathcal{H}_0\tau}S_i^{\alpha}e^{-\mathcal{H}_0\tau}e^{\mathcal{H}_0\tau'}S_{i'}^{\alpha'}e^{-\mathcal{H}_0\tau'} \bigg) \nonumber \\
  & = \tr\bigg(e^{-\beta\mathcal{H}_0}e^{\mathcal{H}_0(\tau-\tau')}S_i^{\alpha}e^{-\mathcal{H}_0(\tau-\tau')}S_{i'}^{\alpha'}\bigg) \nonumber \\ 
  & = \tr\bigg(e^{-\beta\mathcal{H}_0}S_i^{\alpha}(\tau-\tau')S_{i'}^{\alpha'}(0)\bigg)
\end{align}
\end_inset


\end_layout

\begin_layout Standard
where I have used the cyclic property of the trace. Then we have
\end_layout

\begin_layout Standard

\begin_inset Formula \[\Tilde{R}_{ii'}^{\alpha\alpha'}(\tau,\tau') = \Tilde{R}_{ii'}^{\alpha\alpha'}(\tau - \tau')\]
\end_inset


\end_layout

\begin_layout Standard
Furthermore, 
\begin_inset Formula $\Tilde{R}(\tau)$
\end_inset

 is periodic on the interval 
\begin_inset Formula $[0,\beta)$
\end_inset

. Check:
\end_layout

\begin_layout Standard

\begin_inset Formula \begin{equation}
    \lim_{\tau \rightarrow 0^-} \Tilde{R} = Z^{-1}\tr\bigg(e^{-\beta\mathcal{H}_0}S_{i'}^{\alpha}(0)S_i^{\alpha}(0^-)\bigg)
\end{equation}
\end_inset


\end_layout

\begin_layout Standard

\begin_inset Formula \begin{align}
    \lim_{\tau \rightarrow \beta^-} \Tilde{R} & = Z^{-1}\tr\bigg(e^{-\beta\mathcal{H}_0}S_i^{\alpha}(\beta^-)S_{i'}^{\alpha}(0)\bigg) \nonumber \\
                                    & = Z^{-1}\tr\bigg(e^{-\beta\mathcal{H}_0}e^{\beta\mathcal{H}_0}S_i^{\alpha}(^-)e^{-\beta\mathcal{H}_0}S_{i'}^{\alpha}(0)\bigg) \nonumber \\
                                    & = Z^{-1}\tr\bigg(e^{-\beta\mathcal{H}_0}S_{i'}^{\alpha}(0)S_i^{\alpha}(0^-)\bigg)
\end{align}
\end_inset


\end_layout

\begin_layout Standard

\begin_inset Formula \[\lim_{\tau \rightarrow 0^-} \Tilde{R}= \lim_{\tau \rightarrow \beta^-}\Tilde{R}\]
\end_inset


\end_layout

\begin_layout Standard
We further assume the Hamiltonian is translationally invariant, it's convenient to use the Fourier representation defined (legally by above discussions)
\end_layout

\begin_layout Standard

\begin_inset Formula \begin{equation}
    \Tilde{R}(\bm{q},i\omega_n) = \mathcal{N}^{-1}\sum_{ii'}\int_0^{\beta}d\tau e^{-i\bm{q}\cdot(\bm{x}_i-\bm{x}_{i'}-i\omega\tau}\Tilde{R}_{ii'}(\tau)
    \label{eq:imaspectral}
\end{equation}
\end_inset


\end_layout

\begin_layout Standard
where
\end_layout

\begin_layout Standard

\begin_inset Formula \begin{equation}
    \omega_n = \frac{2\pi n}{\beta} \quad n = 0,\pm 1,\pm2,...
\end{equation}
\end_inset


\end_layout

\begin_layout Standard
The frequencies 
\begin_inset Formula $\omega_n$
\end_inset

 are the 
\shape italic
Bose-Matsubara frequencies
\shape default
. 
\begin_inset Formula $\bm{q}$
\end_inset

 satisfies the periodic boundary condition defined by the number of sites 
\begin_inset Formula $\mathcal{N}$
\end_inset

 of the system. It's important to notice that the 
\shape italic
imaginary-time
\shape default
 Green's function can be obtained by an 
\series bold
analytic continuation
\series default
 of the spectral response function 
\begin_inset CommandInset ref
LatexCommand eqref
reference "eq:spectralresponse"
plural "false"
caps "false"
noprefix "false"

\end_inset

. This argument is more transparent by writing 
\begin_inset CommandInset ref
LatexCommand eqref
reference "eq:imaspectral"
plural "false"
caps "false"
noprefix "false"

\end_inset

 in eigenstates representation.
\end_layout

\begin_layout Standard

\begin_inset Formula \begin{equation}
    \left. \Tilde{R}(\bm{q},z) \right\vert_{z \rightarrow \omega+i0^+} = \Re R(\bm{q},\omega) + i\Im R(\bm{q},\omega)
\end{equation}
\end_inset


\end_layout

\begin_layout Standard
Finally, 
\series bold
static susceptibility
\series default
 is defined as the response of the magnetization at momentum 
\begin_inset Formula $\bm{q}$
\end_inset

 to an ordering field in the 
\begin_inset Formula $\alpha$
\end_inset

 direction, of wave vector 
\begin_inset Formula $\bm{q}$
\end_inset


\end_layout

\begin_layout Standard

\begin_inset Formula \begin{equation}
    \chi^{\alpha\alpha}(\bm{q}) = -\frac{1}{2}\Re R^{\alpha\alpha}(\bm{q},0)
\end{equation}
\end_inset


\end_layout

\begin_layout Section
The spin path integral
\end_layout

\begin_layout Subsection
Construction of path integral
\end_layout

\begin_layout Standard
The path integral approach provides formal expressions which lead to useful approximation schemes. Spin coherent states can be used to construct a path integral representation of some spin models. The imaginary-time generating functional is given by
\end_layout

\begin_layout Standard

\begin_inset Formula \begin{align}
    Z[j] & = \tr T_{\tau}\bigg(\exp{-\int_0^{\beta}d\tau\mathcal{H}[j(\tau)]}\bigg) \nonumber \\
         & = \lim_{N_{\epsilon}\rightarrow\infty}\tr T_{\tau} \prod_{n=0}^{N_{\epsilon}-1} (1-\epsilon\mathcal{H}(\tau_n))
\end{align}
\end_inset


\end_layout

\begin_layout Standard
where 
\begin_inset Formula $\epsilon = \beta/N_{\epsilon}$
\end_inset

 is the (imaginary) time step, and 
\begin_inset Formula $\tau_n = n\epsilon$
\end_inset

 is the discrete imaginary time. First of all, let's notice that the trace of any operator can be calculated by using the spin coherent states 
\begin_inset CommandInset ref
LatexCommand eqref
reference "eq:trace"
plural "false"
caps "false"
noprefix "false"

\end_inset

,
\end_layout

\begin_layout Standard

\begin_inset Formula \begin{equation}
    Z[j] = \lim_{N_{\epsilon}\rightarrow\infty} \int d\hat{\Omega}(0)  \bra{\hat{\Omega}(0)}T_{\tau}\prod_{n=0}^{N_{\epsilon}-1} (1-\epsilon\mathcal{H}(\tau_n))\ket{\hat{\Omega}(0)}
    \label{eq:spinpathintegralstarting}
\end{equation}
\end_inset


\end_layout

\begin_layout Standard
From now on 
\begin_inset Formula $d\hat{\Omega}$
\end_inset

 represents the full Haar measure including the prefactor. The boundary condition of the path integral is automatically set
\end_layout

\begin_layout Standard

\begin_inset Formula \begin{equation}
    \hat{\Omega}(\beta) = \hat{\Omega}(0)
\end{equation}
\end_inset


\end_layout

\begin_layout Standard
Insert 
\begin_inset Formula $N_{\epsilon}-1$
\end_inset

 resolution of identity 
\begin_inset CommandInset ref
LatexCommand eqref
reference "eq:resolution1"
plural "false"
caps "false"
noprefix "false"

\end_inset

, the (imaginary) time-ordering is taken care by the process of path integral,
\end_layout

\begin_layout Standard

\begin_inset Formula \begin{align}
    Z[j] & = \lim_{N_{\epsilon\rightarrow\infty}} \prod_{\tau,i}d \hat{\Omega}_{i,\tau} \prod_{\tau = \epsilon}^{\beta}\bra{\hat{\Omega}(\tau)}(1-\epsilon\mathcal{H}(\tau))\ket{\hat{\Omega}(\tau-\epsilon)} \nonumber \\
         & = \lim_{N_{\epsilon\rightarrow\infty}} \prod_{\tau,i}d \hat{\Omega}_{i,\tau} \prod_{\tau = \epsilon}^{\beta}\bra{\hat{\Omega}(\tau)}\ket{\hat{\Omega}(\tau-\epsilon)}(1-\epsilon H(\tau))
\end{align}
\end_inset


\end_layout

\begin_layout Standard
where 
\begin_inset Formula $\hat{\Omega} = (\hat{\Omega}_1, ..., \hat{\Omega}_{\mathcal{N}})$
\end_inset

 lives in a 
\begin_inset Formula $2\mathcal{N}$
\end_inset

-dimensional (two Euler angles for each site, number of site 
\begin_inset Formula $\mathcal{N}$
\end_inset

) parameter space. We have also defined the "classical Hamiltonian" as
\end_layout

\begin_layout Standard

\begin_inset Formula \begin{equation}
    H(\tau) = \frac{\bra{\hat{\Omega}(\tau)}\mathcal{H}(\tau)\ket{\hat{\Omega}(\tau-\epsilon)}}{\bra{\hat{\Omega}(\tau)}\ket{\hat{\Omega}(\tau-\epsilon)}}
\end{equation}
\end_inset


\end_layout

\begin_layout Standard
We should now evaluate 
\begin_inset Formula $\bra{\hat{\Omega}_i(\tau)}\ket{\hat{\Omega}_i(\tau-\epsilon)}$
\end_inset

 for site 
\begin_inset Formula $i$
\end_inset

. According to 
\begin_inset CommandInset ref
LatexCommand eqref
reference "eq:overlap"
plural "false"
caps "false"
noprefix "false"

\end_inset

, if the two coherent states are very close,
\end_layout

\begin_layout Standard

\begin_inset Formula \begin{equation}
    \hat{\Omega}_i(\tau) \cdot \hat{\Omega}_i(\tau-\epsilon) = 1
\end{equation}
\end_inset


\end_layout

\begin_layout Standard

\begin_inset Formula \begin{align}
    e^{-iS\psi_i} & \simeq 1 - iS\psi_i \nonumber \\
                & \simeq 1 - iS (\phi_i(\tau)-\phi_i(\tau-\epsilon))\cos(\theta_i) + \chi_i(\tau)-\chi_i(\tau-\epsilon)
\end{align}
\end_inset


\end_layout

\begin_layout Standard
keeping terms linear in the parameters (Euler angles). Introducing the (imaginary) time derivative
\end_layout

\begin_layout Standard

\begin_inset Formula \begin{equation}
    \Dot{\hat{\Omega}}(\tau) \equiv \frac{\hat{\Omega}(\tau + \epsilon) - \hat{\Omega}(\tau)}{\epsilon}
\end{equation}
\end_inset


\end_layout

\begin_layout Standard
This introduction has implicit assumption: 
\begin_inset Formula $Z$
\end_inset

 is dominated by paths that are smooth, i.e.
\begin_inset Formula $\abs{\Dot{\hat{\Omega}}} < \infty$
\end_inset

. This turns out to be 
\series bold
unjustified
\series default
. The mathematical ground of path integral is 
\series bold
shaky
\series default
. For this reason, path integral results should be checked whenever possible against operator methods. Nevertheless, let's accept the validity of the introduction. Then
\end_layout

\begin_layout Standard

\begin_inset Formula \begin{equation}
    e^{-iS\psi_i} \simeq 1 - iS \epsilon (\dot{\phi}_i \cos(\theta_i) + \dot{\chi}_i)
\end{equation}
\end_inset


\end_layout

\begin_layout Standard
The classical Hamiltonian can be evaluated at equal time
\end_layout

\begin_layout Standard

\begin_inset Formula \begin{equation}
    H[\hat{\Omega}(\tau)] = \frac{\bra{\hat{\Omega}(\tau)}\mathcal{H}(\tau)\ket{\hat{\Omega}(\tau)}}{\bra{\hat{\Omega}(\tau)}\ket{\hat{\Omega}(\tau)}}
\end{equation}
\end_inset


\end_layout

\begin_layout Standard
Define the integration measure as
\end_layout

\begin_layout Standard

\begin_inset Formula \begin{equation}
    D\hat{\Omega}(\tau) = \lim_{N_{\epsilon\rightarrow\infty}} \prod_{i,n}d\hat{\Omega}_i(\tau_n)
\end{equation}
\end_inset


\end_layout

\begin_layout Standard

\begin_inset Formula \begin{equation}
    Z[j] =  \int D\hat{\Omega}(\tau) \prod_{\tau=\epsilon}^{\beta} \big(\sum_i 1 - iS\epsilon(\dot{\phi}_i(\tau)\cos\theta_i(\tau) + \dot{\chi}_i(\tau)\big) (1 -\epsilon H[\hat{\Omega}(\tau)])
\end{equation}
\end_inset


\end_layout

\begin_layout Standard
Exponentiate the function and discard terms higher than 
\begin_inset Formula $\mathcal{O}(\epsilon)$
\end_inset

. The product 
\begin_inset Formula $\prod_{\tau=\epsilon}^{\beta}$
\end_inset

 between exponents are summation and can be changed into an integral in the continuum limit
\end_layout

\begin_layout Standard

\begin_inset Formula \begin{equation}
    Z[j] =  \oint D\hat{\Omega}(\tau) \exp\bigg(- iS \sum_i \int_0^{\beta} d\tau\big(\dot{\phi}_i(\tau)\cos\theta_i(\tau) + \dot{\chi}_i(\tau)\big) -\int_0^{\beta} d\tau H[\hat{\Omega}(\tau)] \bigg)
\end{equation}
\end_inset


\end_layout

\begin_layout Standard
Define the action as
\end_layout

\begin_layout Standard

\begin_inset Formula \begin{equation}
    \tilde{\mathcal{S}}[\hat{\Omega}] = - iS \sum_i \omega_i[\hat{\Omega}_i] + \int_0^{\beta} d\tau H[\hat{\Omega}(\tau)]
\end{equation}
\end_inset


\end_layout

\begin_layout Standard
Then
\end_layout

\begin_layout Standard

\begin_inset Formula \begin{equation}
    Z[j] = \oint D\hat{\Omega}(\tau) \exp(\tilde{\mathcal{S}}[\hat{\Omega}])
\end{equation}
\end_inset


\end_layout

\begin_layout Standard
the first of the action
\end_layout

\begin_layout Standard

\begin_inset Formula \begin{equation}
    \omega[\hat{\Omega}] = -\int_0^{\beta}d\tau \dot{\phi}\cos\theta + \dot{\chi}
\end{equation}
\end_inset


\end_layout

\begin_layout Standard
is pure 
\shape italic
geometric
\shape default
. And the second term is "
\shape italic
dynamic
\shape default
". Choose the gauge 
\begin_inset Formula $\chi = \text{const.}$
\end_inset

, i.e. 
\begin_inset Formula $\dot{\chi} = 0$
\end_inset

, the geometric phase is
\end_layout

\begin_layout Standard

\begin_inset Formula \begin{align}
    \omega[\hat{\Omega}] & = -\int_0^{\beta}d\tau \dot{\phi}\cos\theta \nonumber \\
                         & = -\int_{\phi_0}^{\phi_0} d\phi \cos\theta
\end{align}
\end_inset


\end_layout

\begin_layout Standard
I will prove that this term is in fact the 
\series bold
Berry phase
\series default
, which represents the area enclosed by the trajectory of the of 
\begin_inset Formula $\hat{\Omega}[\tau]$
\end_inset

 on a unit sphere. Let's observe that the shape enclosed by a change in 
\begin_inset Formula $d\phi_i$
\end_inset

 connected by the north pole is a spherical triangle, whose area is
\end_layout

\begin_layout Standard

\begin_inset Formula \begin{equation}
    d\omega' = \int_0^{\theta} d\theta\sin\theta d\phi = (1-\cos\theta)d\phi
\end{equation}
\end_inset


\end_layout

\begin_layout Standard
The total area is then
\end_layout

\begin_layout Standard

\begin_inset Formula \begin{equation}
    \omega' = \int_{\phi_0}^{\phi_0} d\phi (1-\cos\theta) = -\int_{\phi_0}^{\phi_0} d\phi \cos\theta = \omega
\end{equation}
\end_inset


\end_layout

\begin_layout Standard
The identity is satisfied if 
\begin_inset Formula $\phi(\beta)$
\end_inset

 does not cross the "date line" boundary 
\begin_inset Formula $\pm\pi$
\end_inset

. If the crossing is allowed, the two angles differ by 
\begin_inset Formula $2\pi$
\end_inset

. This corresponds to an overall factor of 
\begin_inset Formula $e^{2\pi S}= -1$
\end_inset

 for half-odd integer 
\begin_inset Formula $S$
\end_inset

. To resolve this difficulty, calculate the line integral which surrounds the date line as follows: at longitude 
\begin_inset Formula $\phi = -\pi+\epsilon$
\end_inset

 it goes from 
\begin_inset Formula $\theta$
\end_inset

 to the north pole, surrounds the north pole (cancels the angle difference), and comes down to 
\begin_inset Formula $\theta$
\end_inset

 on the other side of 
\begin_inset Formula $\phi = -\pi -\epsilon$
\end_inset

 (cancels the upper 
\begin_inset Formula $\theta$
\end_inset

-integral). It's useful to express the Berry phase in a gauge invariant form (without specifying the parameters)
\end_layout

\begin_layout Standard

\begin_inset Formula \begin{equation}
    \omega = \int_0^{\beta} d\tau \bm{A}[\hat{\Omega}]\dot{\hat{\Omega}}
\end{equation}
\end_inset


\end_layout

\begin_layout Standard

\begin_inset Formula $\bm{A}[\hat{\Omega}]$
\end_inset

 is a unit 
\series bold
magnetic monopole
\series default
 vector potential. By stokes theorem,
\end_layout

\begin_layout Standard

\begin_inset Formula \begin{equation}
    \nabla \times \bm{A} \cdot \hat{\Omega} = 1
    \label{eq:del}
\end{equation}
\end_inset


\end_layout

\begin_layout Standard
Since the total area of the unit sphere is 
\begin_inset Formula $4\pi$
\end_inset

 (solid angle 
\begin_inset Formula $4\pi*1=4\pi$
\end_inset

). Two standard choices for 
\begin_inset Formula $\bm{A}$
\end_inset

 are as following
\end_layout

\begin_layout Standard

\begin_inset Formula \begin{align}
    \bm{A}^a & = -\frac{\cos\theta}{\sin\theta}\hat{\phi} \nonumber \\
    \bm{A}^b & = \frac{1-\cos\theta}{\sin\theta}\hat{\phi} 
\end{align}
\end_inset


\end_layout

\begin_layout Standard
It's easy to check that they both satisfy 
\begin_inset CommandInset ref
LatexCommand eqref
reference "eq:del"
plural "false"
caps "false"
noprefix "false"

\end_inset

. The two gauges differ by the location of the singularities. For details see Auerbach. 
\begin_inset Newline newline
\end_inset


\end_layout

\begin_layout Standard
We can also write down the real-time Green's function, which describes the 
\shape italic
dynamics
\shape default
 at zero temperature. It is defined as
\end_layout

\begin_layout Standard

\begin_inset Formula \begin{equation}
    G(\hat{\Omega}_0,\hat{\Omega}_t;t) = \bra{\hat{\Omega}_t}T_{t'}\bigg(\exp\bigg[-i\int_0^t dt'\mathcal{H}(t')\bigg]\bigg)\ket{\hat{\Omega}_0}
\end{equation}
\end_inset


\end_layout

\begin_layout Standard
Follow the derivation of imaginary-time Green's function,
\end_layout

\begin_layout Standard

\begin_inset Formula \[\tau \rightarrow it' \quad t'\in[0,t]\]
\end_inset


\end_layout

\begin_layout Standard
We obtain the formal expression
\end_layout

\begin_layout Standard

\begin_inset Formula \begin{equation}
    G(\hat{\Omega}_0,\hat{\Omega}_t;t) = \int_{\hat{\Omega}_0}^{\hat{\Omega}_t}\mathcal{D}\hat{\Omega}(t')\exp(i\mathcal{S}[\hat{\Omega}])
\end{equation}
\end_inset


\end_layout

\begin_layout Standard
where 
\begin_inset Formula $\mathcal{S}$
\end_inset

 is the real-time action
\end_layout

\begin_layout Standard

\begin_inset Formula \begin{equation}
   \mathcal{S}[\Omega] = \int_0^t dt' \bigg(S\sum_i \bm{A}\cdot\dot{\hat{\Omega}} - H[\hat{\Omega}(t'),t']\bigg)
\end{equation}
\end_inset


\end_layout

\begin_layout Subsection
Large-
\begin_inset Formula $S$
\end_inset

 expansion
\end_layout

\begin_layout Standard
The path integral formalism gives us an intuitive way to study the asymptotic behavior of a spin system. As shown in 
\begin_inset CommandInset ref
LatexCommand eqref
reference "eq:overlap"
plural "false"
caps "false"
noprefix "false"

\end_inset

, the classical spin corresponds to the large-
\begin_inset Formula $S$
\end_inset

 limit of quantum spin. In this limit, the phase phase oscillates very fast for any path with 
\begin_inset Formula $\dot{\hat{\Omega}} \neq 0$
\end_inset

. It does not contribute to the Green's function. In another word, the classical Green's function can be written as
\end_layout

\begin_layout Standard

\begin_inset Formula \begin{equation}
    G(\hat{\Omega}_0,\hat{\Omega}_t;t) \simeq \int_{\hat{\Omega}_0}^{\hat{\Omega}_t}\mathcal{D}\hat{\Omega}(t')\exp(-iH[\hat{\Omega}(t')])
\end{equation}
\end_inset


\end_layout

\begin_layout Standard
which tells us that the pure classical spins do not gain Berry phase. At the same time, the magnitude of 
\begin_inset Formula $S$
\end_inset

 can be used as a controlling parameter for semi-classical expansions. Introducing the following scaling transform
\end_layout

\begin_layout Standard

\begin_inset Formula \[\Bar{t}' = S^{-1}\Bar{t}' \quad \Bar{t} = S^{-1}\Bar{t}\]
\end_inset


\end_layout

\begin_layout Standard
The classical Hamiltonian 
\begin_inset Formula $H[\hat{\Omega}(t')]$
\end_inset

 is usually quadratic in 
\begin_inset Formula $\hat{\Omega}(t')$
\end_inset

, then the parameter 
\begin_inset Formula $S$
\end_inset

 can be scaled out of the action with
\end_layout

\begin_layout Standard

\begin_inset Formula \begin{align}
    G(\hat{\Omega}_0,\hat{\Omega}_{\bar{t}};\bar{t}) & = \int_{\hat{\Omega}_0}^{\hat{\Omega}_{\bar{t}}}\mathcal{D}\hat{\Omega}(\bar{t}')\exp(iS\mathcal{S}^{cl}[\hat{\Omega}]) \nonumber \\
    \mathcal{S}^{cl}[\hat{\Omega}] & = \int_0^{\Bar{t}}d\Bar{t}'( \sum_i \bm{A}\cdot\dot{\hat{\Omega}} - H[\hat{\Omega}(\bar{t}')])
    \label{eq:contourMSD}
\end{align}
\end_inset


\end_layout

\begin_layout Standard
where I have used the superscript for the action to distinguish it from the magnitude of spin. We assume the path integral 
\begin_inset CommandInset ref
LatexCommand eqref
reference "eq:contourMSD"
plural "false"
caps "false"
noprefix "false"

\end_inset

 satisfies the requirement of 
\series bold
MSD
\series default
, since 
\begin_inset Formula $S$
\end_inset

 is large. Then the Green's function can be approximated by
\end_layout

\begin_layout Standard

\begin_inset Formula \begin{equation}
    G \simeq \sum_{\alpha} \exp(iS\mathcal{S}[\Hat{\Omega}^{cl,\alpha},\Bar{t}]) G_{\alpha}'
    \label{eq:dynamicgreen}
\end{equation}
\end_inset


\end_layout

\begin_layout Standard
where 
\begin_inset Formula $\Hat{\Omega}^{cl,\alpha}(t')$
\end_inset

 are determined by the saddle point equations
\end_layout

\begin_layout Standard

\begin_inset Formula \begin{equation}
    \left. \frac{\delta \mathcal{S}[\hat{\Omega}]}{\delta\Hat{\Omega}}\right\rvert_{\hat{\Omega}^{cl,\alpha}}  = 0
\end{equation}
\end_inset


\end_layout

\begin_layout Standard
According to my understanding, 
\begin_inset Formula $\hat{\Omega}^{cl}(t')$
\end_inset

 are the saddle points of the action 
\begin_inset Formula $\mathcal{S}$
\end_inset

 at every time, the union of them forms the classical paths of the spins. 
\begin_inset Formula $\alpha$
\end_inset

 indicates that there are multiple possible paths since the actions can have multiple saddle points. One possible confusing (at least for me) comes from the derivation of 
\series bold
MSD
\series default
: we have used the Cauchy theorem to deform the contour (path) because the integrand is analytic in the complex plane; however, I claimed I found a classical path by solving the saddle point equations. The crucial observation is that the two paths are different! The former paths have no time-dependence, the union of them are just the allowed domain of 
\begin_inset Formula $\hat{\Omega}$
\end_inset

 as a complex variable, the contour 
\begin_inset Formula $\mathcal{C}'$
\end_inset

 is actually the real axis. While the latter path is actually the trajectory of spins on the unit sphere, in large-
\begin_inset Formula $S$
\end_inset

 limit, the classical trajectory which extreme the action contributes most because all other trajectories cancel with each other due to the large oscillation. 
\begin_inset Newline newline
\end_inset


\end_layout

\begin_layout Standard
The factor 
\begin_inset Formula $G_{\alpha}'$
\end_inset

 in 
\begin_inset CommandInset ref
LatexCommand eqref
reference "eq:dynamicgreen"
plural "false"
caps "false"
noprefix "false"

\end_inset

 contains the higher order corrections in powers of 
\begin_inset Formula $S^{-1}$
\end_inset

. They can be evaluated by expanding the fluctuation integrals
\end_layout

\begin_layout Standard

\begin_inset Formula \begin{equation}
   G_{\alpha}' = \int_{\hat{\Omega}_0}^{\hat{\Omega}_{\bar{t}}}\mathcal{D}\delta\hat{\Omega}(\bar{t}')\exp\bigg(iS (\mathcal{S}^{cl}[\hat{\Omega}] - \mathcal{S}^{cl}[\hat{\Omega}^{cl,\alpha}])\bigg)
\end{equation}
\end_inset


\end_layout

\begin_layout Standard
where
\end_layout

\begin_layout Standard

\begin_inset Formula \begin{equation}
    \delta\hat{\Omega} = \hat{\Omega} - \hat{\Omega}^{cl,\alpha}
\end{equation}
\end_inset


\end_layout

\begin_layout Standard
The semi-classical dynamics can be derived from the path integral formalism to check with the classical equations of motion. The boundary conditions are chosen to be
\end_layout

\begin_layout Standard

\begin_inset Formula \begin{equation}
    \hat{\Omega}^{cl,\alpha}(0) = \hat{\Omega}_0 \quad \hat{\Omega}^{cl,\alpha}(t) = \hat{\Omega}_t
    \label{eq:boundary}
\end{equation}
\end_inset


\end_layout

\begin_layout Standard
The variation of the Berry phase in the action is given by
\end_layout

\begin_layout Standard

\begin_inset Formula \begin{align}
    \delta\omega[\hat{\Omega}] & = \int_0^t dt' \delta(\bm{A}\cdot\dot{\hat{\Omega}}) \nonumber \\
                               & = \int_0^t dt' \bigg(\frac{\partial A^{\alpha}}{\partial\hat{\Omega}^{\beta}}\delta\hat{\Omega}^{\beta}\dot{\hat{\Omega}}^{\alpha} + A^{\alpha}\dv{\delta\hat{\Omega}^{\beta}}{t'} \nonumber \\
                               & + \frac{\partial A^{\alpha}}{\partial\hat{\Omega}^{\beta}}\dot{\hat{\Omega}}^{\beta}\delta\hat{\Omega}^{\alpha} - \frac{\partial A^{\alpha}}{\partial\hat{\Omega}^{\beta}}\dot{\hat{\Omega}}^{\beta}\delta\hat{\Omega}^{\alpha}\bigg) \nonumber \\
                               & = \int_0^t dt' \frac{\partial A^{\alpha}}{\partial\hat{\Omega}^{\beta}} \varepsilon^{\alpha\beta\gamma}(\dot{\hat{\Omega}} \times \delta \hat{\Omega})_{\gamma} + \int_0^t dt'\dv{}{t'}\bigg(\bm{A}\cdot\delta\hat{\Omega}\bigg) \nonumber \\
                               & = \int_0^t dt' \hat{\Omega}\cdot(\dot{\hat{\Omega}} \times \delta \hat{\Omega})
                               \label{eq:changeinberry}
\end{align}
\end_inset


\end_layout

\begin_layout Standard
The second term in the second last step vanishes because of the boundary condition 
\begin_inset CommandInset ref
LatexCommand eqref
reference "eq:boundary"
plural "false"
caps "false"
noprefix "false"

\end_inset

. The final result can be derived from the first term of the second last step by noticing the relation 
\begin_inset CommandInset ref
LatexCommand eqref
reference "eq:del"
plural "false"
caps "false"
noprefix "false"

\end_inset

 and the unit length of 
\begin_inset Formula $\Hat{\Omega}$
\end_inset

. The classical 
\shape italic
Euler-Lagrange
\shape default
 equations of motion are now
\end_layout

\begin_layout Standard

\begin_inset Formula \begin{equation}
    \hat{\Omega}^{cl} \times \dot{\hat{\Omega}}^{cl} = \frac{\partial H[\hat{\Omega}^{cl}]}{\partial\hat{\Omega}}
\end{equation}
\end_inset


\end_layout

\begin_layout Standard
and
\end_layout

\begin_layout Standard

\begin_inset Formula \begin{equation}
    \dot{\hat{\Omega}}^{cl} = \hat{\Omega}^{cl} \times \frac{\partial H[\hat{\Omega}^{cl}]}{\partial\hat{\Omega}}
\end{equation}
\end_inset


\end_layout

\begin_layout Standard
I have used the cyclic properties of combination of cross product and dot product. This is the classical equations of motion for "fast top" rotors (when the internal rotational energy is much larger than the interaction energy between rotors). The rotors just change the direction of the angular momentum not the magnitude.
\end_layout

\begin_layout Subsection
Spin wave theory
\end_layout

\begin_layout Standard
In the previous section, I derived the classical equations of motion of a spin system by using saddle points expansion (classical path). In this section, I will study the thermodynamics/dynamics of spin fluctuations in the limit 
\begin_inset Formula $\abs{\delta\hat{\Omega}} \ll 1$
\end_inset

. Assume the saddle points are given as 
\begin_inset Formula $\hat{\Omega}^{cl}$
\end_inset

, they are fixed points on the unit sphere. Due to fluctuations, 
\begin_inset Formula $\hat{\Omega} = \hat{\Omega}^{cl} + \delta\hat{\Omega}$
\end_inset

 oscillates near the saddle points. Let's study the path integral of 
\begin_inset Formula $\delta\hat{\Omega}$
\end_inset

.
\end_layout

\begin_layout Standard

\begin_inset Formula \begin{equation}
   G' = \int_{\hat{\Omega}_0}^{\hat{\Omega}_{\bar{t}}}\mathcal{D}\delta\hat{\Omega}(\bar{t}')\exp\bigg(iS (\mathcal{S}[\hat{\Omega}] - \mathcal{S}_0[\hat{\Omega}^{cl}])\bigg)
\end{equation}
\end_inset


\end_layout

\begin_layout Standard
where 
\begin_inset Formula \[\mathcal{S}_0 = \mathcal{S}^{cl}[\hat{\Omega}^{cl}]\]
\end_inset


\end_layout

\begin_layout Standard
To begin with, introduce two (transverse) parameters to parametrize the spin fluctuations,
\end_layout

\begin_layout Standard

\begin_inset Formula \begin{equation}
  \hat{\Omega}^{cl}_i = \hat{\phi}_i \times \hat{\theta}_i
\end{equation}
\end_inset


\end_layout

\begin_layout Standard
Define
\end_layout

\begin_layout Standard

\begin_inset Formula \begin{align}
    \bm{q} & = \{q_i\}_1^{\mathcal{N}} = \{\delta\hat{\Omega}_i\cdot\hat{\phi}_i\}_1^{\mathcal{N}} \nonumber \\
    \bm{p} & = \{p_i\}_1^{\mathcal{N}} = \{S\delta\hat{\Omega}_i\cdot\hat{\theta}_i\}_1^{\mathcal{N}}
\end{align}
\end_inset


\end_layout

\begin_layout Standard
The integration measure changes as following
\end_layout

\begin_layout Standard

\begin_inset Formula \begin{equation}
    \mathcal{D}\delta\hat{\Omega} \rightarrow \nu \mathcal{D}\bm{p}\mathcal{D}\bm{q} 
\end{equation}
\end_inset


\end_layout

\begin_layout Standard
where 
\begin_inset Formula $\nu$
\end_inset

 is the Jacobian of the transform with some normalization constant. The deviation of the action 
\begin_inset Formula $(\mathcal{S}[\hat{\Omega}] - \mathcal{S}_0[\hat{\Omega}^{cl}])$
\end_inset

 consists of two parts: Berry phase and classical Hamiltonian (Hamiltonian evaluated between spin coherent states). The change in Berry phase is given in 
\begin_inset CommandInset ref
LatexCommand eqref
reference "eq:changeinberry"
plural "false"
caps "false"
noprefix "false"

\end_inset


\end_layout

\begin_layout Standard

\begin_inset Formula \begin{align}
    \delta\omega[\hat{\Omega}] & = \frac{1}{2}\int_0^t dt'\hat{\Omega}\cdot(\dot{\hat{\Omega}}\times\delta\hat{\Omega}) \nonumber \\
                               & = \frac{1}{2}\int_0^t dt'(\hat{\Omega}^{cl} + \delta\hat{\Omega}) \cdot (\delta\dot{\hat{\Omega}} \times \delta\hat{\Omega}) \quad \dot{\hat{\Omega}}^{cl} = 0 \nonumber \\
                               & = \frac{1}{2}\int_0^t dt' \hat{\Omega}^{cl} \cdot (\delta\dot{\hat{\Omega}} \times \delta\hat{\Omega}) + \mathcal{O}(\delta\hat{\Omega})^3
                               \label{eq:berrychangefirst}
\end{align}
\end_inset


\end_layout

\begin_layout Standard
where the 
\begin_inset Formula $1/2$
\end_inset

 term is included because the action is expanded up to the second order. Write 
\begin_inset CommandInset ref
LatexCommand eqref
reference "eq:berrychangefirst"
plural "false"
caps "false"
noprefix "false"

\end_inset

 in terms of parameters, notice the vector identity
\end_layout

\begin_layout Standard

\begin_inset Formula \begin{equation}
    (\bm{A}\times\bm{B})\cdot(\bm{C}\times\bm{D}) = (\bm{A}\cdot\bm{C})(\bm{B})\cdot\bm{D}) - (\bm{A}\cdot\bm{D})(\bm{B}\cdot\bm{C})
\end{equation}
\end_inset


\end_layout

\begin_layout Standard

\begin_inset Formula \begin{align}
    S\delta\omega[\hat{\Omega}] & = S\frac{1}{2}\int_0^t dt' \hat{\Omega}^{cl} \cdot (\delta\dot{\hat{\Omega}} \times \delta\hat{\Omega}) \nonumber \\
                                & = S\frac{1}{2}\int_0^t dt'\sum_i (\hat{\phi}_i \times \hat{\theta}_i) \cdot (\delta\dot{\hat{\Omega}}_i \times \delta\hat{\Omega}_i) \nonumber \\
                                & = S\frac{1}{2}\int_0^t dt'\sum_i (\hat{\phi}_i \cdot \delta\dot{\hat{\Omega}}_i)( \hat{\theta}_i\cdot \delta\dot{\hat{\Omega}}_i) - (\hat{\phi}_i \cdot \delta\hat{\Omega}_i)(\hat{\theta}_i\cdot\delta\dot{\hat{\Omega}}_i) \nonumber \\
                                & = \frac{1}{2}\int_0^t dt'\sum_i (\dot{q}_ip_i - q_i\dot{p}_i) \nonumber \\
                                & = \frac{1}{2}\int_0^t dt' (\dot{\bm{q}}\bm{p} - \bm{q}\dot{\bm{p}})
                                \label{eq:kineticberry}
\end{align}
\end_inset


\end_layout

\begin_layout Standard
Next, focus on the change in the classical Hamiltonian
\end_layout

\begin_layout Standard

\begin_inset Formula \begin{equation}
    H[\hat{\Omega}] - H [\hat{\Omega}^{cl}] \simeq \frac{1}{2} \begin{pmatrix} \bm{q} & \bm{p} \end{pmatrix} H^{(2)} \begin{pmatrix}
    \bm{q} \\
    \bm{p}
    \end{pmatrix}
\end{equation}
\end_inset


\end_layout

\begin_layout Standard

\begin_inset Formula $H^{(2)}$
\end_inset

 is defined as a four-component block matrix
\end_layout

\begin_layout Standard

\begin_inset Formula \begin{equation}
   H^{(2)} = \begin{pmatrix}
    K & P \\
    P^T & M^{-1}
    \end{pmatrix}
\end{equation}
\end_inset


\end_layout

\begin_layout Standard
The block elements are defined through partial derivations, for instance
\end_layout

\begin_layout Standard

\begin_inset Formula \begin{equation}
    K = \left.\frac{\partial^2 H}{\partial \bm{q}\partial\bm{q}}\right\vert_{\bm{q}=\bm{p}=0}
\end{equation}
\end_inset


\end_layout

\begin_layout Standard
Now we can find that 
\begin_inset CommandInset ref
LatexCommand eqref
reference "eq:kineticberry"
plural "false"
caps "false"
noprefix "false"

\end_inset

 corresponds to the "kinetic energy", while the classical Hamiltonian corresponds to the "potential energy". As in classical mechanics, we can define the Lagrangian for spin fluctuations
\end_layout

\begin_layout Standard

\begin_inset Formula \begin{equation}
    \mathcal{L}^{(2)} = S\delta\omega[\hat{\Omega}]  - (H[\hat{\Omega}] - H [\hat{\Omega}^{cl}])
\end{equation}
\end_inset


\end_layout

\begin_layout Standard
in matrix form
\end_layout

\begin_layout Standard

\begin_inset Formula \begin{equation}
   \mathcal{L}^{(2)} =  - \begin{pmatrix}
    K & P + \partial_t \\
    P^T - \partial_t & M^{-1}
    \end{pmatrix}
\end{equation}
\end_inset


\end_layout

\begin_layout Standard
The fluctuation path integral up to the second order can be written as
\end_layout

\begin_layout Standard

\begin_inset Formula \begin{equation}
    G' = \int \mathcal{D}\bm{p}\mathcal{D}\bm{q} \exp\bigg(\frac{iS}{2}\int_0^t dt' \begin{pmatrix} \bm{q} & \bm{p} \end{pmatrix}  \mathcal{L}^{(2)} \begin{pmatrix}
    \bm{q} \\
    \bm{p}
    \end{pmatrix}\bigg)
\end{equation}
\end_inset


\end_layout

\begin_layout Standard
To obtain the dynamics, first perform a Fourier transform, then apply the saddle point constraint
\end_layout

\begin_layout Standard

\begin_inset Formula \begin{equation}
    \frac{\delta \mathcal{S}^{(2)}}{\delta \hat{\omega}} = 0
\end{equation}
\end_inset


\end_layout

\begin_layout Standard
which tells us that
\end_layout

\begin_layout Standard

\begin_inset Formula \begin{equation}
   \mathcal{L}^{(2)} \begin{pmatrix}
    \bm{q}_{\bm{k},\alpha} \\
    \bm{p}_{\bm{k},\alpha}
    \end{pmatrix} e^{i\omega_{\bm{k},\alpha}}
\end{equation}
\end_inset


\end_layout

\begin_layout Standard
where 
\begin_inset Formula $\bm{k}$
\end_inset

 are Block wave vectors and 
\begin_inset Formula $\alpha$
\end_inset

 is the band index. Then we obtain the dynamical matrix
\end_layout

\begin_layout Standard

\begin_inset Formula \begin{equation}
    D = \begin{pmatrix}
    K & i\omega + P \\
    -i\omega + P^T & M^{-1} 
    \end{pmatrix}
\end{equation}
\end_inset


\end_layout

\begin_layout Standard
To guarantee non-trivial solutions, require 
\begin_inset Formula $\det D = 0$
\end_inset

, which yields the normal modes 
\begin_inset Formula $\omega_{\bm{k},\alpha}$
\end_inset

. We can also find the thermodynamics
\end_layout

\begin_layout Section
SU(
\begin_inset Formula $N$
\end_inset

) Heisenberg models
\end_layout

\begin_layout Standard
In this section, I will review the formalism of large 
\begin_inset Formula $N$
\end_inset

 approach to the quantum Heisenberg model. This method provides an additional avenue to the static and dynamical correlations of the quantum magnets. The parameters 
\begin_inset Formula $N$
\end_inset

 labels an 
\shape italic
internal
\shape default
 
\begin_inset Formula $SU(N)$
\end_inset

 symmetry at each lattice site (i.e. the number of "flavors" Schwinger bosons or constrained fermions a site can have). In most cases, the large 
\begin_inset Formula $N$
\end_inset

 approximation has been applied to treat spin Hamiltonian, where the symmetry is 
\begin_inset Formula $SU(2)$
\end_inset

 in Heisenberg model, and therefore 
\begin_inset Formula $N$
\end_inset

 is not a truly large number. Nevertheless, the 
\begin_inset Formula $1/N$
\end_inset

 expansion provides an easy method for obtaining simple mean field theories and the corrections can be calculated by Feynman diagrams.
\end_layout

\begin_layout Standard
Depending on the sign of coupling, size of spin, and lattice, we will introduce "suitable models". By "suitable", we mean that the corresponding mean field theory should produce correct qualitative results of the classical ground states.
\end_layout

\begin_layout Subsection
Schwinger bosons for ferromagnets
\end_layout

\begin_layout Standard
The physical subspace of the Fock space of Schwinger bosons are given by the constraints
\end_layout

\begin_layout Standard

\begin_inset Formula \begin{equation}
    a_i^{\dagger}a_i + b_i^{\dagger}b_i = 2S
    \label{eq:subFock}
\end{equation}
\end_inset


\end_layout

\begin_layout Standard
which defines the spin size of the system. The nearest neighbor ferromagnetic Heisenberg model (HFM) of spin 
\begin_inset Formula $S$
\end_inset

 is written as
\end_layout

\begin_layout Standard

\begin_inset Formula \begin{equation}
    \mathcal{H} = - J\sum_{<ij>}\bm{S}_i\cdot\bm{S}_j
\end{equation}
\end_inset


\end_layout

\begin_layout Standard
I will try to write down the Schwinger boson representation of spin operators given in 
\begin_inset CommandInset ref
LatexCommand eqref
reference "eq:Schwingerepresentation"
plural "false"
caps "false"
noprefix "false"

\end_inset

. First of all, notice that the Hamiltonian is bi-quadratic (four) in the Schwinger boson creation/annihilation operators. I claim that the 
\series bold
bond operator
\series default
 defined as
\end_layout

\begin_layout Standard

\begin_inset Formula \begin{equation}
    \mathcal{F}_{ij} = a_i^{\dagger}a_j + b_i^{\dagger}b_j
\end{equation}
\end_inset


\end_layout

\begin_layout Standard
can produce meaningful results for HFM. Write the quadratic operator explicitly,
\end_layout

\begin_layout Standard

\begin_inset Formula \begin{equation}
   \mathcal{F}_{ij}^{\dagger}\mathcal{F}_{ij} = \underbrace{a_j^{\dagger}a_ia_i^{\dagger}a_j + b_j^{\dagger}b_ib_i^{\dagger}b_j}_\text{\rom{1}} + \underbrace{a_j^{\dagger}a_ib_i^{\dagger}b_j + b_j^{\dagger}b_ia_i^{\dagger}a_j}_\text{\rom{2}}
   \label{eq:bondFM}
\end{equation}
\end_inset


\end_layout

\begin_layout Standard
To relate it with the spin operators, we need to prove several identities.
\end_layout

\begin_layout Standard

\begin_inset Formula \begin{equation}
    a_i^{\dagger}a_j^{\dagger}a_ia_j + b_i^{\dagger}b_j^{\dagger}b_ib_j = 2(S^2 + S_i^zS_j^z)
    \label{eq:identity1}
\end{equation}
\end_inset


\end_layout

\begin_layout Standard

\begin_inset ERT
status collapsed

\begin_layout Plain Layout

\backslash
begin{proof}
\end_layout

\end_inset

 
\begin_inset Formula \begin{align}
       S_i^zS_j^z & = \frac{1}{4}(a_i^{\dagger}a_j^{\dagger}a_ia_j + b_i^{\dagger}b_j^{\dagger}b_ib_j \nonumber \\
                  & - a_i^{\dagger}b_j^{\dagger}a_ib_j - b_i^{\dagger}a_j^{\dagger}a_ib_j)
   \end{align}
\end_inset


\end_layout

\begin_layout Standard
The constraint 
\begin_inset CommandInset ref
LatexCommand eqref
reference "eq:subFock"
plural "false"
caps "false"
noprefix "false"

\end_inset

 tells us that
\end_layout

\begin_layout Standard

\begin_inset Formula \begin{equation}
       S^2 = \frac{1}{4}(a_i^{\dagger}a_i + b_i^{\dagger}b_i)(a_j^{\dagger}a_j + b_j^{\dagger}b_j)
   \end{equation}
\end_inset


\end_layout

\begin_layout Standard
Then
\end_layout

\begin_layout Standard

\begin_inset Formula \begin{equation}
    a_i^{\dagger}a_j^{\dagger}a_ia_j + b_i^{\dagger}b_j^{\dagger}b_ib_j = 2(S^2 + S_i^zS_j^z)
\end{equation}
\end_inset


\shape italic
Q.E.D.
\shape default
 
\begin_inset ERT
status collapsed

\begin_layout Plain Layout

\backslash
end{proof}
\end_layout

\end_inset


\end_layout

\begin_layout Standard
The product between two spin operators can be written as
\end_layout

\begin_layout Standard

\begin_inset Formula \begin{equation}
    \bm{S}_i\cdot\bm{S}_j = \frac{1}{2}(S_i^+S_j^- + S_i^-S_j^+) + S_i^zS_j^z
\end{equation}
\end_inset


\end_layout

\begin_layout Standard
Notice that the second term in 
\begin_inset CommandInset ref
LatexCommand eqref
reference "eq:bondFM"
plural "false"
caps "false"
noprefix "false"

\end_inset

 equals to the product of spin creation/annihilation operators
\end_layout

\begin_layout Standard

\begin_inset Formula \begin{equation}
    \text{\rom{2}} = S_i^+S_j^- + S_i^-S_j^+
\end{equation}
\end_inset


\end_layout

\begin_layout Standard
The first term of 
\begin_inset CommandInset ref
LatexCommand eqref
reference "eq:bondFM"
plural "false"
caps "false"
noprefix "false"

\end_inset

 can be rewritten using the commutation relation of boson operators
\end_layout

\begin_layout Standard

\begin_inset Formula \begin{align}
    \text{\rom{1}} & = a_j^{\dagger}a_i^{\dagger}a_ia_j + b_j^{\dagger}b_i^{\dagger}b_ib_j + a_j^{\dagger}a_j + b_j^{\dagger}b_j \nonumber \\
                   & = 2S + a_j^{\dagger}a_i^{\dagger}a_ia_j + b_j^{\dagger}b_i^{\dagger}b_ib_j \nonumber \\
                   & = 2S(S+1) + 2S_i^zS_j^z
\end{align}
\end_inset


\end_layout

\begin_layout Standard
where I have used the constraint 
\begin_inset CommandInset ref
LatexCommand eqref
reference "eq:subFock"
plural "false"
caps "false"
noprefix "false"

\end_inset

 and the identity 
\begin_inset CommandInset ref
LatexCommand eqref
reference "eq:identity1"
plural "false"
caps "false"
noprefix "false"

\end_inset

. Then
\end_layout

\begin_layout Standard

\begin_inset Formula \begin{equation}
    \mathcal{F}_{ij}^{\dagger}\mathcal{F}_{ij} = \text{\rom{1}} + \text{\rom{2}} = 2S(S+1) + 2S_i^zS_j^z + S_i^+S_j^- + S_i^-S_j^+
\end{equation}
\end_inset


\end_layout

\begin_layout Standard
The nearest neighbor HFM Hamiltonian can be written as
\end_layout

\begin_layout Standard

\begin_inset Formula \begin{align}
    \mathcal{H} & = - J\sum_{<ij>}\bm{S}_i\cdot\bm{S}_j \nonumber \\
                & = -\frac{J}{2}\sum_{<ij>}(\mathcal{F}_{ij}^{\dagger}\mathcal{F}_{ij} - 2S(S+1)) \nonumber \\
                & =  -\frac{J}{2}\sum_{<ij>}(:\mathcal{F}_{ij}^{\dagger}\mathcal{F}_{ij}: - 2S^2)
\end{align}
\end_inset


\end_layout

\begin_layout Standard
To get some physical taste of the bond operator, we can play with the two-site Heisenberg model
\end_layout

\begin_layout Standard

\begin_inset Formula \begin{equation}
    \mathcal{H} = -J \bm{S}_i\cdot\bm{S}_j
\end{equation}
\end_inset


\end_layout

\begin_layout Standard
The classical (mean field) ground state is degenerate triplet with energy 
\begin_inset Formula $-\frac{J}{4}J$
\end_inset

. It's easy to check the following results
\end_layout

\begin_layout Standard

\begin_inset Formula \begin{align}
    :\mathcal{F}_{ij}^{\dagger}\mathcal{F}_{ij}:\ket{\uparrow,\downarrow} & = \ket{\downarrow,\uparrow} \nonumber \\
    :\mathcal{F}_{ij}^{\dagger}\mathcal{F}_{ij}:\ket{\downarrow,\uparrow} & = \ket{\uparrow,\downarrow} \nonumber \\
    :\mathcal{F}_{ij}^{\dagger}\mathcal{F}_{ij}:\ket{\uparrow,\uparrow}   & = \ket{\uparrow,\uparrow}  \nonumber \\
    :\mathcal{F}_{ij}^{\dagger}\mathcal{F}_{ij}:\ket{\downarrow,\downarrow}  & = \ket{\downarrow,\downarrow}
\end{align}
\end_inset


\end_layout

\begin_layout Standard
Then we immediately conclude that the classical ground state (triplet) is the eigenstate of 
\begin_inset Formula $-\frac{J}{2}(:\mathcal{F}_{ij}^{\dagger}\mathcal{F}_{ij}:-2S^2)$
\end_inset

, with eigenvalue
\end_layout

\begin_layout Standard

\begin_inset Formula \[-\frac{J}{2}(1-2*(1/2)^2)=-\frac{J}{4}\]
\end_inset


\end_layout

\begin_layout Standard
The bond operator 
\begin_inset Formula $\mathcal{F}_{ij}$
\end_inset

 produces the correct mean-field results for the two-site HFM, and it makes sense. The original HFM has 
\begin_inset Formula $SU(2)$
\end_inset

 symmetry, we can generalize the symmetry to 
\begin_inset Formula $SU(N)$
\end_inset

 by increasing the 
\shape italic
flavors
\shape default
 of bosons at 
\series bold
each site
\series default
 from 2 to 
\begin_inset Formula $N$
\end_inset


\end_layout

\begin_layout Standard

\begin_inset Formula \begin{equation}
    (a_i,b_i) \rightarrow (a_{i1},a_{i2},...a_{iN})
\end{equation}
\end_inset


\end_layout

\begin_layout Standard
The bond operator is generalized to
\end_layout

\begin_layout Standard

\begin_inset Formula \begin{equation}
    \mathcal{F}_{ij} \rightarrow \sum_{m=1}^N a_{im}^{\dagger}a_{jm}
\end{equation}
\end_inset


\end_layout

\begin_layout Standard
which is quadratic in boson operators of the same flavor. The constraints are generalized to
\end_layout

\begin_layout Standard

\begin_inset Formula \begin{equation}
   \sum_{m=1}^N a_{im}^{\dagger}a_{im} = NS
\end{equation}
\end_inset


\end_layout

\begin_layout Standard
The 
\begin_inset Formula $SU(N)$
\end_inset

 ferromagnetic boson (FM-B) Heisenberg model is
\end_layout

\begin_layout Standard

\begin_inset Formula \begin{align}
    \mathcal{H}^{FM-B}(N) & = -\frac{J}{N}\sum_{<ij>}(:\mathcal{F}_{ij}^{\dagger}\mathcal{F}_{ij}:-NS^2) \nonumber \\
                          & = -\frac{J}{N}\sum_{<ij>}(\sum_{mm'}S_i^{mm'}S_j^{m'm} - NS^2)
\end{align}
\end_inset


\end_layout

\begin_layout Standard
where
\end_layout

\begin_layout Standard

\begin_inset Formula \begin{equation}
    S_i^{mm'} = a_{im}^{\dagger}a_{im'}
\end{equation}
\end_inset


\end_layout

\begin_layout Standard
It's easy to check that they are the generators of the 
\begin_inset Formula $\mathfrak{su}(N)$
\end_inset

 algebra for each site (the site index is suppressed)
\end_layout

\begin_layout Standard

\begin_inset Formula \begin{align}
    \comm{S^{mm'}}{S^{\mu\mu'}} & = \comm{a_m^{\dagger}a_{m'}}{a_{\mu}^{\dagger}a_{\mu'}} \nonumber \\
                                & = \delta_{m'\mu}S^{m\mu'} - \delta_{m\mu'}S^{\mu m'}
\end{align}
\end_inset


\end_layout

\begin_layout Standard
The 
\begin_inset Formula $SU(N)$
\end_inset

 FM-B model is invariant under uniform 
\begin_inset Formula $SU(N)$
\end_inset

 transform. Because it's easy to check that, after some simple algebras
\end_layout

\begin_layout Standard

\begin_inset Formula \begin{equation}
    \comm{\mathcal{H}^{FM-B}}{S_{\nu}^{\mu\mu}} = 0
\end{equation}
\end_inset


\end_layout

\begin_layout Subsection
Schwinger bosons for antiferromagnets
\end_layout

\begin_layout Standard
Consider nearest neighbor biparticle lattice HAFM with sublattices 
\begin_inset Formula $A$
\end_inset

 and 
\begin_inset Formula $B$
\end_inset

. A bond 
\begin_inset Formula $<ij>$
\end_inset

 is defined such that 
\begin_inset Formula $i\in A$
\end_inset

 and 
\begin_inset Formula $j\in B$
\end_inset

. The antiferromagnetic bond operator is defined as
\end_layout

\begin_layout Standard

\begin_inset Formula \begin{equation}
    \bm{\mathcal{A}}_{ij} = a_ib_j - b_ia_j
\end{equation}
\end_inset


\end_layout

\begin_layout Standard
We can try to understand this definition by looking at the two-site model again
\end_layout

\begin_layout Standard

\begin_inset Formula \begin{align}
   \mathcal{\hat{O}} & = \bm{\mathcal{A}}_{ij}^{\dagger}\bm{\mathcal{A}}_{ij} \nonumber \\
                     & = b_j^{\dagger}a_i^{\dagger}a_ib_j - b_j^{\dagger}a_i^{\dagger}b_ia_j \nonumber \\
                     & - a_j^{\dagger}b_i^{\dagger}a_ib_j + a_j^{\dagger}b_i^{\dagger}b_ia_j
\end{align}
\end_inset


\end_layout

\begin_layout Standard
It's easy to see that
\end_layout

\begin_layout Standard

\begin_inset Formula \begin{align}
    \mathcal{\hat{O}}\ket{\uparrow,\downarrow} & = \ket{\uparrow, \downarrow} - \ket{\downarrow,\uparrow} \nonumber \\
    \mathcal{\hat{O}}\ket{\downarrow,\uparrow} & = -\ket{\uparrow, \downarrow} + \ket{\downarrow,\uparrow}
\end{align}
\end_inset


\end_layout

\begin_layout Standard
We conclude that the mean field ground state (singlet) of two-site HAFM is the eigenstate of 
\begin_inset Formula $\mathcal{\hat{O}}$
\end_inset

. The definition of bond operator makes sense, but it is antisymmetric with respect to the exchange of 
\begin_inset Formula $i$
\end_inset

 and 
\begin_inset Formula $j$
\end_inset

. To simplify future calculations, we define a spin rotation 
\begin_inset Formula $\pi$
\end_inset

 about the 
\begin_inset Formula $y$
\end_inset

 axis on sublattice 
\begin_inset Formula $B$
\end_inset

, which sends,
\end_layout

\begin_layout Standard

\begin_inset Formula \begin{equation}
    \begin{pmatrix}
    a_j'\\
    b_j'
    \end{pmatrix} = 
    \begin{pmatrix}
    & -1 \\
    1 & 
    \end{pmatrix}
    \begin{pmatrix}
    a_j\\
    b_j
    \end{pmatrix}
\end{equation}
\end_inset


\end_layout

\begin_layout Standard
This a canonical transform which preserves the constraint 
\begin_inset CommandInset ref
LatexCommand eqref
reference "eq:subFock"
plural "false"
caps "false"
noprefix "false"

\end_inset

, which is easy to check, since
\end_layout

\begin_layout Standard

\begin_inset Formula \begin{equation}
     \begin{pmatrix}
    & 1 \\
    -1 & 
    \end{pmatrix}
     \begin{pmatrix}
    1 &  \\
      & 1 
    \end{pmatrix}
     \begin{pmatrix}
    & -1 \\
    1 & 
    \end{pmatrix}
    =
       \begin{pmatrix}
    1 &  \\
      & 1 
    \end{pmatrix}
    \label{eq:canonical}
\end{equation}
\end_inset


\end_layout

\begin_layout Standard
The first matrix is the hermitian conjugate of the canonical transform. On the other hand 
\begin_inset Formula $S_x \rightarrow -S_x$
\end_inset

, 
\begin_inset Formula $S_y \rightarrow S_y$
\end_inset

, and 
\begin_inset Formula $S_z \rightarrow -S_z$
\end_inset

. Also,
\end_layout

\begin_layout Standard

\begin_inset Formula \begin{equation}
    \bm{\mathcal{A}}_{ij} \rightarrow \mathcal{A}_{ij} = a_ia_j + b_ib_j
    \label{eq:rotationAFM}
\end{equation}
\end_inset


\end_layout

\begin_layout Standard
The 
\begin_inset Formula $SU(2)$
\end_inset

 Heisenberg model can be written as
\end_layout

\begin_layout Standard

\begin_inset Formula \begin{equation}
    \mathcal{H} = J\sum_{<ij>}\bm{S}_i\cdot\bm{S}_j \rightarrow - \frac{J}{2}\sum_{<ij>}(\mathcal{A}_{ij}^{\dagger}\mathcal{A}_{ij}-2S^2)
\end{equation}
\end_inset


\end_layout

\begin_layout Standard
The 
\begin_inset Formula $SU(N)$
\end_inset

 model is easily generalized by
\end_layout

\begin_layout Standard

\begin_inset Formula \begin{equation}
    \mathcal{A}_{ij} = \sum_{m=1}^N a_{im}a_{jm}
\end{equation}
\end_inset


\end_layout

\begin_layout Standard

\begin_inset Formula \begin{align}
    \mathcal{H}^{AFM-B} & = -\frac{J}{N}\sum_{<ij>}(\mathcal{A}_{ij}^{\dagger}\mathcal{A}_{ij}-NS^2) \nonumber \\
                        & =  -\frac{J}{N}\sum_{<ij>}(\sum_{mm'}S_i^{mm'}\tilde{S}_j^{m'm}-NS^2)
\end{align}
\end_inset


\end_layout

\begin_layout Standard
where
\end_layout

\begin_layout Standard

\begin_inset Formula \begin{equation}
    \tilde{S}_j^{m'm} = a_{jm'}^{\dagger}a_{jm}
\end{equation}
\end_inset


\end_layout

\begin_layout Standard
are the generators of the 
\shape italic
conjugate representation
\shape default
 on sublattice 
\begin_inset Formula $B$
\end_inset

. Thus we find that the AFM-B model is not invariant under a uniform 
\begin_inset Formula $SU(N)$
\end_inset

 transform, but staggered conjugate rotation on sublattice 
\begin_inset Formula $A$
\end_inset

 and 
\begin_inset Formula $B$
\end_inset

 because of the canonical transform on the operators of sublattice 
\begin_inset Formula $B$
\end_inset

. 
\begin_inset Newline newline
\end_inset


\end_layout

\begin_layout Standard
In Auerbach's book, he also introduced the constrained fermion representation. I will omit the details.
\end_layout

\begin_layout Subsection
The generating functional
\end_layout

\begin_layout Standard
The generating Hamiltonian is defined by
\end_layout

\begin_layout Standard

\begin_inset Formula \begin{equation}
    \mathcal{H}[j] = \mathcal{H} - \sum_{imm'}j_{imm'}(\tau)a_{im}^{\dagger}a_{im'}
\end{equation}
\end_inset


\end_layout

\begin_layout Standard
where 
\begin_inset Formula $\tau \in [0,\beta)$
\end_inset

 is the imaginary time, and the external source couples with the generators of 
\begin_inset Formula $\mathfrak{su}(N)$
\end_inset

 algebra. The constraint 
\begin_inset CommandInset ref
LatexCommand eqref
reference "eq:subFock"
plural "false"
caps "false"
noprefix "false"

\end_inset

 is enforced by the projector 
\begin_inset Formula $P_S$
\end_inset

, which commutes with 
\begin_inset Formula $\mathcal{H}[j]$
\end_inset

. The imaginary time generating functional is given by
\end_layout

\begin_layout Standard

\begin_inset Formula \begin{align}
    Z[j] & = \tr P_S T_{\tau}\bigg[\exp\bigg(-\int_0^{\beta}\mathcal{H}[j]\bigg)\bigg] \nonumber \\
         & = \lim_{\epsilon\rightarrow0}\tr T_{\tau} \prod_{\tau_n = \epsilon}^{\beta}[P_S(\tau)\exp(-\epsilon\mathcal{H}[j(\tau_n)])]
\end{align}
\end_inset


\end_layout

\begin_layout Standard
The constraint can be represented by introducing another field 
\begin_inset Formula $\lambda$
\end_inset

 to couple with the bosons/fermions (if use the constrained fermion representation)
\end_layout

\begin_layout Standard

\begin_inset Formula \begin{equation}
    P_S(\tau) = \int \mathcal{D}\lambda \exp\big(-i\epsilon\sum_{im} \lambda_i(\tau)(a_{im}^{\dagger}a_{im}-S)\big)
\end{equation}
\end_inset


\end_layout

\begin_layout Standard
where the measure of the constraint field is
\end_layout

\begin_layout Standard

\begin_inset Formula \begin{equation}
   \int \mathcal{D}\lambda = \lim_{\epsilon\rightarrow0}\prod_{i\tau}\epsilon\int_{-\pi/\epsilon}^{\pi/\epsilon}d\lambda_{i\tau} 
\end{equation}
\end_inset


\end_layout

\begin_layout Standard
If the constraint 
\begin_inset CommandInset ref
LatexCommand eqref
reference "eq:subFock"
plural "false"
caps "false"
noprefix "false"

\end_inset

 is not satisfied, the phase oscillates very fast, neighboring paths cancel each other. As a result, the contributions to the path integral is zero. The limits of integration of the 
\begin_inset Formula $\lambda$
\end_inset

 fields are given by the fact that 
\begin_inset Formula $\lambda(\tau)$
\end_inset

 is periodic/anti-periodic in imaginary time. Use the boson/fermion coherent state path integral, the generating functional is given by
\end_layout

\begin_layout Standard

\begin_inset Formula \begin{align}
    Z[j] & = \int\mathcal{D}\lambda\int\mathcal{D}^2\bm{z} \exp \bigg(-\int_0^{\beta} d\tau\big(\sum_{im}z_{im}^*\partial_{\tau}z_{im} + H[j] \nonumber \\
         & + i\sum_{im}\lambda_i(\tau)(z_{im}^*z_{im}-S)\big)\bigg)
\end{align}
\end_inset


\end_layout

\begin_layout Standard
The details of the construction can be referred to my another notes on boson/fermion coherent states path integral. It's also similar to the spin coherent states path integral introduced in the previous sections of this note. The Hamiltonian function is
\end_layout

\begin_layout Standard

\begin_inset Formula \begin{equation}
    H[j] = - \frac{J}{N}\sum_{<ij>}\mathcal{Z}_{ij}^*\mathcal{Z}_{ij} - \sum_{imm'}j_{imm'}(\tau)z_{im}^*z_{im} 
\end{equation}
\end_inset


\end_layout

\begin_layout Standard
where
\end_layout

\begin_layout Standard

\begin_inset Formula \begin{equation}
    \mathcal{Z} = \begin{cases}
    \sum_m z_{im}^*z_{jm} \quad \text{FM-B}\\
    \sum_m z_{im}z_{jm} \quad \text{AFM-B}
    \end{cases}
    \label{eq:casesfmafm}
\end{equation}
\end_inset


\end_layout

\begin_layout Standard
The form of the result can be obtained by recalling that the coherent states are the eigenstates of the annihilation operators
\end_layout

\begin_layout Standard

\begin_inset Formula \begin{align}
    \ket{z} & = \exp(\sum_i z_i a_i^{\dagger})\ket{0} \nonumber \\
    a_i \ket{z} & = z_i \ket{z}
\end{align}
\end_inset


\end_layout

\begin_layout Subsection
The Hubbard-Stratonovich transform
\end_layout

\begin_layout Standard
The Hamiltonian function is bi-quadratic (four) in the parameters of the coherent states, which cannot be easily integrated in the path integral. These terms can be decoupled into bilinear terms using the 
\shape italic
Hubbard-Stratonovich identity
\shape default
, which is merely a Gaussian integral by completing the square
\end_layout

\begin_layout Standard

\begin_inset Formula \begin{equation}
    \exp\bigg(\frac{J}{N}\epsilon\mathcal{Z}^*\mathcal{Z}\bigg) = \int_{-\infty}^{\infty}d^2Q\exp\bigg(-\epsilon(\mathcal{Z}^*Q+\mathcal{Z}Q^* + N\frac{\abs{Q}^2}{J}\bigg)\bigg)
\end{equation}
\end_inset


\end_layout

\begin_layout Standard
where
\end_layout

\begin_layout Standard

\begin_inset Formula \begin{equation}
    d^2Q = \frac{\epsilon N}{\pi J}d\Re Qd\Im Q
\end{equation}
\end_inset


\end_layout

\begin_layout Standard
A complex variable 
\begin_inset Formula $Q_{ij}(\tau)$
\end_inset

 is introduced for 
\series bold
each bond
\series default
 
\begin_inset Formula $<ij>$
\end_inset

 at 
\series bold
each time step
\series default
, and the integration measure is defined as
\end_layout

\begin_layout Standard

\begin_inset Formula \begin{equation}
    \mathcal{D}^2Q = \prod_{<ij>,\tau}d^2Q_{ij}(\tau)
\end{equation}
\end_inset


\end_layout

\begin_layout Standard
Combining all the results, the generating functional is written as
\end_layout

\begin_layout Standard

\begin_inset Formula \begin{equation}
    Z[j] = \int\mathcal{D}^2Q\mathcal{D}\lambda\mathcal{D}^2\bm{z}  \exp\bigg(-\int_0^{\beta} \bigg(L[j] + N\frac{\abs{Q_{ij}}^2}{J} - iNS\sum_i\lambda_i\bigg)\bigg)
    \label{eq:generatingfunctionalwithQ}
\end{equation}
\end_inset


\end_layout

\begin_layout Standard
where 
\begin_inset Formula $L[j]$
\end_inset

 contains all the quadratic terms in 
\begin_inset Formula $z$
\end_inset

 variables, hence we call it the Lagrangian
\end_layout

\begin_layout Standard

\begin_inset Formula \begin{align}
    L[\bm{z}^*,\bm{z},j] & = \sum_{im}z_{im}^*\partial_{\tau}z_{im} + \sum_{<ij>}(\mathcal{Z}_{ij}^*Q_{ij}+\mathcal{Z}_{ij}Q_{ij}^*) \nonumber \\
    & - \sum_{imm'}(j_{imm'} + i\lambda_i\delta_{mm'})z_{im}^*z_{im'}
    \label{eq:lagrangianofHS}
\end{align}
\end_inset


\end_layout

\begin_layout Standard
For the 
\begin_inset Formula $FM-B$
\end_inset

 cases, the Lagrangians are 
\shape italic
normal
\shape default
, which contains only 
\begin_inset Formula $z^*z$
\end_inset

 terms. The Green's function matrix 
\begin_inset Formula $G$
\end_inset

 is defined by
\end_layout

\begin_layout Standard

\begin_inset Formula \begin{equation}
    L = \bm{z}^*\hat{G}^{-1}[j]\bm{z}
\end{equation}
\end_inset


\end_layout

\begin_layout Standard

\begin_inset Formula \begin{equation}
    \hat{G}^{-1}[j] = \partial_{\tau} + \hat{\lambda} + \hat{Q} - \hat{j}
    \label{eq:FMGinverse}
\end{equation}
\end_inset


\end_layout

\begin_layout Standard
the definitions are easily read from 
\begin_inset CommandInset ref
LatexCommand eqref
reference "eq:lagrangianofHS"
plural "false"
caps "false"
noprefix "false"

\end_inset

. For 
\begin_inset Formula $AFM-B$
\end_inset

, the Lagrangian contains 
\shape italic
anomalous
\shape default
 terms like 
\begin_inset Formula $z^*z^*$
\end_inset

 and 
\begin_inset Formula $zz$
\end_inset

, then we can define
\end_layout

\begin_layout Standard

\begin_inset Formula \begin{equation}
    L = \begin{pmatrix}
        \bm{z}_A^* & \bm{z}_B 
        \end{pmatrix} \hat{G}^{-1}[j]
        \begin{pmatrix}
        \bm{z}_A \\ 
        \bm{z}_B^* 
        \end{pmatrix}
\end{equation}
\end_inset


\end_layout

\begin_layout Standard
where
\end_layout

\begin_layout Standard

\begin_inset Formula \begin{equation}
    \hat{G}^{-1}[j] = \begin{pmatrix}
    \partial_{\tau} + \hat{\lambda} - \hat{j} & \hat{Q} \\
    \hat{Q}^{\dagger}  & - \partial_{\tau} + \hat{\lambda} - \hat{j}
    \end{pmatrix}
\end{equation}
\end_inset


\end_layout

\begin_layout Standard
where the minus sign for 
\begin_inset Formula $\partial_{\tau}$
\end_inset

 in the "22" elements comes from integration by parts for imaginary time 
\begin_inset Formula $\tau$
\end_inset

. We have also changed the order of the variables for the second and third term of the "22" elements. Because of the statistics of Schwinger bosons, the order change does not cause a sign change. 
\begin_inset Newline newline
\end_inset


\end_layout

\begin_layout Standard
We can integrate over the 
\begin_inset Formula $\bm{z}$
\end_inset

 variables,
\end_layout

\begin_layout Standard

\begin_inset Formula \begin{align}
    & \int\mathcal{D}^2\bm{z} \exp\bigg(-\int_0^{\beta} d\tau \bm{z}^*\hat{G}^{-1}[j]\bm{z}\bigg) \nonumber \\
    & = \prod_{\tau=0}^{\beta-\epsilon}\int\mathcal{D}^2\bm{z}(\tau) \exp\bigg(-\epsilon\sum_{i\tau,i\tau'}\bm{z}_{i\tau}^*\hat{G}_{i\tau,i\tau'}^{-1}[j]\bm{z}_{i\tau'}\bigg) \nonumber \\
    & = \big(\det(\hat{G}[j]\epsilon)\big)^{\eta} \nonumber \\
    & = \exp\big(\eta\tr\ln(\hat{G}[j]\epsilon)\big)
\end{align}
\end_inset


\end_layout

\begin_layout Standard
The generating functional 
\begin_inset CommandInset ref
LatexCommand eqref
reference "eq:generatingfunctionalwithQ"
plural "false"
caps "false"
noprefix "false"

\end_inset

 contains only path integrals of 
\begin_inset Formula $Q$
\end_inset

 and 
\begin_inset Formula $\lambda$
\end_inset

 fields
\end_layout

\begin_layout Standard

\begin_inset Formula \begin{equation}
    Z[j] =  \int\mathcal{D}^2Q\mathcal{D}\lambda \exp(-N\mathcal{S}[\lambda,Q,j])
    \label{eq:generating}
\end{equation}
\end_inset


\end_layout

\begin_layout Standard
the action of the auxiliary fields is
\end_layout

\begin_layout Standard

\begin_inset Formula \begin{equation}
    \mathcal{S} = -\frac{\eta}{N}\tr\ln(\hat{G}[j]\epsilon) + \int_0^{\beta}d\tau\bigg(\sum_{<ij>}\frac{\abs{Q_{ij}}^2}{J}-iS\sum_i\lambda_i\bigg)
    \label{eq:action}
\end{equation}
\end_inset


\end_layout

\begin_layout Standard
From this equation, we see that 
\begin_inset Formula $N$
\end_inset

 can be used as a controlling parameter for asymptotic expansion.
\end_layout

\begin_layout Section
Schwinger bosons mean field theory
\end_layout

\begin_layout Standard
Assuming the group index 
\begin_inset Formula $N$
\end_inset

 is very large, we can apply MSD on 
\begin_inset CommandInset ref
LatexCommand eqref
reference "eq:generating"
plural "false"
caps "false"
noprefix "false"

\end_inset

. The SBMFT is given by replacing the auxiliary field in the action 
\begin_inset Formula $\mathcal{S}$
\end_inset

 of 
\begin_inset Formula $\eqref{eq:generating}$
\end_inset

 by 
\shape italic
static and uniform
\shape default
 saddle point parameters:
\end_layout

\begin_layout Standard

\begin_inset Formula \begin{equation}
    N\mathcal{S}[Q_{ij}(\tau),\lambda(\tau)] \rightarrow N\mathcal{S}_0(Q,-i\lambda)
\end{equation}
\end_inset


\end_layout

\begin_layout Standard
The mean field generating (partition) function (without sources) can be written as
\end_layout

\begin_layout Standard

\begin_inset Formula \begin{equation}
    Z^{MF} = e^{-N\mathcal{S}_0}
\end{equation}
\end_inset


\end_layout

\begin_layout Standard
The MF free energy is related to the MF partition function through the identity in thermodynamics
\end_layout

\begin_layout Standard

\begin_inset Formula \begin{equation}
    F^{MF} = - \beta^{-1} \ln Z^{MF} = - \beta^{-1} \ln \tr_{im} e^{-\beta H^{MF}(Q,\lambda)}
    \label{eq:MFfree}
\end{equation}
\end_inset


\end_layout

\begin_layout Standard
The MF free energy can be further written as
\end_layout

\begin_layout Standard

\begin_inset Formula \begin{equation}
    \beta F^{MF} = N\mathcal{S}_0
    \label{eq:MFandAction}
\end{equation}
\end_inset


\end_layout

\begin_layout Standard

\begin_inset CommandInset ref
LatexCommand eqref
reference "eq:MFfree"
plural "false"
caps "false"
noprefix "false"

\end_inset

 also defines the MF Hamiltonian, which allows us to work with the creation/annihilation operators of Schwinger bosons.
\end_layout

\begin_layout Subsection
Ferromagnetic systems
\end_layout

\begin_layout Standard
Refer to the exact form of the action given in 
\begin_inset CommandInset ref
LatexCommand eqref
reference "eq:generatingfunctionalwithQ"
plural "false"
caps "false"
noprefix "false"

\end_inset

, the (imaginary) time- and position-dependence are eliminated by the MF values. The integral of time is just 
\begin_inset Formula $\beta$
\end_inset

. The MF Hamiltonian (classical value) is obtained through Legendre transform on the Lagrangian
\end_layout

\begin_layout Standard

\begin_inset Formula \begin{equation}
    H = \big(\sum_{im} z_{im}^*\partial_{\tau}z_{im} - L\big) + N\mathcal{N}\frac{zQ^2}{2J} - N\mathcal{N}S\lambda
    \label{eq:MFH1}
\end{equation}
\end_inset


\end_layout

\begin_layout Standard
Notice that the measure for the 
\begin_inset Formula $Q$
\end_inset

 field is
\end_layout

\begin_layout Standard

\begin_inset Formula \begin{equation}
    \mathcal{D}^2Q = \prod_{<ij>\tau}d^2Q_{ij}(\tau)
\end{equation}
\end_inset


\end_layout

\begin_layout Standard
Since the time-dependence is eliminated, the product of all bonds simply reduced to a sum in the exponential. The number of bonds is given by the number of site 
\begin_inset Formula $\mathcal{N}$
\end_inset

 times the number of nearest neighbors 
\begin_inset Formula $z$
\end_inset

 divided by two because of over counting. The last term of 
\begin_inset CommandInset ref
LatexCommand eqref
reference "eq:MFH1"
plural "false"
caps "false"
noprefix "false"

\end_inset

 is obtained by
\end_layout

\begin_layout Standard

\begin_inset Formula \begin{equation}
    -iNS\sum_i\lambda_i \rightarrow -iNS \mathcal{N} * -i\lambda = -N\mathcal{N}S\lambda
\end{equation}
\end_inset


\end_layout

\begin_layout Standard
The MF Hamiltonian (operator) can be recovered by using boson coherent states
\end_layout

\begin_layout Standard

\begin_inset Formula \begin{align}
    H^{MF} & = \sum_{im} \lambda a_{im}^{\dagger}a_{im} - Q \sum_{<ij>,m}(a_{im}^{\dagger}a_{jm} + a_{jm}^{\dagger}a_{im}) \nonumber \\
           & + N\mathcal{N}\frac{zQ^2}{2J} - N\mathcal{N}S\lambda \nonumber \\
           & = \sum_{\bm{k},m}\epsilon_{\bm{k}}a_{\bm{k}m}^{\dagger}a_{\bm{k}m} + N\mathcal{N}\frac{zQ^2}{2J} - N\mathcal{N}S\lambda
\end{align}
\end_inset


\end_layout

\begin_layout Standard
The last step is given by Fourier transform on the position space operators
\end_layout

\begin_layout Standard

\begin_inset Formula \begin{equation}
    a_{\bm{k}m} = \frac{1}{\sqrt{\mathcal{N}}}\sum_{i}e^{i\bm{k}\cdot\bm{x}_i}a_{im}
\end{equation}
\end_inset


\end_layout

\begin_layout Standard
The dispersion relation can be easily derived without out further diagonalization since the Hamiltonian does not have anomalous terms
\end_layout

\begin_layout Standard

\begin_inset Formula \begin{align}
    \epsilon_{\bm{k}} & = \lambda - zQ\gamma_{\bm{k}} \nonumber \\
    \gamma_{\bm{k}}   & = z^{-1}\sum_{\bm{k}}e^{i\bm{k}\cdot\bm{\delta}}
\end{align}
\end_inset


\end_layout

\begin_layout Standard
where 
\begin_inset Formula $\bm{\delta}$
\end_inset

 are the nearest neighbor bond vectors. 
\begin_inset Newline newline
\end_inset


\end_layout

\begin_layout Standard
We should keep in mind that the MF values of 
\begin_inset Formula $Q$
\end_inset

 and 
\begin_inset Formula $\lambda$
\end_inset

 are not determined yet. They should be determined by varying 
\begin_inset Formula $\mathcal{S}_0$
\end_inset

 (i.e. 
\begin_inset Formula $F^{MF}$
\end_inset

) to find the saddle points. The saddle points correspond to the 
\series bold
free theory
\series default
 of Schwinger bosons, since the MF Hamiltonian is quadratic in the boson operators. The many-body non-interacting partition function is given by
\end_layout

\begin_layout Standard

\begin_inset Formula \begin{align}
    Z & = \sum_{n_{\bm{k}} = 0}^{\infty} e^{-\beta\sum_{\bm{k},m}n_{\bm{k}}h^{MF}_{\bm{k}} } \quad h^{MF}_{\bm{k}} = \epsilon_{\bm{k}} + N\mathcal{N}\frac{zQ^2}{2J} - N\mathcal{N}S\lambda \nonumber \\
      & = \sum_{n_{\bm{k}} = 0}^{\infty}\prod_{\bm{k},m}e^{-\beta n_{\bm{k}}h^{MF}_{\bm{k}}} \nonumber \\
      & = e^{-\beta(N\mathcal{N}\frac{zQ^2}{2J} - N\mathcal{N}S\lambda)}\sum_{n_{\bm{k}} = 0}^{\infty}\prod_{\bm{k},m}e^{-\beta n_{\bm{k}}\epsilon_{\bm{k}}} \nonumber \\
      & = e^{-\beta(N\mathcal{N}\frac{zQ^2}{2J} - N\mathcal{N}S\lambda)}\prod_{\bm{k},m}(1-e^{-\beta\epsilon_{\bm{k}}})^{-1}
\end{align}
\end_inset


\end_layout

\begin_layout Standard
The MF free energy (action) is given by
\end_layout

\begin_layout Standard

\begin_inset Formula \begin{align}
    F^{MF} & = - \beta^{-1}\ln Z \nonumber \\
           & = - \beta^{-1}\ln \bigg(e^{-\beta(N\mathcal{N}\frac{zQ^2}{2J} - N\mathcal{N}S\lambda)}\prod_{\bm{k},m}(1-e^{-\beta\epsilon_{\bm{k}}})^{-1}\bigg) \nonumber \\
           & = - \beta^{-1}\bigg(-\beta(N\mathcal{N}\frac{zQ^2}{2J} - N\mathcal{N}S\lambda) - \sum_{\bm{k},m} \ln(1-e^{-\beta\epsilon_{\bm{k}}})\bigg) \nonumber \\
           & = N \beta^{-1}\sum_{\bm{k}}\ln(1-e^{-\beta\epsilon_{\bm{k}}}) + N\mathcal{N}\frac{zQ^2}{2J} - N\mathcal{N}S\lambda
\end{align}
\end_inset


\end_layout

\begin_layout Standard
The static fields are determined through
\end_layout

\begin_layout Standard

\begin_inset Formula \begin{align}
    \frac{\partial F^{MF}}{\partial \lambda} & = 0 \nonumber \\
                                             & =  N \beta^{-1}\sum_{\bm{k}}\frac{\beta e^{-\beta\epsilon_{\bm{k}}}}{1-e^{-\beta\epsilon_{\bm{k}}}} - N\mathcal{N}S
\end{align}
\end_inset


\end_layout

\begin_layout Standard
and
\end_layout

\begin_layout Standard

\begin_inset Formula \begin{align}
    \frac{\partial F^{MF}}{\partial Q} & = 0 \nonumber \\ 
                                       & = N \beta^{-1}\sum_{\bm{k}} \frac{\beta e^{-\beta\epsilon_{\bm{k}}}(-z\gamma_{\bm{k}})}{1-e^{-\beta\epsilon_{\bm{k}}}} + N\mathcal{N}\frac{Q}{J}   
                                       \label{eq:lambdaQ}
\end{align}
\end_inset


\end_layout

\begin_layout Standard
We have
\end_layout

\begin_layout Standard

\begin_inset Formula \begin{align}
    \frac{1}{\mathcal{N}}\sum_{\bm{k}}n_{\bm{k}} & = S \nonumber \\
    \frac{1}{\mathcal{N}}\sum_{\bm{k}}n_{\bm{k}}\gamma_{\bm{k}} & = Q/J\nonumber \\
    \text{where} \quad n_{\bm{k}} & = (e^{\beta\epsilon_{\bm{k}}}-1)^{-1}
    \label{eq:distributionbose}
\end{align}
\end_inset


\end_layout

\begin_layout Standard
which uniquely determines the MF static fields 
\begin_inset Formula $Q(T,S)$
\end_inset

 and 
\begin_inset Formula $\lambda(T,S)$
\end_inset

.
\begin_inset Newline newline
\end_inset


\end_layout

\begin_layout Standard
It's convenient to use the more physical parametrization of the mean field variables
\end_layout

\begin_layout Standard

\begin_inset Formula \begin{align}
    \epsilon_{\bm{k}} & = zQ(1-\gamma_{\bm{k}} + \frac{1}{4z}\kappa^2) \nonumber \\
    \kappa^2          & = 4z(\lambda/zQ-1)
\end{align}
\end_inset


\end_layout

\begin_layout Standard
It's obvious that 
\begin_inset Formula $zQ$
\end_inset

 describes the bandwidth. The quantity 
\begin_inset Formula $\kappa$
\end_inset

 is related to the inverse of the correlation length, let's make it clear. First of all, notice that 
\begin_inset Formula $\kappa$
\end_inset

 has temperature dependence through 
\begin_inset Formula $Q$
\end_inset

. At finite temperature, 
\begin_inset Formula $\frac{1}{4z}\kappa^2$
\end_inset

 is the 
\series bold
gap
\series default
 of the spectrum of MFT Schwinger bosons. I will claim that the gap is closely related to the inverse of the correlation length. Consider the 
\series bold
ground state
\series default
 spin correlation at the same position but different imaginary time
\end_layout

\begin_layout Standard

\begin_inset Formula \begin{align}
    \expval{S(0,0)S(\tau,0)} & = \bra{\Psi_0}S(0,0)S(\tau,0)\ket{\Psi_0} \nonumber \\
                             & = \bra{\Psi_0}S(0,0)(\sum_m \ket{\Psi_m}\bra{\Psi_m})S(\tau,0)\ket{\Psi_0} \nonumber \\
                             & = \sum_m \bra{\Psi_0}S(0,0)\ket{\Psi_m}\bra{\Psi_m}e^{-\tau H} S(0,0) e^{\tau H}\ket{\Psi_0} \nonumber \\
                             & = \sum_m e^{-\tau(E_m-E_0)}\abs{\bra{\Psi_0}S(0,0)\ket{\Psi_m}}^2
\end{align}
\end_inset


\end_layout

\begin_layout Standard
The decay periods in imaginary time are 
\begin_inset Formula $E_m-E_0$
\end_inset

. The fastest decay period is given by the lowest excitations, i.e. the 
\series bold
gap
\series default
 
\begin_inset Formula $\Delta$
\end_inset

. On the other hand, if the spin correlations in space decay exponentially in distance as 
\begin_inset Formula $e^{-x_{ij}/\xi}$
\end_inset

. The gap is related to the correlation length by some constant 
\begin_inset Formula $c$
\end_inset


\end_layout

\begin_layout Standard

\begin_inset Formula \begin{equation}
  \Delta = c/\xi
\end{equation}
\end_inset


\end_layout

\begin_layout Standard
where I have treated the space and imaginary time equally. The rigorous argument will be derived later. The explicit expression for 
\begin_inset Formula $Q$
\end_inset

 can be obtained from 
\begin_inset CommandInset ref
LatexCommand eqref
reference "eq:lambdaQ"
plural "false"
caps "false"
noprefix "false"

\end_inset


\end_layout

\begin_layout Standard

\begin_inset Formula \begin{equation}
    Q = JS - \frac{J}{\mathcal{N}}\sum_{\bm{k}}n_{\bm{k}}(1-\gamma_{\bm{k}})
\end{equation}
\end_inset


\end_layout

\begin_layout Standard
At low temperature, 
\begin_inset Formula $n_{\bm{k}} \simeq (\beta\epsilon_{\bm{k}})^{-1}$
\end_inset

, but the integrand on the right hand side is bounded at 
\begin_inset Formula $\bm{k} \rightarrow 0$
\end_inset


\end_layout

\begin_layout Standard

\begin_inset Formula \begin{equation}
    Q = JS + \mathcal{O}(T/JS)
\end{equation}
\end_inset


\end_layout

\begin_layout Standard
As we take 
\begin_inset Formula $T \rightarrow 0$
\end_inset

, and 
\begin_inset Formula $\mathcal{N} \rightarrow \infty$
\end_inset

 (
\shape italic
the thermodynamic limit
\shape default
), the boson occupation distribution 
\begin_inset CommandInset ref
LatexCommand eqref
reference "eq:distributionbose"
plural "false"
caps "false"
noprefix "false"

\end_inset

 vanishes for all 
\begin_inset Formula $\bm{k} \neq 0$
\end_inset

. However the total number of Schwinger bosons should be conserved,
\end_layout

\begin_layout Standard

\begin_inset Formula \begin{equation}
    S = \frac{1}{\mathcal{N}}\sum_{\bm{k}}n_{\bm{k}} \simeq \frac{1}{\mathcal{N}}n_0 \simeq \frac{1}{\mathcal{N}}(\beta\epsilon_0)^{-1} = \bigg(\beta\frac{Q\kappa^2}{4\mathcal{N}}\bigg)^{-1} \simeq  \bigg(\beta\frac{JS\kappa^2}{4\mathcal{N}}\bigg)^{-1}
\end{equation}
\end_inset


\end_layout

\begin_layout Standard
The inverse of correlation length reads
\end_layout

\begin_layout Standard

\begin_inset Formula \begin{equation}
    \lim_{T\rightarrow0}\kappa = \sqrt{\frac{4T}{\mathcal{N}JS^2}}
\end{equation}
\end_inset


\end_layout

\begin_layout Standard
The MFT of Schwinger boson predicts a long range order for 
\begin_inset Formula $SU(N)$
\end_inset

 Heisenberg model, since the inverse of the correlation length goes to zero as 
\begin_inset Formula $T$
\end_inset

 goes to zero. 
\begin_inset Newline newline
\end_inset


\end_layout

\begin_layout Standard
The MFT theory is derived by assuming a large group index 
\begin_inset Formula $N$
\end_inset

. In order to study the spin correlations, we have to restrict ourselves to 
\begin_inset Formula $N=2$
\end_inset

. The MFT theory already predicts a broken symmetry at 
\begin_inset Formula $T\rightarrow0$
\end_inset

 (
\begin_inset Formula $\kappa \rightarrow 0$
\end_inset

). We can break the symmetry by introducing a small magnetic field in the 
\begin_inset Formula $z$
\end_inset

 direction
\end_layout

\begin_layout Standard

\begin_inset Formula \begin{equation}
    H^{MF} \rightarrow H^{MF} - h\sum_{i,s=\pm\frac{1}{2}}s a_{is}^{\dagger}a_{is}
\end{equation}
\end_inset


\end_layout

\begin_layout Standard
The field splits the degeneracy between the two Schwinger boson dispersions (it is an explicitly symmetry breaking, not SSB).
\end_layout

\begin_layout Standard

\begin_inset Formula \begin{equation}
    \epsilon_{\bm{k},s} = \epsilon_{\bm{k}} - hs
    \label{eq:disperwiths}
\end{equation}
\end_inset


\end_layout

\begin_layout Standard
As 
\begin_inset Formula $T \rightarrow 0$
\end_inset

, the system will condensate to 
\begin_inset Formula $s = 1/2$
\end_inset

, 
\begin_inset Formula $\bm{k} = 0$
\end_inset


\end_layout

\begin_layout Standard

\begin_inset Formula \begin{equation}
    S = \frac{1}{2}\lim_{T\rightarrow0}\mathcal{N}^{-1} \sum_{\bm{k}} \bigg(n(\epsilon_{\bm{k},\frac{1}{2}}) + n(\epsilon_{\bm{k},-\frac{1}{2}})\bigg) \rightarrow \frac{1}{2}\mathcal{N}^{-1}n(\epsilon_{0,\frac{1}{2}})
\end{equation}
\end_inset


\end_layout

\begin_layout Standard
where the 
\begin_inset Formula $\frac{1}{2}$
\end_inset

 is from the group index 
\begin_inset Formula $N=2$
\end_inset

. The magnetization is given by
\end_layout

\begin_layout Standard

\begin_inset Formula \begin{equation}
    m_0 = \frac{1}{2}\lim_{h\rightarrow0^+}\mathcal{N}^{-1} \sum_{\bm{k}} \bigg(n(\epsilon_{\bm{k},\frac{1}{2}}) - n(\epsilon_{\bm{k},-\frac{1}{2}})\bigg) \rightarrow \frac{1}{2}\mathcal{N}^{-1}n(\epsilon_{0,\frac{1}{2}}) = S
\end{equation}
\end_inset


\end_layout

\begin_layout Standard
It's also interesting to notice that, the spectrum becomes gapless at 
\begin_inset Formula $T \rightarrow 0$
\end_inset


\end_layout

\begin_layout Standard

\begin_inset Formula \begin{equation}
    \lim_{T \rightarrow 0}\epsilon_{\bm{k}} = zJS(1-\gamma_{\bm{k}}) \simeq JS\abs{\bm{k}}^2
\end{equation}
\end_inset


\end_layout

\begin_layout Standard
which matches the FM spin wave results. The magnetic susceptibility which describes the long range spin correlation (the magnetic field acts as a uniform "current" for all sites, thus the second derivative of the current with respect to the free energy (generating function) should give us the average long range spin correlation)
\end_layout

\begin_layout Standard

\begin_inset Formula \begin{equation}
    \chi^{MF} = -\mathcal{N}^{-1}\dv{^2F(h)}{^2h} = - \frac{1}{4\mathcal{N}}\dv{^2F(h)}{^2\lambda} = \frac{1}{4\mathcal{N}T}\sum_{\bm{k}}n_{\bm{k}}(n_{\bm{k}}+1)
\end{equation}
\end_inset


\end_layout

\begin_layout Standard
Notice that the derivative with respect to 
\begin_inset Formula $h$
\end_inset

 is equivalent to on quarter of the derivative with respect to 
\begin_inset Formula $\lambda$
\end_inset

 as evident in 
\begin_inset CommandInset ref
LatexCommand eqref
reference "eq:disperwiths"
plural "false"
caps "false"
noprefix "false"

\end_inset

. The mean field spin correlations are given by the 
\series bold
free boson
\series default
 expression
\end_layout

\begin_layout Standard

\begin_inset Formula \begin{align}
    \expval{S_i^+S_j^-}_{MF} & = \expval{a_{i,\frac{1}{2}}^{\dagger}a_{i,-\frac{1}{2}}a_{j,-\frac{1}{2}}^{\dagger}a_{j,\frac{1}{2}}} \nonumber \\
                             & = \expval{\frac{1}{\mathcal{N}}\sum_{\bm{k}} a_{\bm{k},\frac{1}{2}}^{\dagger}a_{\bm{k},\frac{1}{2}}e^{i\bm{k}\cdot\bm{x}_{ij}}\frac{1}{\mathcal{N}}\sum_{\bm{k}'} a_{\bm{k}',-\frac{1}{2}}^{\dagger}a_{\bm{k}',-\frac{1}{2}}e^{i\bm{k}'\cdot\bm{x}_{ij}}} \nonumber \\
                             & = \abs{R_{ij}}^2 + S\delta_{ij}
                             \label{eq:MFcorrelationfunction}
\end{align}
\end_inset


\end_layout

\begin_layout Standard
where
\end_layout

\begin_layout Standard

\begin_inset Formula \begin{equation}
    R_{ij} = \frac{1}{\mathcal{N}}\sum_{\bm{k}}n_{\bm{k}}e^{i\bm{k}\cdot\bm{x}_{ij}}
    \label{eq:Rij}
\end{equation}
\end_inset


\end_layout

\begin_layout Standard

\series bold
Warning
\series default
: I omitted some importance steps in the calculation. For detail discussions, see the derivation for the spin correlation function of antiferromagnetic systems, where I clarified everything. 
\begin_inset Newline newline
\end_inset


\end_layout

\begin_layout Standard
The additional 
\begin_inset Formula $\delta_{ij}$
\end_inset

 comes from the commutation relations of boson operators on the same site and we also use the constraint 
\begin_inset CommandInset ref
LatexCommand eqref
reference "eq:distributionbose"
plural "false"
caps "false"
noprefix "false"

\end_inset

. The local spin fluctuation is given by
\end_layout

\begin_layout Standard

\begin_inset Formula \begin{equation}
   \expval{S_i^+S_i^-}_{MF} = S(S+1)
\end{equation}
\end_inset


\end_layout

\begin_layout Standard
The exact result for 
\begin_inset Formula $SU(2)$
\end_inset

 theory is of course 
\begin_inset Formula $\frac{2}{3}S(S+1)$
\end_inset

, the MFT fails to predict the right factor. Next, we will concentrate on the physics at low temperature 
\begin_inset Formula $T \ll JS$
\end_inset

. We will discuss the physics of one dimension and two dimension separately.
\end_layout

\begin_layout Subsubsection
One dimension
\end_layout

\begin_layout Standard
In one dimension at low but finite temperature 
\begin_inset Formula $T \ll JS$
\end_inset

,
\end_layout

\begin_layout Standard

\begin_inset Formula \begin{align}
    & \epsilon \simeq JS(\frac{1}{4}\kappa^2 + k^2) \nonumber \\
    & n(\epsilon)  \simeq T/\epsilon
\end{align}
\end_inset


\end_layout

\begin_layout Standard
The conservation of number of Schwinger bosons transforms into an integral, where the integration limit can be extended to infinity since the integrand decays very fast in 
\begin_inset Formula $k$
\end_inset

.
\end_layout

\begin_layout Standard

\begin_inset Formula \begin{align}
    \frac{1}{\mathcal{N}}\sum_{\bm{k}}n_{\bm{k}} & \simeq 2\frac{T}{JS}\int_0^{\infty}\frac{dk}{2\pi}\frac{1}{\frac{1}{4}\kappa^2+k^2} \quad ,\text{even integrand} \nonumber \\
                                                 & = 2\frac{T}{JS\pi}\frac{\pi}{2\sqrt{\frac{1}{4}\kappa^2}} \nonumber \\
                                                 & = \frac{T}{JS\kappa} \simeq S
\end{align}
\end_inset


\end_layout

\begin_layout Standard
Thus, we obtain
\end_layout

\begin_layout Standard

\begin_inset Formula \begin{equation}
    \kappa \simeq \frac{T}{JS^2}
\end{equation}
\end_inset


\end_layout

\begin_layout Standard
We are ready to discuss the spin correlations at low temperatures in one dimension. The MF susceptibility can also be evaluated as an integral
\end_layout

\begin_layout Standard

\begin_inset Formula \begin{align}
    \chi^{MF} & \simeq \frac{1}{4T\pi}\frac{T^2}{(JS)^2}\int_0^{\infty}dk\frac{1}{(\frac{1}{4}\kappa^2+k^2)^2} \quad \text{integrand}: n^2(\epsilon) \nonumber \\
              & = \frac{1}{4T\pi}\frac{T^2}{(JS)^2} \frac{\pi}{4(\frac{1}{4}\kappa^2)^{3/2}} \nonumber \\
              & = \frac{1}{2T}\frac{T^2}{(JS)^2}\frac{1}{\kappa^3} \nonumber \\
              & \simeq \frac{JS^4}{2T^2}
\end{align}
\end_inset


\end_layout

\begin_layout Standard
The MFT predicts that 
\begin_inset Formula $\chi^{MF} \propto T^{-2}$
\end_inset

, which matches the exact result calculated by Bethe's solution. The spin correlation between two different sites is given by 
\begin_inset Formula $R_{ij}$
\end_inset

 defined in 
\begin_inset CommandInset ref
LatexCommand eqref
reference "eq:Rij"
plural "false"
caps "false"
noprefix "false"

\end_inset

.
\end_layout

\begin_layout Standard

\begin_inset Formula \begin{align}
    R_{ij} & = \frac{1}{\mathcal{N}}\sum_{\bm{k}} n_{\bm{k}}e^{i\bm{k}\cdot\bm{x}_{ij}} \nonumber \\
           & = \frac{T}{JS\pi}\int_0^{\infty}\frac{e^{ikx_{ij}}}{k^2+\frac{1}{4}\kappa^2} \nonumber \\
           & = 2\pi i\frac{T}{JS\pi}\Res\bigg(f, \frac{i}{2}\kappa\bigg) \quad f = \frac{e^{ikx_{ij}}}{k^2+\frac{1}{4}\kappa^2} \nonumber \\
           & = 2\frac{T}{JS\kappa}e^{-\frac{1}{2}\kappa x_{ij}} \nonumber \\
           & \simeq 2Se^{-\frac{1}{2}\kappa x_{ij}}
\end{align}
\end_inset


\end_layout

\begin_layout Standard
Thus for 
\begin_inset Formula $i \neq j$
\end_inset


\end_layout

\begin_layout Standard

\begin_inset Formula \begin{equation}
    \expval{S_i^+S_j^-}_{MF} = \abs{R_{ij}}^2 = 4S^2 e^{-\kappa x_{ij}} = 4S^2 e^{-x_{ij}/\xi}
\end{equation}
\end_inset


\end_layout

\begin_layout Standard
where 
\begin_inset Formula $\xi$
\end_inset

 is the correlation length. This establishes that the parameter 
\begin_inset Formula $\kappa$
\end_inset

 is the inverse of the correlation length.
\end_layout

\begin_layout Subsubsection
Two dimensions
\end_layout

\begin_layout Standard
The dispersion relation of 2d square lattice (
\begin_inset Formula $z=4$
\end_inset

) is given by
\end_layout

\begin_layout Standard

\begin_inset Formula \begin{equation}
    \epsilon_{\bm{k}} \simeq 4JS(1-\gamma_{\bm{k}} + \frac{\kappa^2}{16})
\end{equation}
\end_inset


\end_layout

\begin_layout Standard

\begin_inset Formula \begin{equation}
    n_{\bm{k}} \simeq T/\epsilon_{\bm{k}}
\end{equation}
\end_inset


\end_layout

\begin_layout Standard
The low-energy temperature dependence of the correlation length can again be obtained by using the first saddle point equation 
\begin_inset CommandInset ref
LatexCommand eqref
reference "eq:distributionbose"
plural "false"
caps "false"
noprefix "false"

\end_inset


\end_layout

\begin_layout Standard

\begin_inset Formula \begin{align}
  \mathcal{N}^{-1}\sum_{\bm{k}}n_{\bm{k}} & \simeq \frac{T}{4JS}\int_{-\pi}^{\pi}\frac{dk_x}{2\pi}\frac{dk_y}{2\pi}\frac{1}{1-\gamma_{\bm{k}} + \frac{\kappa^2}{16}} \nonumber \\
                                & = \frac{T}{JS} \int d\gamma \rho(\gamma) \frac{1}{4(1-\gamma) + \frac{\kappa^2}{4}}
\end{align}
\end_inset


\end_layout

\begin_layout Standard
The density of states for 2d case is discussed in detail in the 
\series bold
Appendix B
\series default
, which reads
\end_layout

\begin_layout Standard

\begin_inset Formula \begin{equation}
    \rho(\gamma) = \frac{2}{\pi^2}K(1-\gamma^2)
\end{equation}
\end_inset


\end_layout

\begin_layout Standard
The properties of the function can be easily found from any book. Since we are interested in the low-energy excitations, where 
\begin_inset Formula $\gamma\simeq1$
\end_inset

, 
\begin_inset Formula $\rho(\gamma)\simeq\frac{1}{\pi}$
\end_inset

. Then we simplify the density of states by
\end_layout

\begin_layout Standard

\begin_inset Formula \begin{equation}
    \rho(\gamma) \rightarrow \bar{\rho} = \frac{1}{\pi}
\end{equation}
\end_inset


\end_layout

\begin_layout Standard
To keep a unit total density, we should extend the domain of 
\begin_inset Formula $\gamma$
\end_inset

 from 
\begin_inset Formula $1-\pi$
\end_inset

 to 
\begin_inset Formula $1$
\end_inset

. Thus
\end_layout

\begin_layout Standard

\begin_inset Formula \begin{align}
  \mathcal{N}^{-1}\sum_{\bm{k}} & = \frac{T}{JS} \int d\gamma \rho(\gamma) \frac{1}{4(1-\gamma) + \frac{\kappa^2}{4}} \nonumber \\
                                & \simeq \frac{T}{JS}\int_{1-\pi}^1 d\gamma \bar{\rho}\frac{1}{4(1-\gamma) + \frac{\kappa^2}{4}} \nonumber \\
                                & =\frac{T}{4JS\pi}\ln\bigg(\frac{16\pi}{\kappa^2}+1\bigg) = S
\end{align}
\end_inset


\end_layout

\begin_layout Standard
The inverse of the correlation length is given by
\end_layout

\begin_layout Standard

\begin_inset Formula \begin{equation}
    \kappa = \sqrt{\frac{16\pi}{e^{4JS^2\pi/T}-1}} \simeq 4\sqrt{\pi}e^{-2JS^2\pi/T}
\end{equation}
\end_inset


\end_layout

\begin_layout Standard
the "1" in the denominator is small compared with 
\begin_inset Formula $e^{4JS^2\pi/T}$
\end_inset

 for low temperature, thus ignored. The relation of the parameter 
\begin_inset Formula $\kappa$
\end_inset

 and the inverse of the correlation length has not been established yet. As discussed in 
\begin_inset CommandInset ref
LatexCommand eqref
reference "eq:MFcorrelationfunction"
plural "false"
caps "false"
noprefix "false"

\end_inset

, the off-site spin correlation function is proportional to 
\begin_inset Formula $\abs{R_{ij}}^2$
\end_inset

 given in 
\begin_inset CommandInset ref
LatexCommand eqref
reference "eq:Rij"
plural "false"
caps "false"
noprefix "false"

\end_inset

. Consider spin correlations at large distances, i.e. long wavelengths (small momenta) in momentum space.
\end_layout

\begin_layout Standard

\begin_inset Formula \begin{align}
    R_{ij} & = \int_{-\pi}^{\pi}\frac{dk_xdk_y}{(2\pi)^2}n_{\bm{k}}e^{i\bm{k}\cdot\bm{x}_{ij}}\nonumber \\
           & \simeq \int_{-\pi}^{\pi}\frac{dk_xdk_y}{(2\pi)^2} \frac{T}{\epsilon_{\bm{k}}}e^{i\bm{k}\cdot\bm{x}_{ij}}\nonumber \\
           & \simeq \frac{T}{JS}\int_{-\pi}^{\pi}\frac{dk_xdk_y}{(2\pi)^2}\frac{e^{i\bm{k}\cdot\bm{x}_{ij}}}{k^2+\kappa^2/4}
\end{align}
\end_inset


\end_layout

\begin_layout Standard
where I have used the fact that 
\begin_inset Formula $\gamma_{\bm{k}}\simeq 1-1/4k^2$
\end_inset

 for small 
\begin_inset Formula $\bm{k}$
\end_inset

. In 3d, if the integration limit is extended to infinity, the Fourier transform is a Yukawa potential. In 2d, the calculation is similar.
\end_layout

\begin_layout Standard

\begin_inset Formula \begin{align}
   & \int_{-\pi}^{\pi}\frac{dk_xdk_y}{(2\pi)^2}\frac{e^{i\bm{k}\cdot\bm{x}_{ij}}}{k^2+\kappa^2/4} \nonumber \\
   & \simeq \int_{\text{all space}}\frac{dk_xdk_y}{(2\pi)^2}\frac{e^{i\bm{k}\cdot\bm{x}_{ij}}}{k^2+\kappa^2/4} \quad \text{integrand decays fast}\nonumber \\
   & = \frac{1}{2\pi}\int_{-1}^{+1}dt\int_0^{\infty}dk\frac{k^2e^{ik\abs{x_{ij}}t}}{k^2+\kappa^2/4} \nonumber \\
   & = \frac{1}{2\pi}\int_0^{\infty}dk\frac{1}{ik\abs{x_{ij}}}(e^{ik\abs{x_{ij}}} - e^{-ik\abs{x_{ij}}})\frac{k^2}{k^2+\kappa^2/4} \nonumber \\
   & = \frac{1}{2\pi}\int_0^{\infty}dk\frac{1}{i\abs{x_{ij}}}(e^{ik\abs{x_{ij}}} - e^{-ik\abs{x_{ij}}})\frac{k}{k^2+\kappa^2/4} \nonumber \\
   & = \big(\Res(f_1;i\kappa/2) + \Res(f_2;-i\kappa/2)\big) 
   \label{eq:rediduetheorem}
\end{align}
\end_inset


\end_layout

\begin_layout Standard
where
\end_layout

\begin_layout Standard

\begin_inset Formula \begin{align}
    f_1 & = \frac{1}{\abs{x_{ij}}}e^{ik\abs{x_{ij}}}\frac{k}{k^2+\kappa^2/4} \nonumber \\
    f_2 & = \frac{1}{\abs{x_{ij}}}e^{-ik\abs{x_{ij}}}\frac{k}{k^2+\kappa^2/4}
\end{align}
\end_inset


\end_layout

\begin_layout Standard
the two residues should be added together because the contour is clockwise in the lower half plane. Then
\end_layout

\begin_layout Standard

\begin_inset Formula \begin{equation}
    R_{ij} \simeq \frac{T}{JS}\frac{1}{\abs{x_{ij}}}e^{-\abs{x_{ij}}\kappa/2} 
\end{equation}
\end_inset


\end_layout

\begin_layout Standard
The relation is formally established 
\begin_inset Formula $\chi = \kappa^{-1}$
\end_inset

. In accordance with the 
\series bold
Mermin and Wagner theorem
\series default
, the MFT predicts there is no long-range order for any finite 
\begin_inset Formula $T>0$
\end_inset

.
\end_layout

\begin_layout Subsection
Antiferromagnetic systems
\end_layout

\begin_layout Standard
Consider nearest neighbor HAFM on square lattice. Refer to the action 
\begin_inset CommandInset ref
LatexCommand eqref
reference "eq:generatingfunctionalwithQ"
plural "false"
caps "false"
noprefix "false"

\end_inset

, the MF Hamiltonian can be written as
\end_layout

\begin_layout Standard

\begin_inset Formula \begin{align}
    H^{MF} & = \sum_{im}\lambda a_{im}^{\dagger}a_{im} + Q\sum_{<ij>,m}(a_{im}^{\dagger}a_{jm}^{\dagger} + a_{im}a_{jm}) \nonumber \\
           & + N\mathcal{N}\frac{zQ^2}{2J} - N\mathcal{N}S\lambda \nonumber \\
           & = \sum_{\bm{k}m}\lambda a_{\bm{k}m}^{\dagger}a_{\bm{k}m} + \frac{1}{2}zQ\gamma_{\bm{k}}(a_{\bm{k}m}^{\dagger}a_{-\bm{k}m}^{\dagger} + a_{\bm{k}m}a_{-\bm{k}m}) \nonumber \\
           & + N\mathcal{N}\frac{zQ^2}{2J} - N\mathcal{N}S\lambda 
\end{align}
\end_inset


\end_layout

\begin_layout Standard
Notice that the AFM-MF Hamiltonian contains anomalous terms because of different 
\begin_inset Formula $\mathcal{Z}$
\end_inset

 defined in 
\begin_inset CommandInset ref
LatexCommand eqref
reference "eq:casesfmafm"
plural "false"
caps "false"
noprefix "false"

\end_inset

, and 
\begin_inset Formula $1/2$
\end_inset

 comes from sum over all bonds. Consider 
\begin_inset Formula $SU(2)$
\end_inset

 Heisenberg model, i.e. 
\begin_inset Formula $m=1,2$
\end_inset

. The Hamiltonian can be diagonalized using the standard 
\shape italic
Bogoliubov transform
\shape default
,
\end_layout

\begin_layout Standard

\begin_inset Formula \begin{equation}
    \alpha_{\bm{k}m}  = \cosh\theta_{\bm{k}} a_{\bm{k}m} - \sinh\theta_{\bm{k}} a_{-\bm{k}m}^{\dagger} 
\end{equation}
\end_inset


\end_layout

\begin_layout Standard
or inversely
\end_layout

\begin_layout Standard

\begin_inset Formula \begin{equation}
    a_{\bm{k}m} = \cosh\theta_{\bm{k}}\alpha_{\bm{k}m} + \sinh\theta_{\bm{k}}\alpha_{-\bm{k}m}^{\dagger}
\end{equation}
\end_inset


\end_layout

\begin_layout Standard
In other texts, the Bogoliubov parameters are written as 
\begin_inset Formula $u_{\bm{k}}$
\end_inset

 and 
\begin_inset Formula $v_{\bm{k}}$
\end_inset

. But since 
\begin_inset Formula $u_{\bm{k}}^2-v_{\bm{k}}^2=1$
\end_inset

 to guarantee the commutation relations of the boson operators, the parametrization here is obvious. The MF Hamiltonian after the Bogoliubov transform is
\end_layout

\begin_layout Standard

\begin_inset Formula \begin{align}
    H^{MF} & = \frac{1}{2}\sum_{\bm{k}m}\bigg((\lambda\cosh2\theta_{\bm{k}} + zQ\gamma_{\bm{k}}\sinh2\theta_{\bm{k}})(\alpha_{\bm{k}m}^{\dagger}\alpha_{\bm{k}m} + \alpha_{\bm{k}m}\alpha_{\bm{k}m}^{\dagger}) \nonumber \\
           & + (\lambda\sinh2\theta_{\bm{k}} + zQ\gamma_{\bm{k}}\cosh2\theta_{\bm{k}})(\alpha_{\bm{k}m}^{\dagger}\alpha_{-\bm{k}m}^{\dagger} + \alpha_{\bm{k}m}\alpha_{-\bm{k}m})\bigg) \nonumber \\
           & + N\mathcal{N}\frac{zQ^2}{2J} - N\mathcal{N}\bigg(S+\frac{1}{2}\bigg)\lambda 
\end{align}
\end_inset


\end_layout

\begin_layout Standard
where I have written the Hamiltonian in symmetric form to introduce the 
\begin_inset Formula $1/2$
\end_inset

 in the last term. The parameter 
\begin_inset Formula $\theta_{\bm{k}}$
\end_inset

 is determined such that the MF Hamiltonian does not contain the anomalous terms, which reads
\end_layout

\begin_layout Standard

\begin_inset Formula \begin{equation}
    \tanh2\theta_{\bm{k}} = - \frac{zQ\gamma_{\bm{k}}}{\lambda}
\end{equation}
\end_inset


\end_layout

\begin_layout Standard
The Hamiltonian is now normal and diagonal,
\end_layout

\begin_layout Standard

\begin_inset Formula \begin{equation}
    H^{MF} = \sum_{\bm{k}m}\omega_{\bm{k}}\bigg(\alpha_{\bm{k}m}^{\dagger}\alpha_{\bm{k}m} + \frac{1}{2}\bigg)+ N\mathcal{N}\frac{zQ^2}{2J} - N\mathcal{N}\bigg(S+\frac{1}{2}\bigg)\lambda
\end{equation}
\end_inset


\end_layout

\begin_layout Standard
where
\end_layout

\begin_layout Standard

\begin_inset Formula \begin{align}
    \omega_{\bm{k}} & = \lambda\cosh2\theta_{\bm{k}} + zQ\gamma_{\bm{k}}\sinh2\theta_{\bm{k}} \nonumber \\
                    & = \bigg(\lambda + zQ\gamma_{\bm{k}}\bigg(-\frac{zQ\gamma_{\bm{k}}}{\lambda}\bigg)\bigg)\cosh2\theta_{\bm{k}} \nonumber \\
                    & = \bigg(\lambda + zQ\gamma_{\bm{k}}\bigg(-\frac{zQ\gamma_{\bm{k}}}{\lambda}\bigg)\bigg)\cdot\bigg(\sqrt{1-\frac{(zQ\gamma_{\bm{k}})^2}{\lambda^2}}\bigg)^{-1} \nonumber \\
                    & = \sqrt{\lambda^2 - (zQ\gamma_{\bm{k}})^2}
                    \label{eq:stepsome}
\end{align}
\end_inset


\end_layout

\begin_layout Standard
From the MF Hamiltonian we can determine the MF free energy, i.e. the MF action, as discussed in the beginning of this chapter 
\begin_inset CommandInset ref
LatexCommand eqref
reference "eq:MFandAction"
plural "false"
caps "false"
noprefix "false"

\end_inset

. The static quantities 
\begin_inset Formula $\lambda$
\end_inset

 and 
\begin_inset Formula $Q$
\end_inset

 can be determined by minimizing the MF action, which yields the 
\series bold
saddle point equations
\series default
.
\end_layout

\begin_layout Standard

\begin_inset Formula \begin{align}
    Z & = \tr(e^{-\beta H^{MF}}) \nonumber \quad \text{notice} \quad \hat{n}_{\bm{k}m} = \alpha_{\bm{k}m}^{\dagger}\alpha_{\bm{k}m}\\
      & = (\text{\rom{1}})\sum_{n_{\bm{k}=0}}^{\infty} e^{-\beta\sum_{\bm{k}m}(n_{\bm{k}m}+\frac{1}{2})\omega_{\bm{k}}} \nonumber \\
      & = (\text{\rom{1}})\sum_{n_{\bm{k}=0}}^{\infty}\prod_{\bm{k},m} e^{-\beta(n_{\bm{k}m}+\frac{1}{2})\omega_{\bm{k}}} \nonumber \\
      & = (\text{\rom{1}})\prod_{\bm{k},m}\bigg(\frac{1}{1-e^{-\frac{1}{2}\beta\omega_{\bm{k}}}} - 1\bigg) \nonumber \\
      & = (\text{\rom{1}})\prod_{\bm{k},m}\big(\sinh(\frac{1}{2}\beta\omega_{\bm{k}})\big)^{-1}
\end{align}
\end_inset


\end_layout

\begin_layout Standard
where
\end_layout

\begin_layout Standard

\begin_inset Formula \begin{equation}
    (\text{\rom{1}}) = \exp\bigg(-\beta\bigg(N\mathcal{N}\frac{zQ^2}{2J} - N\mathcal{N}\bigg(S+\frac{1}{2}\bigg)\lambda\bigg)\bigg)
\end{equation}
\end_inset


\end_layout

\begin_layout Standard
The MF free energy is then
\end_layout

\begin_layout Standard

\begin_inset Formula \begin{align}
    F^{MF} & = - \beta^{-1}\ln Z \nonumber \\
           & = \beta^{-1}\sum_{\bm{k}m}\ln\sinh(\frac{1}{2}\beta\omega_{\bm{k}})+ N\mathcal{N}\frac{zQ^2}{2J} - N\mathcal{N}\bigg(S+\frac{1}{2}\bigg)\lambda \nonumber \\
           & = \beta^{-1}N\sum_{\bm{k}}\ln\sinh(\frac{1}{2}\beta\omega_{\bm{k}})+ N\mathcal{N}\frac{zQ^2}{2J} - N\mathcal{N}\bigg(S+\frac{1}{2}\bigg)\lambda
\end{align}
\end_inset


\end_layout

\begin_layout Standard
where I have used the fact that the energy spectrum for the 
\begin_inset Formula $N$
\end_inset

 flavors are 
\series bold
degenerate
\series default
. The saddle point equations are given by differentiating 
\begin_inset Formula $F^{MF}$
\end_inset

 with respect to 
\begin_inset Formula $\lambda$
\end_inset

 and 
\begin_inset Formula $Q$
\end_inset

:
\end_layout

\begin_layout Standard

\begin_inset Formula \begin{align}
    \frac{1}{\mathcal{N}}\sum_{\bm{k}}\frac{\lambda}{\sqrt{\lambda^2 - (zQ\gamma_{\bm{k}})^2}}\bigg(n_{\bm{k}} + \frac{1}{2}\bigg) & = S + \frac{1}{2} \nonumber \\
    \frac{1}{\mathcal{N}}\sum_{\bm{k}}\frac{z^2\gamma_{\bm{k}}^2Q}{\sqrt{\lambda^2 - (zQ\gamma_{\bm{k}})^2}}\bigg(n_{\bm{k}} + \frac{1}{2}\bigg) & = \frac{zQ}{J}
    \label{eq:saddleofAFM1}
\end{align}
\end_inset


\end_layout

\begin_layout Standard
For further calculations, it is convenient to introduce the parametrizations
\end_layout

\begin_layout Standard

\begin_inset Formula \begin{align}
    \omega_{\bm{k}} & = c\sqrt{(\kappa/2)^2 + \frac{z}{2}(1-\gamma_{\bm{k}}^2)} \nonumber \\
    c               & = Q\sqrt{2z} \nonumber \\
    \kappa          & = \frac{2}{c}\sqrt{\lambda^2 - (zQ)^2} \nonumber \\
    t               & = \frac{T}{zQ}
    \label{eq:physicalparametrization}
\end{align}
\end_inset


\end_layout

\begin_layout Standard
where 
\begin_inset Formula $\kappa$
\end_inset

, 
\begin_inset Formula $c$
\end_inset

, and 
\begin_inset Formula $t$
\end_inset

 describe the inverse correlation length, spin wave velocity, and dimensionless temperature, respectively. It's easy to notice that near the Brillouin zone center and zone corner the mean field dispersions are those of the 
\series bold
free massive relativistic bosons
\series default

\end_layout

\begin_layout Standard

\begin_inset Formula \begin{equation}
    \omega_{\bm{k}} \simeq c\sqrt{(\kappa/2)^2 + \abs{\bm{k}-\bm{k}_{\gamma}}} \quad \bm{k}_{\gamma} = 0, \Vec{\pi}
\end{equation}
\end_inset


\end_layout

\begin_layout Standard
When the gap (mass 
\begin_inset Formula $c\kappa/2$
\end_inset

) vanishes, 
\begin_inset Formula $\omega_{\bm{k}}$
\end_inset

 are Goldstone modes with linear dispersion relations. 
\begin_inset Newline newline
\end_inset


\end_layout

\begin_layout Standard
For any spin theory, the most important quantity to calculate is the 
\shape italic
spin correlation function
\shape default
. Because it provides direct results to compare with the experiments, such as the correlation length, and the critical exponent of phase transition. Some important steps are omitted in the calculation of ferromagnetic correlation function 
\begin_inset CommandInset ref
LatexCommand eqref
reference "eq:MFcorrelationfunction"
plural "false"
caps "false"
noprefix "false"

\end_inset

. If not clarified, it's difficult to derive that for the antiferromagnetic system. 
\begin_inset Newline newline
\end_inset


\end_layout

\begin_layout Standard
Consider 
\begin_inset Formula $SU(2)$
\end_inset

 model, 
\begin_inset Formula $m = 1/2$
\end_inset

, 
\begin_inset Formula $m' = -1/2$
\end_inset

. Choose lattice 
\begin_inset Formula $i$
\end_inset

 as the reference lattice, the spin correlation function for even lattice is defined as
\end_layout

\begin_layout Standard

\begin_inset Formula \begin{align}
    S^{MF}(\bm{x}_{ij}) & \equiv \expval{\frac{1}{2}(S_i^+S_j^- + S_i^-S_j^+) + S_i^zS_j^z}^{MF} \nonumber \\
                        &  = \expval{\frac{1}{2}(a_{im}^{\dagger}a_{im'}a_{jm'}^{\dagger}a_{jm} + a_{im'}^{\dagger}a_{im}a_{jm}^{\dagger}a_{jm'})}^{MF} \nonumber \\
                        &  + \expval{\frac{1}{4}(a_{im}^{\dagger}a_{im} - a_{im'}^{\dagger}a_{im'})(a_{jm}^{\dagger}a_{jm} - a_{jm'}^{\dagger}a_{jm'})}^{MF} \quad j\in\text{even}
                        \label{eq:AFMmeanFcorrelation}
\end{align}
\end_inset


\end_layout

\begin_layout Standard
while the spin correlation function for odd lattice is defined as
\end_layout

\begin_layout Standard

\begin_inset Formula \begin{align}
    S^{MF}(\bm{x}_{ij}) & \equiv \expval{\frac{1}{2}(S_i^+S_j^- + S_i^-S_j^+) + S_i^zS_j^z}^{MF} \nonumber \\
                        &  = \expval{\frac{1}{2}(a_{im}^{\dagger}a_{im'}(-)a_{jm}^{\dagger}a_{jm'} + a_{im'}^{\dagger}a_{im}(-)a_{jm'}^{\dagger}a_{jm})}^{MF} \nonumber \\
                        &  + \expval{\frac{1}{4}(a_{im}^{\dagger}a_{im} - a_{im'}^{\dagger}a_{im'})(a_{jm'}^{\dagger}a_{jm'} - a_{jm}^{\dagger}a_{jm})}^{MF} \quad j\in\text{odd}
                        \label{eq:AFMmeanFcorrelation1}
\end{align}
\end_inset


\end_layout

\begin_layout Standard
Notice that for odd lattice, we have used the inverse of the canonical transform 
\begin_inset CommandInset ref
LatexCommand eqref
reference "eq:canonical"
plural "false"
caps "false"
noprefix "false"

\end_inset

 to recover the definition of spin operators on odd lattices. 
\begin_inset Newline newline
\end_inset


\end_layout

\begin_layout Standard
The MF average of four operators can be calculated by using 
\series bold
Hartree-Fock decoupling
\series default
, guided by the Wick theorem. The problem now reduces to the evaluation of MF average of pairs of operators. First, consider
\end_layout

\begin_layout Standard

\begin_inset Formula \begin{equation}
    \expval{a_{im}^{\dagger}a_{im}}^{MF} \equiv \frac{1}{\mathcal{N}}\sum_i \expval{a_{im}^{\dagger}a_{im}}
\end{equation}
\end_inset


\end_layout

\begin_layout Standard
where 
\begin_inset Formula $\expval{...}$
\end_inset

 is the 
\series bold
ensemble average
\series default
, and 
\begin_inset Formula $\mathcal{N}$
\end_inset

 is the number of sites of the system. This expression matches the definition of MF, since it averages over all sites. To discuss the collective behaviors, we need to introduce Fourier transform
\end_layout

\begin_layout Standard

\begin_inset Formula \begin{equation}
    a_{im} = \frac{1}{\sqrt{\mathcal{N}}}\sum_{\bm{k}} a_{\bm{k}m} e^{-i\bm{k}\cdot\bm{x}_i}
\end{equation}
\end_inset


\end_layout

\begin_layout Standard
For AFM system, we should be careful with the prefactor of the transform. Since we have already performed a canonical transform to make the bond operators symmetric, the operators of two sublattices are equivalent. The normalization constant obtained from periodic boundary condition should be 
\begin_inset Formula $1/\sqrt{\mathcal{N}}$
\end_inset

. Had we started with the asymmetric form of the bond operators, the constant is 
\begin_inset Formula $1/\sqrt{\mathcal{N}/2}$
\end_inset

. Let's go back to the calculation,
\end_layout

\begin_layout Standard

\begin_inset Formula \begin{align}
    \expval{a_{im}^{\dagger}a_{im}}^{MF} & \equiv \frac{1}{\mathcal{N}}\sum_i \expval{a_{im}^{\dagger}a_{im}} \nonumber \\
                                         & = \frac{1}{\mathcal{N}^2}\sum_i \sum_{\bm{k}\bm{k}'}\expval{a_{\bm{k}m}^{\dagger}a_{\bm{k}'m} e^{i(\bm{k}-\bm{k}')\cdot\bm{x}_i}} \nonumber \\
                                         & = \frac{1}{\mathcal{N}^2}\sum_{\bm{k}\bm{k}'}\expval{a_{\bm{k}m}^{\dagger}a_{\bm{k}'m} \delta_{\bm{k},\bm{k}'}} \nonumber \\
                                         & = \frac{1}{\mathcal{N}}\sum_{\bm{k}}\expval{a_{\bm{k}m}^{\dagger}a_{\bm{k}m}} \quad \text{number of wave vectors in BZ is}: \mathcal{N}\nonumber \\
                                         & = \frac{1}{\mathcal{N}}\sum_{\bm{k}}\expval{(\cosh\theta_{\bm{k}}\alpha_{\bm{k}m}^{\dagger} + \sinh\theta_{\bm{k}}\alpha_{-\bm{k}m})(\cosh\theta_{\bm{k}}\alpha_{\bm{k}m} + \sinh\theta_{\bm{k}}\alpha_{-\bm{k}m}^{\dagger})} \nonumber \\
                                         & = \frac{1}{\mathcal{N}}\sum_{\bm{k}}\expval{\cosh^2\theta_{\bm{k}}\alpha_{\bm{k}m}^{\dagger}\alpha_{\bm{k}m} +  \sinh^2\theta_{\bm{k}}\alpha_{-\bm{k}m}\alpha_{-\bm{k}m}^{\dagger})} \nonumber \\
                                         & = \frac{1}{\mathcal{N}}\sum_{\bm{k}}\expval{\cosh^2\theta_{\bm{k}}\alpha_{\bm{k}m}^{\dagger}\alpha_{\bm{k}m} +  \sinh^2\theta_{\bm{k}}\alpha_{\bm{k}m}\alpha_{\bm{k}m}^{\dagger})} \nonumber \\
                                         & = \frac{1}{\mathcal{N}}\sum_{\bm{k}}\expval{\cosh^2\theta_{\bm{k}}\alpha_{\bm{k}m}^{\dagger}\alpha_{\bm{k}m} +  \sinh^2\theta_{\bm{k}}(1+\alpha_{\bm{k}m}^{\dagger}\alpha_{\bm{k}m})} \nonumber \\
                                         & = \frac{1}{\mathcal{N}}\sum_{\bm{k}}\expval{(\cosh^2\theta_{\bm{k}}+\sinh^2\theta_{\bm{k}})\alpha_{\bm{k}m}^{\dagger}\alpha_{\bm{k}m} + \sinh^2\theta_{\bm{k}}} \nonumber \\
                                         & = \frac{1}{\mathcal{N}}\sum_{\bm{k}}\expval{\cosh2\theta_{\bm{k}}\alpha_{\bm{k}m}^{\dagger}\alpha_{\bm{k}m} + \frac{\cosh2\theta_{\bm{k}}-1}{2}} \nonumber \\
                                         & = \frac{1}{\mathcal{N}}\sum_{\bm{k}}\expval{\cosh2\theta_{\bm{k}}\bigg(\alpha_{\bm{k}m}^{\dagger}\alpha_{\bm{k}m}+\frac{1}{2}\bigg))- \frac{1}{2}} \nonumber \\
                                         & = \frac{1}{\mathcal{N}}\sum_{\bm{k}}\expval{\cosh2\theta_{\bm{k}}\bigg(\alpha_{\bm{k}m}^{\dagger}\alpha_{\bm{k}m}+\frac{1}{2}\bigg)}- \frac{1}{2} \nonumber \\
                                         & = \frac{1}{\mathcal{N}}\sum_{\bm{k}}\cosh2\theta_{\bm{k}}\bigg(n_{\bm{k}}+\frac{1}{2}\bigg) - \frac{1}{2}
                                         \label{eq:twooperatorsMF}
\end{align}
\end_inset


\end_layout

\begin_layout Standard
where 
\begin_inset Formula $n_{\bm{k}}$
\end_inset

 is the Bose-Einstein distribution (ensemble average of boson number operators) defined in 
\begin_inset CommandInset ref
LatexCommand eqref
reference "eq:distributionbose"
plural "false"
caps "false"
noprefix "false"

\end_inset

. Next, I will calculate
\end_layout

\begin_layout Standard

\begin_inset Formula \begin{equation}
   \expval{a_{im}^{\dagger}a_{jm}}^{MF} \quad \text{for} \quad i\neq j
\end{equation}
\end_inset


\end_layout

\begin_layout Standard
All the steps should be the same except the extra lattice factor 
\begin_inset Formula $e^{i\bm{k}\cdot\bm{x}_{ij}}$
\end_inset

 (
\begin_inset Formula $\bm{x}_{ij} = \bm{x}_i - \bm{x}_j$
\end_inset

 obtained in Fourier transform. The ensemble average of the extra factor
\end_layout

\begin_layout Standard

\begin_inset Formula \begin{equation}
    \frac{1}{\mathcal{N}}\sum_{\bm{k}} e^{i\bm{k}\cdot\bm{x}_{ij}} = 0
\end{equation}
\end_inset


\end_layout

\begin_layout Standard
due to the inversion symmetry of a normal lattice. In summary
\end_layout

\begin_layout Standard

\begin_inset Formula \begin{equation}
    \expval{a_{im}^{\dagger}a_{jm}}^{MF} =  \frac{1}{\mathcal{N}}\sum_{\bm{k}}\cosh2\theta_{\bm{k}}\bigg(n_{\bm{k}}+\frac{1}{2}\bigg)e^{i\bm{k}\cdot\bm{x}_{ij}} - \frac{1}{2}\delta_{ij}
\end{equation}
\end_inset


\end_layout

\begin_layout Standard
Follow the detail calculations given in 
\begin_inset CommandInset ref
LatexCommand eqref
reference "eq:twooperatorsMF"
plural "false"
caps "false"
noprefix "false"

\end_inset

, we find the general expression (even for 
\begin_inset Formula $i=j$
\end_inset

) is
\end_layout

\begin_layout Standard

\begin_inset Formula \begin{equation}
    \expval{a_{im}^{\dagger}a_{jm}^{\dagger}}^{MF} = \expval{a_{im}a_{jm}}^{MF} = \frac{1}{\mathcal{N}}\sum_{\bm{k}}\sinh2\theta_{\bm{k}}\bigg(n_{\bm{k}}+\frac{1}{2}\bigg)e^{i\bm{k}\cdot\bm{x}_{ij}}
\end{equation}
\end_inset


\end_layout

\begin_layout Standard
All other MF expectation values are 
\series bold
zero
\series default
 for 
\begin_inset Formula $m \neq m'$
\end_inset

. The H-F decoupling of even sublattice (
\begin_inset Formula $i\neq j$
\end_inset

) spin correlation 
\begin_inset CommandInset ref
LatexCommand eqref
reference "eq:AFMmeanFcorrelation"
plural "false"
caps "false"
noprefix "false"

\end_inset

 function is given by
\end_layout

\begin_layout Standard

\begin_inset Formula \begin{equation}
    \frac{3}{2} \abs{f(\bm{x}_{ij})}^2 + \frac{1}{2}\abs{g(\bm{x}_{ij})}^2
\end{equation}
\end_inset


\end_layout

\begin_layout Standard
H-F decoupling for odd sublattice spin correlation function 
\begin_inset CommandInset ref
LatexCommand eqref
reference "eq:AFMmeanFcorrelation1"
plural "false"
caps "false"
noprefix "false"

\end_inset

 is given by
\end_layout

\begin_layout Standard

\begin_inset Formula \begin{equation}
    -\frac{1}{2}\abs{f(\bm{x}_{ij})}^2- \frac{3}{2}\abs{g(\bm{x}_{ij})}^2
\end{equation}
\end_inset


\end_layout

\begin_layout Standard
where
\end_layout

\begin_layout Standard

\begin_inset Formula \begin{equation}
   f(\bm{x}_{ij})  =  \frac{1}{\mathcal{N}}\sum_{\bm{k}}\cosh2\theta_{\bm{k}}\bigg(n_{\bm{k}}+\frac{1}{2}\bigg)e^{i\bm{k}\cdot\bm{x}_{ij}} 
\end{equation}
\end_inset


\end_layout

\begin_layout Standard

\begin_inset Formula \begin{equation}
  \cosh2\theta_{\bm{k}} = \bigg(\sqrt{1-\bigg(\frac{zQ\gamma_{\bm{k}}}{\lambda}\bigg)^2}\bigg)^{-1}
\end{equation}
\end_inset


\end_layout

\begin_layout Standard
the expression of 
\begin_inset Formula $\cosh2\theta_{\bm{k}}$
\end_inset

 is derived in 
\begin_inset CommandInset ref
LatexCommand eqref
reference "eq:stepsome"
plural "false"
caps "false"
noprefix "false"

\end_inset

. But we want to express everything in terms of physical parameters introduced in 
\begin_inset CommandInset ref
LatexCommand eqref
reference "eq:physicalparametrization"
plural "false"
caps "false"
noprefix "false"

\end_inset

, thus we should solve for 
\begin_inset Formula $Q/\lambda = a$
\end_inset

,
\end_layout

\begin_layout Standard

\begin_inset Formula \begin{align}
    \kappa & = \frac{2}{c}\sqrt{\lambda^2 - (zQ)^2} \nonumber \\
           & = \frac{2\lambda}{\sqrt{2z}Q}\sqrt{ 1 - (zQ)^2/\lambda^2} \nonumber \\
           & = \frac{2}{\sqrt{2z}a}\sqrt{1-z^2a^2}
\end{align}
\end_inset


\end_layout

\begin_layout Standard

\begin_inset Formula \begin{equation}
    a^2 = Q^2/\lambda^2 = \frac{2}{(\kappa^2+2z)z}
\end{equation}
\end_inset


\end_layout

\begin_layout Standard
We conclude that
\end_layout

\begin_layout Standard

\begin_inset Formula \begin{equation}
   f(\bm{x}_{ij}) =  \frac{1}{\mathcal{N}}\sum_{\bm{k}} \frac{\bigg(n_{\bm{k}}+\frac{1}{2}\bigg)e^{i\bm{k}\cdot\bm{x}_{ij}}}{\sqrt{1-\gamma_{\bm{k}}^2(\kappa^2/(2z)+1)^{-1}}}
\end{equation}
\end_inset


\end_layout

\begin_layout Standard
and
\end_layout

\begin_layout Standard

\begin_inset Formula \begin{align}
   g(\bm{x}_{ij}) & = \frac{1}{\mathcal{N}}\sum_{\bm{k}}\sinh2\theta_{\bm{k}}\bigg(n_{\bm{k}}+\frac{1}{2}\bigg)e^{i\bm{k}\cdot\bm{x}_{ij}} \nonumber \\
                  & = \frac{1}{\mathcal{N}}\sum_{\bm{k}}\sqrt{1-\cosh^22\theta_{\bm{k}}}\bigg(n_{\bm{k}}+\frac{1}{2}\bigg)e^{i\bm{k}\cdot\bm{x}_{ij}} \nonumber\\
                  & = \frac{1}{\mathcal{N}}\sum_{\bm{k}} \gamma_{\bm{k}}\frac{\bigg(n_{\bm{k}}+\frac{1}{2}\bigg)e^{i\bm{k}\cdot\bm{x}_{ij}}}{\sqrt{1-\gamma_{\bm{k}}^2(\kappa^2/(2z)+1)^{-1}}}
\end{align}
\end_inset


\end_layout

\begin_layout Standard
Combine the results from even (including the on-site correlation) and odd sublattices, we find that
\end_layout

\begin_layout Standard

\begin_inset Formula \begin{equation}
    S^{MF}(\bm{x}_{ij}) = \abs{f(\bm{x}_{ij})}^2 - \abs{g(\bm{x}_{ij})}^2 - \frac{1}{4}\delta_{ij}
\end{equation}
\end_inset


\end_layout

\begin_layout Standard
Finally, we are ready to discuss the predictions given by the spin correlation function. At low temperature and long distances, only long wavelength modes contribute. The dispersion relation 
\begin_inset CommandInset ref
LatexCommand eqref
reference "eq:physicalparametrization"
plural "false"
caps "false"
noprefix "false"

\end_inset

 can be expanded near the BZ center and the integration limit can be extended to the whole momentum space, because the integrand decays very fast with respect to 
\begin_inset Formula $k$
\end_inset

. I will prove this argument.
\end_layout

\begin_layout Standard
We can exploit some symmetries before explicit calculations. For a normal lattice with inverse symmetry, we should conclude
\end_layout

\begin_layout Standard

\begin_inset Formula \begin{equation}
    \gamma_{\bm{k}} = \gamma_{-\bm{k}}
\end{equation}
\end_inset


\end_layout

\begin_layout Standard
thus both
\end_layout

\begin_layout Standard

\begin_inset Formula \begin{equation}
   \cosh2\theta_{\bm{k}} = \frac{\lambda}{\sqrt{\lambda^2 - (zQ\gamma_{\bm{k}})^2}}
\end{equation}
\end_inset


\end_layout

\begin_layout Standard
and
\end_layout

\begin_layout Standard

\begin_inset Formula \begin{equation}
   \sinh2\theta_{\bm{k}} = \frac{z\gamma_{\bm{k}}Q}{\sqrt{\lambda^2 - (zQ\gamma_{\bm{k}})^2}}
\end{equation}
\end_inset


\end_layout

\begin_layout Standard
are even functions in first BZ. Then
\end_layout

\begin_layout Standard

\begin_inset Formula \begin{equation}
  f(\bm{x}_{ij}) =  \frac{2}{\mathcal{N}}\sum'_{\bm{k}} \frac{\bigg(n_{\bm{k}}+\frac{1}{2}\bigg)\cos(i\bm{k}\cdot\bm{x}_{ij})}{\sqrt{1-\gamma_{\bm{k}}^2(\kappa^2/(2z)+1)^{-1}}}  
\end{equation}
\end_inset


\end_layout

\begin_layout Standard
where 
\begin_inset Formula $\sum'$
\end_inset

 indicates sum over half of BZ. Same for 
\begin_inset Formula $g(\bm{x}_{ij})$
\end_inset

.
\end_layout

\begin_layout Standard

\begin_inset Formula \begin{equation}
  g(\bm{x}_{ij}) =  \frac{2}{\mathcal{N}}\sum'_{\bm{k}} \gamma_{\bm{k}}\frac{\bigg(n_{\bm{k}}+\frac{1}{2}\bigg)\cos(i\bm{k}\cdot\bm{x}_{ij})}{\sqrt{1-\gamma_{\bm{k}}^2(\kappa^2/(2z)+1)^{-1}}}
\end{equation}
\end_inset


\end_layout

\begin_layout Standard
For simplicity, consider the case of one-dimension. 
\begin_inset Formula $\cos(k\cdot x_{ij})$
\end_inset

 are even for 
\begin_inset Formula $j \in \text{even}$
\end_inset

, odd for 
\begin_inset Formula $j \in \text{odd}$
\end_inset

. The result is given by counting the number of nodes. Notice that the wave vectors in the BZ are given by 
\begin_inset Formula $k = \frac{2n\pi}{\mathcal{N}a}$
\end_inset

. If 
\begin_inset Formula $j \in \text{even}$
\end_inset

, the number of nodes of 
\begin_inset Formula $\cos(k\cdot x_{ij})$
\end_inset

 in 
\begin_inset Formula $(0,\pi/a)$
\end_inset

 is even , and 
\shape italic
vice versa
\shape default
. On the other hand, 
\begin_inset Formula $\gamma_k = -\gamma_{\pi-k}$
\end_inset

 for square lattice. This indicates that 
\begin_inset Formula $\gamma_{\bm{k}}$
\end_inset

 is odd in half of BZ. In summary, 
\begin_inset Formula $f(\bm{x}_{ij})$
\end_inset

 vanishes for 
\begin_inset Formula $j \in \text{odd}$
\end_inset

, while 
\begin_inset Formula $g(\bm{x}_{ij})$
\end_inset

 vanishes for 
\begin_inset Formula $j \in \text{even}$
\end_inset

.
\end_layout

\begin_layout Standard

\begin_inset Formula \begin{align}
   f(\bm{x}_{ij}) & =  \frac{2}{\mathcal{N}}\sum'_{\bm{k}} \frac{\bigg(n_{\bm{k}}+\frac{1}{2}\bigg)\cos(\bm{k}\cdot\bm{x}_{ij})}{\sqrt{1-\gamma_{\bm{k}}^2(\kappa^2/(2z)+1)^{-1}}} \nonumber \\
                  & = 2\int'\frac{d^dk}{(2\pi)^d}\frac{\bigg(n_{\bm{k}}+\frac{1}{2}\bigg)\cos(\bm{k}\cdot\bm{x}_{ij})}{\sqrt{1-\gamma_{\bm{k}}^2(\kappa^2/(2z)+1)^{-1}}} \nonumber \\
                  & = 2\int'\frac{d^dk}{(2\pi)^d}\frac{\bigg(n_{\bm{k}}+\frac{1}{2}\bigg)e^{i\bm{k}\cdot\bm{x}_{ij}}}{\sqrt{1-\gamma_{\bm{k}}^2(\kappa^2/(2z)+1)^{-1}}}
\end{align}
\end_inset


\end_layout

\begin_layout Standard
Notice that for non-vanishing 
\begin_inset Formula $f(\bm{x}_{ij})$
\end_inset

, i.e. 
\begin_inset Formula $j\in \text{even}$
\end_inset

, 
\begin_inset Formula $\sin (\bm{k}\cdot\bm{x}_{ij})$
\end_inset

 is odd, which justifies the last step. At low temperature and large distance, 
\begin_inset Formula $n_{\bm{k}} + 1/2 \simeq T/\omega_{\bm{k}}$
\end_inset

,
\end_layout

\begin_layout Standard

\begin_inset Formula \begin{align}
    \cosh2\theta_{\bm{k}}(n_{\bm{k}} + 1/2) & \simeq \lambda T/\omega_{\bm{k}}^2 \nonumber \\
                                            & \simeq \frac{\lambda T}{c^2}\frac{1}{(\kappa/2)^2 + k^2} \nonumber \\
                                            & = \frac{\lambda zQt}{Q^2 2z}\frac{1}{(\kappa/2)^2 + k^2} \nonumber \\
                                            & = \frac{t\lambda}{2}Q\frac{1}{(\kappa/2)^2 + k^2} \nonumber \\
                                            & = \frac{t}{2}\sqrt{\frac{(\kappa^2+2z)z}{2}}\frac{1}{(\kappa/2)^2 + k^2} \nonumber \\
                                            &  \simeq \frac{tz}{2}\frac{1}{(\kappa/2)^2 + k^2}
\end{align}
\end_inset


\end_layout

\begin_layout Standard

\begin_inset Formula \begin{align}
   f(\bm{x}_{ij}) & \simeq zt (1 + e^{i\bm{\pi}\cdot\bm{x}_{ij}})\int'\frac{d^dk}{(2\pi)^d}\frac{e^{i\bm{k}\cdot\bm{x}_{ij}}}{(\kappa/2)^2 + k^2} \nonumber \\
                  & \propto (1 + e^{i\bm{\pi}\cdot\bm{x}_{ij}}) (\abs{x_{ij}}/\xi)^{-(d-1)/2}e^{-\abs{x_{ij}}\kappa/2}
\end{align}
\end_inset


\end_layout

\begin_layout Standard
The integral is evaluated by residue theorem in hyper-spherical coordinates, similar to 
\begin_inset CommandInset ref
LatexCommand eqref
reference "eq:rediduetheorem"
plural "false"
caps "false"
noprefix "false"

\end_inset

. Similarly,
\end_layout

\begin_layout Standard

\begin_inset Formula \begin{equation}
   g(\bm{x}_{ij})  \propto (1 - e^{i\bm{\pi}\cdot\bm{x}_{ij}}) (\abs{x_{ij}}/\xi)^{-(d-1)/2}e^{-\abs{x_{ij}}\kappa/2}
\end{equation}
\end_inset


\end_layout

\begin_layout Standard
Thus,
\end_layout

\begin_layout Standard

\begin_inset Formula \begin{equation}
    S^{MF}(\bm{x}_{ij}) \propto e^{i\bm{\pi}\cdot\bm{x}_{ij}}\bigg(\frac{\xi}{\abs{x_{ij}}}\bigg)^{d-1}e^{-\abs{x_{ij}}/\xi}
\end{equation}
\end_inset


\end_layout

\begin_layout Standard
where 
\begin_inset Formula $\xi = \kappa^{-1}$
\end_inset

. The sign of the correlation function makes sense. Because for even lattice the correlation should be positive, while for odd lattice the correlation should be negative. In momentum space, the correlation function (static structure factor) is given by
\end_layout

\begin_layout Standard

\begin_inset Formula \begin{align}
    S^{MF}(\bm{q}) & = \sum_{\bm{r}}e^{i\bm{q}\cdot\bm{r}} S^{MF}(\bm{x}_{0\bm{r}}) \nonumber \\
                   & = \sum_{\bm{r}}\big(e^{i\bm{q}\cdot\bm{r}}(f+g)(f-g) - \frac{1}{4}\delta_{0\bm{r}}\big) \nonumber \\
                   & = \frac{1}{\mathcal{N}^2}\sum_{\bm{k}\bm{k}'}\sum_{\bm{r}}e^{i(\bm{q}-\bm{k}-\bm{k'})\cdot\bm{r}} \nonumber\\
                   & \cdot \big((\cosh2\theta_{\bm{k}}+\sinh2\theta_{\bm{k}})(\cosh2\theta_{\bm{k}'}-\sinh2\theta_{\bm{k}'})(n_{\bm{k}}+1/2)(n_{\bm{k}'}+1/2)- \frac{1}{4}\delta_{0\bm{r}}\big) \nonumber \\
                   & = \frac{1}{\mathcal{N}^2}\sum_{\bm{k}\bm{k}'}\delta_{\bm{k}',\bm{q}-\bm{k}} \nonumber\\
                   & \cdot \big((\cosh2\theta_{\bm{k}}+\sinh2\theta_{\bm{k}})(\cosh2\theta_{\bm{k}'}-\sinh2\theta_{\bm{k}'})(n_{\bm{k}}+1/2)(n_{\bm{k}'}+1/2)- \frac{1}{4}\big) \nonumber \\
                   & = \frac{1}{\mathcal{N}}\sum_{\bm{k}}(\cosh2\theta_{\bm{k}}+\sinh2\theta_{\bm{k}}) \nonumber \\
                   & \cdot(\cosh2\theta_{\bm{q}-\bm{k}}-\sinh2\theta_{\bm{q}-\bm{k}})(n_{\bm{k}}+1/2)(n_{\bm{q}-\bm{k}}+1/2)- \frac{1}{4} \nonumber \\
                   & = \frac{1}{\mathcal{N}}\sum_{\bm{k}}\cosh(2\theta_{\bm{k}}-2\theta_{\bm{q}-\bm{k}})(n_{\bm{k}}+1/2)(n_{\bm{q}-\bm{k}}+1/2)- \frac{1}{4}
\end{align}
\end_inset


\end_layout

\begin_layout Standard
where in the last step, I have used the fact 
\begin_inset Formula $\bm{k}$
\end_inset

 is dummy variable, thus 
\begin_inset Formula $\bm{k} \leftrightarrow \bm{q}-\bm{k}$
\end_inset

 does not change the result. Using the constraint 
\begin_inset CommandInset ref
LatexCommand eqref
reference "eq:saddleofAFM1"
plural "false"
caps "false"
noprefix "false"

\end_inset

, it's also easy to check
\end_layout

\begin_layout Standard

\begin_inset Formula \begin{equation}
    \frac{1}{\mathcal{N}}\sum_{\bm{q}}S^{MF}(\bm{q}) = S(S+1)
\end{equation}
\end_inset


\end_layout

\begin_layout Standard
the MF sum rule exceeds the exact result by a factor of 
\begin_inset Formula $3/2$
\end_inset

.
\end_layout

\begin_layout Section

\begin_inset Formula $1/N$
\end_inset

-correction to the Schwinger boson theory
\end_layout

\begin_layout Standard
The starting point of large 
\begin_inset Formula $N$
\end_inset

 expansion is the path integral of the auxiliary fields given in 
\begin_inset CommandInset ref
LatexCommand eqref
reference "eq:action"
plural "false"
caps "false"
noprefix "false"

\end_inset

,
\end_layout

\begin_layout Standard

\begin_inset Formula \begin{equation}
     \mathcal{S} = -\frac{1}{N}\tr\ln(\hat{G}[j]\epsilon) + \int_0^{\beta}d\tau\bigg(\sum_{<ij>}\frac{\abs{Q_{ij}}^2}{J}-iS\sum_i\lambda_i\bigg)
\end{equation}
\end_inset


\end_layout

\begin_layout Standard

\begin_inset Formula $\eta = 1$
\end_inset

 for bosonic systems, as it is here. The symbol 
\begin_inset Formula $\hat{G}[j]\epsilon$
\end_inset

 is in fact a representation of an operator of size 
\begin_inset Formula $[(\beta/\epsilon)\mathcal{N}N] \times [(\beta/\epsilon)\mathcal{N}N]$
\end_inset

. The generating functional is given by
\end_layout

\begin_layout Standard

\begin_inset Formula \begin{align}
    Z[j] & = \int\mathcal{D}^2Q\mathcal{D}\lambda \exp(-N\mathcal{S}[\lambda,Q,j]) \nonumber \\
         & = \int\mathcal{D}^2Q\mathcal{D}\lambda Z_B \exp\bigg(-N\int_0^{\beta}d\tau\bigg(\sum_{<ij>}\frac{\abs{Q_{ij}}^2}{J}-iS\sum_i\lambda_i\bigg)\bigg)
    \label{eq:generating3}
\end{align}
\end_inset


\end_layout

\begin_layout Standard
where I have defined the generating functional of the bosonic fields
\end_layout

\begin_layout Standard

\begin_inset Formula \begin{equation}
    Z_B = e^{-N(-\frac{1}{N}\tr\ln(\hat{G}[j]\epsilon))} = e^{\tr\ln(\hat{G}[j]\epsilon)} = \det (\hat{G}[j]\epsilon)
\end{equation}
\end_inset


\end_layout

\begin_layout Standard
I have used the Jacobi formula for any invertible matrix 
\begin_inset Formula $A$
\end_inset


\end_layout

\begin_layout Standard

\begin_inset Formula \begin{equation}
    \tr (\log A) = \log (\det A)
\end{equation}
\end_inset


\end_layout

\begin_layout Standard
Thus, we can define the free energy
\end_layout

\begin_layout Standard

\begin_inset Formula \begin{equation}
    F_B = -\ln Z_B = - \ln \det (\hat{G}[j]\epsilon) = \ln (\hat{G}[j]\epsilon)^{-1}
\end{equation}
\end_inset


\end_layout

\begin_layout Standard
Thus
\end_layout

\begin_layout Standard

\begin_inset Formula \begin{equation}
    \mathcal{S} = \frac{1}{N}F_B + \int_0^{\beta}d\tau\bigg(\sum_{<ij>}\frac{\abs{Q_{ij}}^2}{J}-iS\sum_i\lambda_i\bigg)
    \label{eq:exactFBaction}
\end{equation}
\end_inset


\end_layout

\begin_layout Standard
We conclude that the free energy 
\begin_inset Formula $F_B$
\end_inset

 is the action 
\begin_inset Formula $\mathcal{S}_B$
\end_inset

, which is 
\series bold
quadratic
\series default
 in the boson fields 
\begin_inset Formula $z$
\end_inset

.
\end_layout

\begin_layout Standard

\begin_inset Formula \begin{equation}
    F_B[\lambda,Q,Q^*,j] = \mathcal{S}_B[\lambda,Q,Q^*,j] = \int_0^{\beta}d\tau' L_B[\lambda,Q,Q^*,j]
    \label{eq:FBfree}
\end{equation}
\end_inset


\end_layout

\begin_layout Standard
where 
\begin_inset Formula $L_B$
\end_inset

 is defined in 
\begin_inset CommandInset ref
LatexCommand eqref
reference "eq:lagrangianofHS"
plural "false"
caps "false"
noprefix "false"

\end_inset

, which is quadratic in the fields 
\begin_inset Formula $z$
\end_inset

.
\end_layout

\begin_layout Subsection
Saddle point equation
\end_layout

\begin_layout Standard
The condition for the saddle points are
\end_layout

\begin_layout Standard

\begin_inset Formula \begin{equation}
    \frac{\delta\mathcal{S}[j=0]}{\delta \lambda_i(\tau)} = \frac{\delta\mathcal{S}[j=0]}{\delta Q_{ij}(\tau)} = \frac{\delta\mathcal{S}[j=0]}{\delta Q_{ij}^*(\tau)} = 0
\end{equation}
\end_inset


\end_layout

\begin_layout Standard
Refer to 
\begin_inset CommandInset ref
LatexCommand eqref
reference "eq:lagrangianofHS"
plural "false"
caps "false"
noprefix "false"

\end_inset


\end_layout

\begin_layout Standard

\begin_inset Formula \begin{align}
   \frac{\delta\mathcal{S}[j=0]}{\delta \lambda_i(\tau)} & = -\frac{1}{N}Z_B^{-1}\int\mathcal{D}^2\bm{z}(-i)\sum_m z^*_{im}(\tau)z_{im}(\tau)e^{-\int_0^{\beta}d\tau' L_B[j=0]} - iS  \nonumber \\
  & = i\bigg(\frac{1}{N}\sum_m\expval{z^*_{im}(\tau)z_{im}(\tau)}_{B,j=0}-S\bigg)  \\
  \frac{\delta\mathcal{S}[j=0]}{\delta Q^*_{ij}(\tau)} & = - \frac{1}{N}Z_B^{-1}\int\mathcal{D}^2\bm{z} \mathcal{Z}_{ij}(\tau)e^{-\int_0^{\beta}d\tau' L_B[j=0]} + \frac{Q_{ij}(\tau)}{J}  \nonumber\\
  & = - \frac{1}{N}\sum_m \expval{z_{im}^*(\tau)z_{jm}(\tau)}_{B,j=0} + \frac{Q_{ij}(\tau)}{J} \quad \text{FM-B} \\
  & = - \frac{1}{N}\sum_m \expval{z_{im}(\tau)z_{jm}(\tau)}_{B,j=0} + \frac{Q_{ij}(\tau)}{J} \quad \text{AFM-B}\\
   \frac{\delta\mathcal{S}[j=0]}{\delta Q_{ij}(\tau)} & = - \frac{1}{N}Z_B^{-1}\int\mathcal{D}^2\bm{z} \mathcal{Z}^*_{ij}(\tau)e^{-\int_0^{\beta}d\tau' L_B[j=0]} + \frac{Q_{ij}^*(\tau)}{J} \nonumber \\
   & = - \frac{1}{N}\sum_m \expval{z_{im}(\tau)z_{jm}^*(\tau)}_{B,j=0} + \frac{Q_{ij}^*(\tau)}{J} \quad \text{FM-B} \\
   & =  - \frac{1}{N}\sum_m \expval{z_{im}^*(\tau)z_{jm}^*(\tau)}_{B,j=0} + \frac{Q_{ij}(\tau)}{J} \quad \text{AFM-B} 
   \label{eq:allsaddlepointseq}
\end{align}
\end_inset


\end_layout

\begin_layout Subsection
Expansion of the action near the saddle point
\end_layout

\begin_layout Standard
Let 
\begin_inset Formula $(\lambda_{\text{sp}},Q_{\text{sp}},Q^*_{\text{sp}})$
\end_inset

 be a saddle point of 
\begin_inset Formula $\mathcal{S}[j=0]$
\end_inset

 determined by 
\begin_inset CommandInset ref
LatexCommand eqref
reference "eq:allsaddlepointseq"
plural "false"
caps "false"
noprefix "false"

\end_inset

. Below we 
\series bold
assume
\series default
 
\begin_inset Formula $(\lambda_{\text{sp}},Q_{\text{sp}},Q^*_{\text{sp}})$
\end_inset

 has neither temporal nor spatial dependence, while such an extension will not cause a big difference. By introducing a short-hand vector notation for the deviation of the auxiliary fields with respect to the saddle point
\end_layout

\begin_layout Standard

\begin_inset Formula \begin{equation}
    \phi(\tau) = \bigg(i\lambda_i(\tau) - i\lambda_{\text{sp}},Q_{ij}(\tau) -Q_{\text{sp}},Q^*_{ij}(\tau)-Q^*_{\text{sp}}\bigg)
\end{equation}
\end_inset


\end_layout

\begin_layout Standard
we expand 
\begin_inset Formula $\mathcal{S}[j]$
\end_inset

 around the saddle point. To begin with, we start from the evaluation of the Taylor expansion of 
\begin_inset Formula $F_B[\phi,j]$
\end_inset

 around 
\begin_inset Formula $\phi=0$
\end_inset

. Refer to 
\begin_inset CommandInset ref
LatexCommand eqref
reference "eq:FBfree"
plural "false"
caps "false"
noprefix "false"

\end_inset

, let the operator absorbs 
\begin_inset Formula $\epsilon$
\end_inset

,
\end_layout

\begin_layout Standard

\begin_inset Formula \begin{align}
    F_B[\phi,j] & = \ln \det \hat{G}^{-1}[j] \nonumber \\
                & = \ln \det \{\hat{G}_{\text{sp}}^{-1}[j][1+\hat{G}_{sp}[j](\hat{G}^{-1}[j] - \hat{G}^{-1}_{\text{sp}}[j])]\} \nonumber \\
                & = \Tr \ln \{\hat{G}_{\text{sp}}^{-1}[j][1+\hat{G}_{\text{sp}}[j](\hat{G}^{-1}[j] - \hat{G}^{-1}_{\text{sp}}[j])]\} \nonumber \\ 
                & = \Tr \ln \hat{G}_{\text{sp}}^{-1} + \Tr \ln [1+\hat{G}_{\text{sp}}[j](\hat{G}^{-1}[j] - \hat{G}^{-1}_{\text{sp}}[j])] \nonumber \\
                & = \ln \det \hat{G}_{\text{sp}}^{-1} + \Tr \ln [1+\hat{G}_{\text{sp}}[j](\hat{G}^{-1}[j] - \hat{G}^{-1}_{\text{sp}}[j])] \nonumber \\
                & = F_B[\phi = 0, j] + \Tr\ln\bigg(1+\hat{G}_{\text{sp}}[j]\sum_{\alpha}\phi_{\alpha}\Hat{v}_{\alpha}\bigg)
                \label{eq:FBphi}
\end{align}
\end_inset


\end_layout

\begin_layout Standard
where we have used the definition 
\begin_inset CommandInset ref
LatexCommand eqref
reference "eq:FMGinverse"
plural "false"
caps "false"
noprefix "false"

\end_inset


\end_layout

\begin_layout Standard

\begin_inset Formula \begin{align}
    \hat{G}^{-1}[j] - \hat{G}^{-1}_{\text{sp}}[j] & = \hat{\lambda} - \hat{\lambda}_{\text{sp}} + \hat{Q} - \hat{Q}_{\text{sp}} \nonumber \\
    & = \sum_{\alpha}\phi_{\alpha} \left.\frac{\partial \Hat{G}^{-1}[j]}{\partial\phi_{\alpha}}\right\vert_{\phi=0} = \sum_{\alpha}\phi_{\alpha}\Hat{v}_{\alpha}
\end{align}
\end_inset


\end_layout

\begin_layout Standard
this identity should be understood as
\end_layout

\begin_layout Standard

\begin_inset Formula \begin{equation}
    \int_{x_\text{sp}}^x dx \frac{\partial \hat{G}^{-1}[j]}{\partial x} =  \hat{G}^{-1}[j] - \hat{G}^{-1}_{\text{sp}}[j]
\end{equation}
\end_inset


\end_layout

\begin_layout Standard
where 
\begin_inset Formula $\alpha$
\end_inset

 specifies the auxiliary field component of 
\begin_inset Formula $\phi$
\end_inset

 as well as its imaginary time 
\begin_inset Formula $\tau$
\end_inset

. 
\begin_inset Formula $\hat{v}_{\alpha}$
\end_inset

 is an operator called the 
\shape italic
internal vertex
\shape default
.
\end_layout

\begin_layout Standard
The matrix elements of this operators are
\end_layout

\begin_layout Standard

\begin_inset Formula \begin{equation}
    v_{\alpha}^{\xi\eta} = \begin{cases}
    -\epsilon_{\tau}\sum_m \delta_{\xi,(n_{\tau},i,m)}\delta_{\eta,(n_{\tau},i,m)} &\text{for} \quad\alpha = \alpha_{\lambda_i(n_{\tau}\epsilon_{\tau})}\\
    \epsilon_{\tau}\sum_m \delta_{\xi,(n_{\tau},j,m)}\delta_{\eta,(n_{\tau},i,m)} & \text{for} \quad\alpha = \alpha_{Q_{ij}(n_{\tau}\epsilon_{\tau})}\\
     \epsilon_{\tau}\sum_m \delta_{\xi,(n_{\tau},i,m)}\delta_{\eta,(n_{\tau},j,m)} & \text{for} \quad\alpha = \alpha_{Q^*_{ij}(n_{\tau}\epsilon_{\tau})}
    \end{cases}
\end{equation}
\end_inset


\end_layout

\begin_layout Standard
where all the flavors at the same space-(imaginary)time are summed over.
\end_layout

\begin_layout Standard
Since the index 
\begin_inset Formula $\alpha$
\end_inset

 specifies both the space-time and the auxiliary field component, we can introduce additional notations for the component of 
\begin_inset Formula $\phi$
\end_inset

:
\end_layout

\begin_layout Standard

\begin_inset Formula \begin{align}
    \phi_{\alpha\lambda_i(\tau)} & = i\lambda_i(n_{\tau}\epsilon_{\tau}) - i\lambda_{\text{sp}} \nonumber \\
    \phi_{\alpha Q_{ij}(\tau)} & =  Q_{ij}(n_{\tau}\epsilon_{\tau}) - Q_{\text{sp}} \nonumber \\
    \phi_{\alpha Q^*_{ij}(\tau)} & =  Q^*_{ij}(n_{\tau}\epsilon_{\tau}) - Q^*_{\text{sp}}
\end{align}
\end_inset


\end_layout

\begin_layout Standard
where 
\begin_inset Formula $\sum_{\tau} n_{\tau} \epsilon_{\tau} = \beta$
\end_inset

. We can view 
\begin_inset Formula $\alpha$
\end_inset

 as a polarization (includes space-(imaginary)time, component of auxiliary field, and summation of flavors) of the fluctuation of the auxiliary fields. Then the vertices should carry two indices to connect two polarizations.
\end_layout

\begin_layout Standard
Then
\end_layout

\begin_layout Standard

\begin_inset Formula \begin{align}
   &F_B[\phi,j] - F_B[\phi = 0, j] \nonumber \\
   & = \Tr\ln\bigg(1+\hat{G}_{\text{sp}}[j]\sum_{\alpha}\phi_{\alpha}\Hat{v}_{\alpha}\bigg) \nonumber \\
   & = \sum_{n=1}^{\infty}\frac{(-1)^{n+1}}{n}\Tr\bigg(\hat{G}_{\text{sp}}[j]\sum_{\alpha}\phi_{\alpha}\Hat{v}_{\alpha}\bigg)^n \nonumber\\
   & = \sum_{n=1}^{\infty}\frac{(-1)^{n+1}}{n}\sum_{\alpha_1,...,\alpha_n}\phi_{\alpha_1}...\phi_{\alpha_n}\Tr(\hat{G}_{\text{sp}}[j]\hat{v}_{\alpha_1}...\hat{G}_{\text{sp}}[j]\hat{v}_{\alpha_n}) \nonumber\\
   & =  \sum_{n=1}^{\infty}\frac{(-1)^{n+1}}{n}\frac{1}{n!}\sum_{\alpha_1,...,\alpha_n}\phi_{\alpha_1}...\phi_{\alpha_n}\sum_{P(\alpha_1,...,\alpha_n)}\Tr(\hat{G}_{\text{sp}}[j]\hat{v}_{\alpha_{P_1}}...\hat{G}_{\text{sp}}[j]\hat{v}_{\alpha_{P_n}})
\end{align}
\end_inset


\end_layout

\begin_layout Standard
where we have used the result of Taylor expansion of logarithm of matrix,
\end_layout

\begin_layout Standard

\begin_inset Formula \begin{equation}
    \Tr\ln(1+A) = \sum_{n=1}^{\infty}\frac{(-1)^{n+1}}{n}\Tr(A)^n
\end{equation}
\end_inset


\end_layout

\begin_layout Standard
the permutations of 
\begin_inset Formula $(\alpha_1,...\alpha_n)$
\end_inset

 is denoted by 
\begin_inset Formula $P(\alpha_1,...\alpha_n)$
\end_inset

. Plug the results of the expansion back to the expression of the exact action 
\begin_inset Formula $\mathcal{S}[j]$
\end_inset

 
\begin_inset CommandInset ref
LatexCommand eqref
reference "eq:exactFBaction"
plural "false"
caps "false"
noprefix "false"

\end_inset

,
\end_layout

\begin_layout Standard

\begin_inset Formula \begin{align}
    \mathcal{S}[j] & = \frac{1}{N}F_B[\phi = 0, j] \nonumber \\
    & + \frac{1}{N}\sum_{n=1}^{\infty}\frac{(-1)^{n+1}}{n}\frac{1}{n!}\sum_{\alpha_1,...,\alpha_n}\phi_{\alpha_1}...\phi_{\alpha_n}\sum_{P(\alpha_1,...,\alpha_n)}\Tr(\hat{G}_{\text{sp}}[j]\hat{v}_{\alpha_{P_1}}...\hat{G}_{\text{sp}}[j]\hat{v}_{\alpha_{P_n}}) \nonumber \\
    & + \int_0^{\beta}d\tau\bigg(\sum_{<ij>}\frac{\abs{Q_{ij}}^2}{J}-iS\sum_i\lambda_i\bigg) \nonumber \\
    & \equiv \sum_{n=0}^{\infty}\frac{1}{n!}\sum_{\alpha_1,...,\alpha_n}\phi_{\alpha_1}...\phi_{\alpha_n}\mathcal{S}^{(n)}_{\alpha_1...\alpha_n}[j]
    \label{eq:sjneq0}
\end{align}
\end_inset


\end_layout

\begin_layout Standard
To find the explicit expression of 
\begin_inset Formula $\mathcal{S}^{(n)}_{\alpha_1...\alpha_n}[j]$
\end_inset

, notice that
\end_layout

\begin_layout Standard

\begin_inset Formula \begin{align}
    & \int_0^{\beta}d\tau\bigg(\sum_{<ij>}\frac{\abs{Q_{ij}}^2}{J}-iS\sum_i\lambda_i\bigg) \nonumber \\
    & = \sum_{n_{\tau}} \epsilon_{\tau} \bigg(\sum_{<ij>}\frac{\abs{\phi_{\alpha Q_{ij}(\tau)}+ Q_{\text{sp}}}^2}{J} \nonumber \\
    & - S\sum_i (\phi_{\alpha \lambda_i(\tau)} + i\lambda_{\text{sp}})\bigg)
    \label{eq:S[j]}
\end{align}
\end_inset


\end_layout

\begin_layout Standard
Thus
\end_layout

\begin_layout Standard

\begin_inset Formula \begin{align}
    \mathcal{S}^{(0)}[j] & = \mathcal{S}_{\text{sp}}[j] \nonumber \\
    \mathcal{S}_{\alpha}^{(1)}[j] & = \frac{1}{N}\Tr(\hat{G}_{\text{sp}}[j]\hat{v}_{\alpha}) \nonumber \\
                         & + \sum_{n_{\tau}} \epsilon_{\tau}\bigg[\sum_{<ij>}\bigg(\frac{Q^*_{\text{sp}}}{J}\delta_{\alpha,\alpha Q_{ij}(\tau)} + \frac{Q_{\text{sp}}}{J}\delta_{\alpha,\alpha Q^*_{ij}(\tau)}\bigg) - S\sum_i\delta_{\alpha,\alpha\lambda_i(\tau)}\bigg] \nonumber \\
    \mathcal{S}_{\alpha\alpha'}^{(2)}[j] & = - \frac{1}{N}\Tr(\hat{G}_{\text{sp}}[j]\hat{v}_{\alpha}\hat{G}_{\text{sp}}[j]\hat{v}_{\alpha'}) \nonumber \\
                         & + \sum_{n_{\tau}} \epsilon_{\tau} \sum_{<ij>}\frac{1}{J}\bigg(\delta_{\alpha,\alpha Q_{ij}(\tau)}\delta_{\alpha',\alpha Q^*_{ij}(\tau)} + \delta_{\alpha,\alpha Q^*_{ij}(\tau)}\delta_{\alpha',\alpha Q_{ij}(\tau)}\bigg) \nonumber \\
                         & = 2(\hat{\Pi}_0 - \hat{\Pi})_{\alpha\alpha'} \nonumber \\
    \mathcal{S}_{\alpha_1...\alpha_n}^{(n\geq 3)}[j] & = \frac{(-1)^{n+1}}{nN}\sum_{P(\alpha_1,...,\alpha_n)}\Tr(\hat{G}_{\text{sp}}[j]\hat{v}_{\alpha_{P_1}}...\hat{G}_{\text{sp}}[j]\hat{v}_{\alpha_{P_n}}) 
    \label{eq:Sorderbyorder}
\end{align}
\end_inset


\end_layout

\begin_layout Standard
where we have defined the 
\shape italic
polarization operator
\shape default
,
\end_layout

\begin_layout Standard

\begin_inset Formula \begin{equation}
    \hat{\Pi}_{\alpha\alpha'}(j) = \frac{1}{2N}\Tr(\hat{G}_{\text{sp}}[j]\hat{v}_{\alpha}\hat{G}_{\text{sp}}[j]\hat{v}_{\alpha'})
\end{equation}
\end_inset


\end_layout

\begin_layout Standard
and
\end_layout

\begin_layout Standard

\begin_inset Formula \begin{equation}
   (\hat{\Pi_0})_{\alpha\alpha'} = \sum_{n_{\tau}} \epsilon_{\tau} \sum_{<ij>}\frac{1}{2J}\bigg(\delta_{\alpha,\alpha Q_{ij}(\tau)}\delta_{\alpha',\alpha Q^*_{ij}(\tau)} + \delta_{\alpha,\alpha Q^*_{ij}(\tau)}\delta_{\alpha',\alpha Q_{ij}(\tau)}\bigg)
\end{equation}
\end_inset


\end_layout

\begin_layout Standard
Because of the cyclic symmetry of the trace, we can represent 
\begin_inset Formula $\mathcal{S}_{\alpha_1...\alpha_n}^{(n\geq 3)}[j = 0]$
\end_inset

 as well as 
\begin_inset Formula $\hat{\Pi}$
\end_inset

 by loop diagrams.
\end_layout

\begin_layout Standard
By sending 
\begin_inset Formula $j=0$
\end_inset

, the linear term should vanish, since the expansion is around the saddle point of the source-free action. Thus
\end_layout

\begin_layout Standard

\begin_inset Formula \begin{align}
    \mathcal{S}[j=0] & = \sum_{n\neq 1} \frac{1}{n!}\sum_{\alpha_1...\alpha_n}\phi_{\alpha_1}...\phi_{\alpha_n}\mathcal{S}_{\alpha_1...\alpha_n}^{(n)}[j=0] \nonumber \\
                     & \equiv \mathcal{S}_{\text{sp}}[j=0] + \mathcal{S}_{\text{RPA}} + \mathcal{S}_{\text{int}}
                     \label{eq:sj=0}
\end{align}
\end_inset


\end_layout

\begin_layout Standard
where 
\begin_inset Formula $\mathcal{S}_{\text{RPA}}$
\end_inset

 is the action in the random-phase approximation (RPA) describing quadratic fluctuations around the saddle point
\end_layout

\begin_layout Standard

\begin_inset Formula \begin{align}
    \mathcal{S}_{\text{RPA}} & \equiv \frac{1}{2}\sum_{\alpha\alpha'}\mathcal{S}_{\alpha\alpha'}^{(2)}[j=0] \nonumber \\
                             & = \sum_{\alpha\alpha'}(\hat{\Pi}_0 - \hat{\Pi}[j=0])_{\alpha\alpha'}\phi_{\alpha}\phi_{\alpha'} \nonumber \\
                             & = \sum_{\alpha\alpha'}D^{-1}_{\alpha\alpha'}\phi_{\alpha}\phi_{\alpha'}
\end{align}
\end_inset


\end_layout

\begin_layout Standard
where
\end_layout

\begin_layout Standard

\begin_inset Formula \begin{equation}
    D_{\alpha\alpha'} = (\hat{\Pi}_0 - \hat{\Pi}[j=0])^{-1}_{\alpha\alpha'}
\end{equation}
\end_inset


\end_layout

\begin_layout Standard
is called the RPA propagator. The cubic and higher order terms in the auxiliary fields are denoted as 
\begin_inset Formula $\mathcal{S}_{\text{int}}$
\end_inset

:
\end_layout

\begin_layout Standard

\begin_inset Formula \begin{equation}
   \mathcal{S}_{\text{int}} = \sum_{n\geq 3} \frac{1}{n!}\sum_{\alpha_1...\alpha_n}\phi_{\alpha_1}...\phi_{\alpha_n}\mathcal{S}_{\alpha_1...\alpha_n}^{(n)}[j=0] 
\end{equation}
\end_inset


\end_layout

\begin_layout Subsection
Correlation functions
\end_layout

\begin_layout Standard
The connected two-spin 
\series bold
dynamical correlation function
\series default
 for FM-B is given by
\end_layout

\begin_layout Standard

\begin_inset Formula \begin{align}
    R^{mm'}(1,2) & = \expval{a_{1,m}^{\dagger}a_{1,m'}a_{2,m'}^{\dagger}a_{2,m}} \nonumber \\
                 & = \expval{T_{\tau} a_{1,m}^{\dagger}a_{1,m'}a_{2,m'}^{\dagger}a_{2,m}} - \expval{a_{1,m}^{\dagger}a_{1,m'}}\expval{a_{2,m'}^{\dagger}a_{2,m}}\nonumber \\
                 & = \left.\frac{\delta^2 \ln Z[j]}{\delta j_1^{mm'} \delta j_2^{m'm}}\right\vert_{j=0}
\label{eq:dynamicalcorrelation1}
\end{align}
\end_inset


\end_layout

\begin_layout Standard
where 1, 2 denote points in space-time (imaginary). In high-energy physics, the two point correlation functions vanish because we assume that the symmetry is 
\series bold
not
\series default
 broken. Let's prove the result explicitly
\end_layout

\begin_layout Standard

\begin_inset Formula \begin{align}
   \left.\frac{\delta^2 \ln Z[j]}{\delta j_1^{\mu\nu} \delta j_2^{\nu\mu}}\right\vert_{j=0} & = \frac{\delta}{\delta j_1^{\mu\nu}}\bigg(\frac{1}{Z[j=0]}\int \mathcal{D}\lambda \mathcal{D}^2Q(-N) e^{-N\mathcal{S}}\frac{\delta \mathcal{S}}{\delta j_2^{\nu\mu}}\bigg) \nonumber \\
   & =  \frac{\delta}{\delta j_1^{\mu\nu}}\bigg(\frac{1}{Z[j=0]}\bigg) \int \mathcal{D}\lambda \mathcal{D}^2Q(-N) e^{-N\mathcal{S}}\left.\frac{\delta \mathcal{S}}{\delta j_2^{\nu\mu}}\right\vert_{j=0} \nonumber \\
   & + \frac{1}{Z[j=0]}\frac{\delta}{\delta j_1^{\mu\nu}}\bigg(\int \mathcal{D}\lambda \mathcal{D}^2Q(-N) e^{-N\mathcal{S}}\left.\frac{\delta \mathcal{S}}{\delta j_2^{\nu\mu}}\right\vert_{j=0}\bigg) \nonumber \\
   & = - \frac{1}{Z^2[j=0]}\int \mathcal{D}\lambda \mathcal{D}^2Q (-N) e^{-N\mathcal{S}}\left.\frac{\delta \mathcal{S}}{\delta j_1^{\mu\nu}}\right\vert_{j=0} \nonumber \\
   & \times \int \mathcal{D}\lambda' \mathcal{D}^2Q' (-N) e^{-N\mathcal{S}}\left.\frac{\delta \mathcal{S}}{\delta j_2^{\nu\mu}}\right\vert_{j=0} \nonumber \\
   & + \frac{1}{Z[j=0]}\int \mathcal{D}\lambda \mathcal{D}^2Q \bigg(-N \left.\frac{\delta^2\mathcal{S}[j]}{\delta j_1^{\mu\nu} \delta j_2^{\nu\mu}}\right\vert_{j=0} \nonumber \\
   & + N^2\left.\frac{\delta \mathcal{S}[j]}{\delta j_1^{\mu\nu}}\right\vert_{j=0}\left.\frac{\delta \mathcal{S}[j]}{\delta j_2^{\nu\mu}}\right\vert_{j=0}\bigg)e^{-N\mathcal{S}[j=0]} \nonumber \\
   & = \expval{T_{\tau} a_{1,\mu}^{\dagger}a_{1,\nu}a_{2,\nu}^{\dagger}a_{2,\mu}}  - \expval{a_{1,m}^{\dagger}a_{1,m'}}\expval{a_{2,m'}^{\dagger}a_{2,m}} \nonumber \\
   & \equiv R_{\text{\rom{1}}}^{\mu\nu}(1,2) + R_{\text{\rom{2}}}^{\mu\nu}(1,2) - R_{\text{\rom{3}}}^{\mu\nu}(1)R_{\text{\rom{3}}}^{\nu\mu}(2)
   \label{eq:fullspindynamicalcorrelation1}
\end{align}
\end_inset


\end_layout

\begin_layout Standard
where
\end_layout

\begin_layout Standard

\begin_inset Formula \begin{align}
    R_{\text{\rom{1}}}^{\mu\nu}(1,2) & =  \frac{1}{Z[j=0]}\int \mathcal{D}\lambda \mathcal{D}^2Q (-N) \left.\frac{\delta^2\mathcal{S}[j]}{\delta j_1^{\mu\nu} \delta j_2^{\nu\mu}}\right\vert_{j=0}e^{-N\mathcal{S}[j=0]} \nonumber \\
   R_{\text{\rom{2}}}^{\mu\nu}(1,2) & = \frac{1}{Z[j=0]}\int \mathcal{D}\lambda \mathcal{D}^2Q (N^2)\left.\frac{\delta \mathcal{S}[j]}{\delta j_1^{\mu\nu}}\right\vert_{j=0}\left.\frac{\delta \mathcal{S}[j]}{\delta j_2^{\nu\mu}}\right\vert_{j=0}e^{-N\mathcal{S}[j=0]} \nonumber \\
    R_{\text{\rom{3}}}^{\mu\nu}(1) & = \frac{1}{Z[j=0]}\int \mathcal{D}\lambda \mathcal{D}^2Q (N) \left.\frac{\delta \mathcal{S}[j]}{\delta j_1^{\mu\nu}}\right\vert_{j=0}e^{-N\mathcal{S}[j=0]}\nonumber \\
    R_{\text{\rom{3}}}^{\nu\mu}(2) & = \frac{1}{Z[j=0]}\int \mathcal{D}\lambda \mathcal{D}^2Q (N) \left.\frac{\delta \mathcal{S}[j]}{\delta j_2^{\nu\mu}}\right\vert_{j=0}e^{-N\mathcal{S}[j=0]}
    \label{eq:123correlationfunctions}
\end{align}
\end_inset


\end_layout

\begin_layout Standard
Note that 
\begin_inset Formula $R_{\text{\rom{3}}}^{\mu\nu}(1) =  R_{\text{\rom{3}}}^{\nu\mu}(2) = 0$
\end_inset

 if the symmetry of the system is not broken.
\end_layout

\begin_layout Subsubsection
Correlation functions at saddle point
\end_layout

\begin_layout Standard
First, we evaluate 
\begin_inset Formula $R^{mm'}(1,2)$
\end_inset

 at the saddle point.
\end_layout

\begin_layout Standard

\begin_inset Formula \begin{align}
    R^{mm'}(1,2) & = \left.\frac{\delta^2 \ln Z_B^{(\text{sp})}[j]}{\delta j_1^{mm'}\delta j_2^{m'm}}\right\vert_{j=0} \nonumber \\
                 & = \expval{z_m^*(1)z_{m'}(1)z_{m'}^*(2)z_m(2)}_B^{\text{sp}} - \expval{z_m^*(1)z_{m'}(1)}_B^{\text{sp}}\expval{z_{m'}^*(2)z_m(2)}_B^{\text{sp}} \nonumber \\
                 & = \expval{z_m^*(1)z_m(2)}_B^{\text{sp}}\expval{z_{m'}^*(2)z_{m'}(1)}_B^{\text{sp}}
                 \label{eq:saddlespindynamical}
\end{align}
\end_inset


\end_layout

\begin_layout Standard
where in the last step we have used the property of Gaussian integral. Since at the saddle point, the Schwinger bosons don't interact with the auxiliary fields (which are static fields determined by the saddle point equations 
\begin_inset CommandInset ref
LatexCommand eqref
reference "eq:allsaddlepointseq"
plural "false"
caps "false"
noprefix "false"

\end_inset

), thus the integral is Gaussian.
\end_layout

\begin_layout Standard
We interpret the result as an exchange of 
\series bold
free
\series default
 spinons. The following discussion is an alternative derivation of 
\begin_inset CommandInset ref
LatexCommand eqref
reference "eq:saddlespindynamical"
plural "false"
caps "false"
noprefix "false"

\end_inset

, which provides an expression with unified style as in the 
\begin_inset Formula $1/N$
\end_inset

 correlations we will discuss later. Similar to the introduction of internal vertex operator, which is related to the difference of the exact inverse Green function operator and the inverse of the saddle point Green function operator, we can introduce the 
\shape italic
external vertex
\shape default
 operator, which is related to the difference of the inverse of the Green function operator and the inverse of the source-free Green function operator. From 
\begin_inset CommandInset ref
LatexCommand eqref
reference "eq:FMGinverse"
plural "false"
caps "false"
noprefix "false"

\end_inset

, we see that
\end_layout

\begin_layout Standard

\begin_inset Formula \begin{equation}
    \hat{G}^{-1}_{\text{sp}}[j] - \hat{G}^{-1}_{\text{sp}}[j = 0] = - \hat{j}
\end{equation}
\end_inset


\end_layout

\begin_layout Standard
Define
\end_layout

\begin_layout Standard

\begin_inset Formula \begin{align}
   \hat{j} & \equiv - \sum_{n_{\tau},i,m,m'}j_i^{mm'}(n_{\tau}\epsilon_{\tau})\frac{\partial \hat{G}^{-1}}{\partial j_i^{mm'}(n_{\tau}\epsilon_{\tau})} \nonumber \\
           & = \sum_{n_{\tau},i,m,m'}j_i^{mm'}(n_{\tau}\epsilon_{\tau})\Hat{u}_i^{mm'}(n_{\tau}\epsilon_{\tau})
\end{align}
\end_inset


\end_layout

\begin_layout Standard
Just a reminder, 
\begin_inset Formula $\Hat{j}$
\end_inset

 has the same dimension as 
\begin_inset Formula $\hat{G}^{-1}_{\text{sp}}[j]$
\end_inset

: 
\begin_inset Formula $[(\beta/\epsilon)\mathcal{N}N] \times [(\beta/\epsilon)\mathcal{N}N]$
\end_inset

. The matrix elements of the external vertex operator are
\end_layout

\begin_layout Standard

\begin_inset Formula \begin{equation}
    \big[\Hat{u}_i^{mm'}(n_{\tau}\epsilon_{\tau})\big]_{\xi\eta} = - \frac{\partial \hat{G}_{\xi\eta}^{-1}}{\partial j_i^{mm'}(n_{\tau}\epsilon_{\tau})} = \delta_{\xi,(n_{\tau},i,m)}\delta_{\eta,(n_{\tau},i,m')}
    \label{eq:externalvertices}
\end{equation}
\end_inset


\end_layout

\begin_layout Standard
The external vertices carry the flavor indices. Notice that
\end_layout

\begin_layout Standard

\begin_inset Formula \begin{equation}
    -N\mathcal{S}_{\text{sp}}[j] = \ln Z_B^{\text{sp}}[j]
\end{equation}
\end_inset


\end_layout

\begin_layout Standard
To find the spin dynamical correlation function, we can also express 
\begin_inset Formula $\mathcal{S}_{\text{sp}}[j]$
\end_inset

 in terms of the external vertices. Below, (...) stands for the term 
\begin_inset CommandInset ref
LatexCommand eqref
reference "eq:S[j]"
plural "false"
caps "false"
noprefix "false"

\end_inset

 of the saddle point action, which is independent of 
\begin_inset Formula $j$
\end_inset

. Follow the method given in 
\begin_inset CommandInset ref
LatexCommand eqref
reference "eq:FBphi"
plural "false"
caps "false"
noprefix "false"

\end_inset

,
\end_layout

\begin_layout Standard

\begin_inset Formula \begin{align}
    \mathcal{S}_{\text{sp}}[j] & = \frac{1}{N}F_B[\phi=0,j] + (...) \nonumber \\
    & = \frac{1}{N}\ln\det(\Hat{G}^{-1}_{\text{sp}}[j]) + (...) \nonumber \\
    & = \frac{1}{N}\ln\det\bigg\{\Hat{G}^{-1}_{\text{sp}}[j=0]\big[1 + \Hat{G}_{\text{sp}}[j=0](\Hat{G}^{-1}_{\text{sp}}[j] - \Hat{G}^{-1}_{\text{sp}}[j=0])\big]\bigg\} + (...) \nonumber \\
    & = \frac{1}{N}\Tr\ln\Hat{G}^{-1}_{\text{sp}}[j=0] + \frac{1}{N}\Tr\ln\big[1 + \Hat{G}_{\text{sp}}[j=0](\Hat{G}^{-1}_{\text{sp}}[j] - \Hat{G}^{-1}_{\text{sp}}[j=0])\big] + (...) \nonumber \\
    & = \frac{1}{N}F_B[\phi=0,j=0] + \frac{1}{N}\Tr\ln(1-\Hat{G}_{\text{sp}}[j=0]\Hat{j})  + (...) \nonumber \\
    & = \frac{1}{N}F_B[\phi=0,j=0] - \frac{1}{N}\sum_{n=1}^{\infty}\frac{1}{n}\Tr(\Hat{G}_{\text{sp}}[j=0]\Hat{j})^n+ (...) 
    \label{eq:Sspfb}
\end{align}
\end_inset


\end_layout

\begin_layout Standard
Thus,
\end_layout

\begin_layout Standard

\begin_inset Formula \begin{align}
    R_{\text{sp}}^{mm'}(1,2) & = -N\left.\frac{\delta^2\mathcal{S}_{\text{sp}}[j]}{\delta j_1^{mm'}\delta j_2^{m'm}}\right\vert_{j=0} \nonumber \\
    & = \frac{1}{2}\left.\frac{\delta^2}{\delta j_1^{mm'}\delta j_2^{m'm}}\Tr\big(\Hat{G}_{\text{sp}}[j=0]\Hat{j}\Hat{G}_{\text{sp}}[j=0]\Hat{j}\big)\right\vert_{j=0} \nonumber \\
    & = \frac{1}{2}\frac{\delta^2}{\delta j_1^{mm'}\delta j_2^{m'm}}\Tr\bigg[\Hat{G}_{\text{sp}}[j=0](\Hat{G}^{-1}_{\text{sp}}[j] - \Hat{G}^{-1}_{\text{sp}}[j=0]) \nonumber \\
    & \times \Hat{G}_{\text{sp}}[j=0](\Hat{G}^{-1}_{\text{sp}}[j] - \Hat{G}^{-1}_{\text{sp}}[j=0])\bigg]\vert_{j=0} \nonumber \\
    & = \frac{1}{2}\frac{\delta}{\delta j_1^{mm'}}\bigg(\Tr\bigg[\Hat{G}_{\text{sp}}[j=0](-)\frac{\partial\Hat{G}^{-1}_{\text{sp}}[j]}{\partial j_2^{mm'}} \nonumber \\
    & \times \Hat{G}_{\text{sp}}[j=0]\Hat{j} + \Hat{G}_{\text{sp}}[j=0]\Hat{j}\Hat{G}_{\text{sp}}[j=0](-)\frac{\partial\Hat{G}^{-1}_{\text{sp}}[j]}{\partial j_2^{mm'}}\bigg]\bigg)\vert_{j=0} \nonumber \\
    & = \Tr\bigg(\Hat{G}_{\text{sp}}[j=0](-)\frac{\partial \Hat{G}^{-1}_{\text{sp}}[j]}{\partial j_1^{mm'}}\Hat{G}_{\text{sp}}[j=0](-)\frac{\partial \Hat{G}^{-1}_{\text{sp}}[j]}{\partial j_2^{m'm}}\bigg) \nonumber \\
    & = \Tr\bigg(\Hat{G}_{\text{sp}}[j=0]\Hat{u}_1^{mm'}\Hat{G}_{\text{sp}}[j=0]\Hat{u}_2^{m'm}\bigg)
\end{align}
\end_inset


\end_layout

\begin_layout Standard
In the above calculations, we have focused only on the 
\begin_inset Formula $n=2$
\end_inset

 term of 
\begin_inset CommandInset ref
LatexCommand eqref
reference "eq:Sspfb"
plural "false"
caps "false"
noprefix "false"

\end_inset

. Because only this term contains non-zero contribution when sending 
\begin_inset Formula $j=0$
\end_inset

. Again, we recover the same interpretation given by 
\begin_inset CommandInset ref
LatexCommand eqref
reference "eq:saddlespindynamical"
plural "false"
caps "false"
noprefix "false"

\end_inset

: the saddle point spin dynamical correlation functions correspond to the exchange of free spinons between the two external sources.
\end_layout

\begin_layout Subsubsection

\begin_inset Formula $1/N$
\end_inset

 corrections to the correlation functions
\end_layout

\begin_layout Standard
For simplicity, we introduce short-handed notations
\end_layout

\begin_layout Standard

\begin_inset Formula \begin{align}
    \mathcal{S}^{(n;1)}_{\alpha_1...\alpha_n;(1,\mu\nu)} & = N\left.\frac{\delta\mathcal{S}^{(n)}_{\alpha_1...\alpha_n}[j]}{\delta j_1^{\mu\nu}}\right\vert_{j=0} \nonumber \\
    \mathcal{S}^{(n;2)}_{\alpha_1...\alpha_n;(1,\mu\nu),(2,\nu\mu)} & = N\left.\frac{\delta^2\mathcal{S}^{(n)}_{\alpha_1...\alpha_n}[j]}{\delta j_1^{\mu\nu}\delta j_2^{\nu\mu}}\right\vert_{j=0}
\end{align}
\end_inset


\end_layout

\begin_layout Standard
We expand 
\begin_inset Formula $\mathcal{S}[j]$
\end_inset

 using 
\begin_inset CommandInset ref
LatexCommand eqref
reference "eq:sjneq0"
plural "false"
caps "false"
noprefix "false"

\end_inset

 and 
\begin_inset Formula $\mathcal{S}[j=0]$
\end_inset

 using 
\begin_inset CommandInset ref
LatexCommand eqref
reference "eq:sj=0"
plural "false"
caps "false"
noprefix "false"

\end_inset

, the exact spin dynamical correlation functions 
\begin_inset CommandInset ref
LatexCommand eqref
reference "eq:123correlationfunctions"
plural "false"
caps "false"
noprefix "false"

\end_inset

 take the forms:
\end_layout

\begin_layout Standard

\begin_inset Formula \begin{align}
    R^{\mu\nu}_{\text{{1}}}(1,2) & = Z^{-1}\int\mathcal{D}\phi\bigg(-\sum_{n=0}^{\infty}\frac{1}{n!}\phi_{\alpha_1}...\phi_{\alpha_n}\mathcal{S}^{(n;2)}_{\alpha_1...\alpha_n;(1,\mu\nu),(2,\nu\mu)}\bigg) \nonumber \\
    & \times \bigg[\sum_{L=0}^{\infty}\frac{(-N)^L}{L!}(\mathcal{S}_{\text{int}})^L\bigg]\exp[-N(\mathcal{S}_{\text{sp}}[j=0] + \mathcal{S}_{\text{RPA}})] \nonumber \\
    R^{\mu\nu}_{\text{\rom{2}}}(1,2) & = Z^{-1}\int\mathcal{D}\phi\bigg(\sum_{n=0}^{\infty}\frac{1}{n!}\phi_{\alpha_1}...\phi_{\alpha_n}\mathcal{S}^{(n;1)}_{\alpha_1...\alpha_n;(1,\mu\nu)}\bigg)\bigg(\sum_{m=0}^{\infty}\frac{1}{m!}\phi_{\alpha_1}...\phi_{\alpha_m}\mathcal{S}^{(m;1)}_{\alpha_1...\alpha_m;(2,\nu\mu)}\bigg) \nonumber \\
    & \times \bigg[\sum_{L=0}^{\infty}\frac{(-N)^L}{L!}(\mathcal{S}_{\text{int}})^L\bigg]\exp[-N(\mathcal{S}_{\text{sp}}[j=0] + \mathcal{S}_{\text{RPA}})] \nonumber \\
     R^{\mu\nu}_{\text{\rom{3}}}(1) & = Z^{-1}\int\mathcal{D}\phi\bigg(\sum_{n=0}^{\infty}\frac{1}{n!}\phi_{\alpha_1}...\phi_{\alpha_n}\mathcal{S}^{(n;1)}_{\alpha_1...\alpha_n;(1,\mu\nu)}\bigg) \nonumber \\
     & \times \bigg[\sum_{L=0}^{\infty}\frac{(-N)^L}{L!}(\mathcal{S}_{\text{int}})^L\bigg]\exp[-N(\mathcal{S}_{\text{sp}}[j=0] + \mathcal{S}_{\text{RPA}})]
     \label{eq:321correlationfunctions}
\end{align}
\end_inset


\end_layout

\begin_layout Standard
where we have replaced the measure 
\begin_inset Formula $\int \mathcal{D}\lambda\mathcal{D}^2Q$
\end_inset

 by the measure of the fluctuation fields 
\begin_inset Formula $\mathcal{D}\phi$
\end_inset

; also 
\begin_inset Formula $Z^{-1}$
\end_inset

 is evaluated at 
\begin_inset Formula $j=0$
\end_inset

.
\end_layout

\begin_layout Standard
Refer to 
\begin_inset CommandInset ref
LatexCommand eqref
reference "eq:sjneq0"
plural "false"
caps "false"
noprefix "false"

\end_inset

 and 
\begin_inset CommandInset ref
LatexCommand eqref
reference "eq:Sorderbyorder"
plural "false"
caps "false"
noprefix "false"

\end_inset

, the 
\begin_inset Formula $j$
\end_inset

-derivative of 
\begin_inset Formula $\mathcal{S}^{n\geq1}_{\alpha_1...\alpha_n}$
\end_inset

 
\shape italic
i.e.
\shape default

\end_layout

\begin_layout Standard

\begin_inset Formula \[\mathcal{S}^{(n;n)}_{\alpha_1...\alpha_n;(1,\mu_1\nu_1),(2,\mu_2\nu_2)...(n,\mu_n\nu_n)}\]
\end_inset


\end_layout

\begin_layout Standard
need to be evaluated in the trace of 
\begin_inset Formula $\hat{G}_{\text{sp}}[j]\hat{v}_{\alpha_{P_1}}...\hat{G}_{\text{sp}}[j]\hat{v}_{\alpha_{P_n}}$
\end_inset

. Using the identity
\end_layout

\begin_layout Standard

\begin_inset Formula \begin{equation}
    \sum_{\xi'}(\Hat{G}_{\text{sp}})_{\xi\xi'}(\Hat{G}^{-1}_{\text{sp}})_{\xi'\zeta} = \delta_{\xi\zeta}
\end{equation}
\end_inset


\end_layout

\begin_layout Standard
We find that
\end_layout

\begin_layout Standard

\begin_inset Formula \begin{equation}
     \sum_{\xi'} \frac{\partial(\Hat{G}_{\text{sp}})_{\xi\xi'}}{\partial j_i^{\mu\nu}(n_{\tau}\epsilon_{\tau})}(\Hat{G}^{-1}_{\text{sp}})_{\xi'\zeta} + (\Hat{G}_{\text{sp}})_{\xi\xi'}\frac{\partial(\Hat{G}^{-1}_{\text{sp}})_{\xi'\zeta}}{\partial j_i^{\mu\nu}(n_{\tau}\epsilon_{\tau})} = 0
     \label{eq:usefulidentityforG}
\end{equation}
\end_inset


\end_layout

\begin_layout Standard
The second term of 
\begin_inset CommandInset ref
LatexCommand eqref
reference "eq:usefulidentityforG"
plural "false"
caps "false"
noprefix "false"

\end_inset

 matches the definition of the 
\shape italic
external vertices
\shape default
 given in 
\begin_inset CommandInset ref
LatexCommand eqref
reference "eq:externalvertices"
plural "false"
caps "false"
noprefix "false"

\end_inset

, thus
\end_layout

\begin_layout Standard

\begin_inset Formula \begin{equation}
   \frac{\partial(\Hat{G}_{\text{sp}})_{\xi\xi'}}{\partial j_i^{\mu\nu}(n_{\tau}\epsilon_{\tau})}(\Hat{G}^{-1}_{\text{sp}})_{\xi'\zeta} = (\Hat{G}_{\text{sp}})_{\xi\xi'}\big[\Hat{u}_i^{\mu\nu}(n_{\tau}\epsilon_{\tau})\big]_{\xi'\zeta}
\end{equation}
\end_inset


\end_layout

\begin_layout Standard
multiply the above equation by 
\begin_inset Formula $(\Hat{G}_{\text{sp}})_{\xi'\zeta}$
\end_inset

 on both sides and then sum over 
\begin_inset Formula $\zeta$
\end_inset

, we have
\end_layout

\begin_layout Standard

\begin_inset Formula \begin{equation}
     \frac{\partial(\Hat{G}_{\text{sp}})_{\xi\xi'}}{\partial j_i^{\mu\nu}(n_{\tau}\epsilon_{\tau})} = \sum_{\zeta} (\Hat{G}_{\text{sp}})_{\xi\xi'}\big[\Hat{u}_i^{\mu\nu}(n_{\tau}\epsilon_{\tau})\big]_{\xi'\zeta}(\Hat{G}_{\text{sp}})_{\xi'\zeta}
\end{equation}
\end_inset


\end_layout

\begin_layout Standard
showing that taking the functional derivative of a saddle point propagator with respect to 
\begin_inset Formula $j$
\end_inset

 results in splitting it into two by inserting an external vertex 
\begin_inset Formula $\Hat{u}_i^{\mu\nu}(n_{\tau}\epsilon_{\tau})$
\end_inset

. Before further calculation, it's important to notice that the 
\shape italic
internal vertex
\shape default
 
\begin_inset Formula $\Hat{v}_{\alpha}$
\end_inset

 does not depend on the current by definition. Because it is the difference of the exact inverse propagator and the saddle point inverse propagator, which cancels the 
\begin_inset Formula $j$
\end_inset

-dependence. Thus, we can write down some low-order derivatives explicitly by refer to 
\begin_inset CommandInset ref
LatexCommand eqref
reference "eq:Sorderbyorder"
plural "false"
caps "false"
noprefix "false"

\end_inset

 again:
\end_layout

\begin_layout Standard

\begin_inset Formula \begin{align}
    \mathcal{S}^{(1;1)}_{\alpha;(1,\mu\nu)} & = N\left.\frac{\delta S^{(1)}_{\alpha}[j]}{\delta j_1^{\mu\nu}}\right\vert_{j=0} = \left.\frac{\delta}{\delta j_1^{\mu\nu}}\Tr(\Hat{G}_{\text{sp}}[j]\Hat{v}_{\alpha})\right\vert_{j=0} \nonumber \\
    & = \Tr(\Hat{G}_{\text{sp}}[j=0]\Hat{u}_1^{\mu\nu}\Hat{G}_{\text{sp}}[j=0]\Hat{v}_{\alpha})
\end{align}
\end_inset


\end_layout

\begin_layout Standard

\begin_inset Formula \begin{align}
    \mathcal{S}^{(1;2)}_{\alpha;(1,\mu\nu),(2,\nu\mu)} & = N\left.\frac{\delta^2 S^{(1)}_{\alpha}[j]}{\delta j_1^{\mu\nu}\delta j_2^{\nu\mu}}\right\vert_{j=0} = \left.\frac{\delta}{\delta j_2^{\nu\mu}}\Tr(\Hat{G}_{\text{sp}}[j]\Hat{u}_1^{\mu\nu}\Hat{G}_{\text{sp}}[j]\Hat{v}_{\alpha})\right\vert_{j=0} \nonumber \\
    & = \Tr(\Hat{G}_{\text{sp}}[j=0]\Hat{u}_2^{\nu\mu}\Hat{G}_{\text{sp}}[j=0]\Hat{u}_1^{\mu\nu}\Hat{G}_{\text{sp}}[j=0]\Hat{v}_{\alpha}) \nonumber \\
    & + \Tr(\Hat{G}_{\text{sp}}[j=0]\Hat{u}_1^{\mu\nu}\Hat{G}_{\text{sp}}[j=0]\Hat{u}_2^{\nu\mu}\Hat{G}_{\text{sp}}[j=0]\Hat{v}_{\alpha})
\end{align}
\end_inset


\end_layout

\begin_layout Standard
Similarly,
\end_layout

\begin_layout Standard

\begin_inset Formula \begin{align}
   \mathcal{S}^{(2;1)}_{\alpha\alpha';(1,\mu\nu)} & = N\left.\frac{\delta S^{(2)}_{\alpha\alpha'}[j]}{\delta j_1^{\mu\nu}}\right\vert_{j=0} = -\left.\frac{\delta}{\delta j_1^{\mu\nu}}\Tr(\Hat{G}_{\text{sp}}[j]\Hat{v}_{\alpha}\Hat{G}_{\text{sp}}[j]\Hat{v}_{\alpha'})\right\vert_{j=0} \nonumber \\
   & = - \big[\Tr(\Hat{G}_{\text{sp}}[j=0]\Hat{u}_1^{\mu\nu}\Hat{G}_{\text{sp}}[j=0]\Hat{v}_{\alpha}\Hat{G}_{\text{sp}}[j=0]\Hat{v}_{\alpha'}) \nonumber \\
   & + \Tr(\Hat{G}_{\text{sp}}[j=0]\Hat{v}_{\alpha}\Hat{G}_{\text{sp}}[j=0]\Hat{u}_1^{\mu\nu}\Hat{G}_{\text{sp}}[j=0]\Hat{v}_{\alpha'})\big]
\end{align}
\end_inset


\end_layout

\begin_layout Standard

\begin_inset Formula \begin{align}
    \mathcal{S}^{(2;2)}_{\alpha\alpha';(1,\mu\nu),(2,\nu\mu)} & =  N\left.\frac{\delta^2 S^{(2)}_{\alpha\alpha'}[j]}{\delta j_1^{\mu\nu}\delta j_2^{\nu\mu}}\right\vert_{j=0} \nonumber \\
    & = -\frac{\delta}{\delta j_2^{\nu\mu}}\bigg[\Tr(\Hat{G}_{\text{sp}}\Hat{u}_1^{\mu\nu}\Hat{G}_{\text{sp}}\Hat{v}_{\alpha}\Hat{G}_{\text{sp}}\Hat{v}_{\alpha'}) + \Tr(\Hat{G}_{\text{sp}}\Hat{v}_{\alpha}\Hat{G}_{\text{sp}}\Hat{u}_1^{\mu\nu}\Hat{G}_{\text{sp}}\Hat{v}_{\alpha'})\bigg] \nonumber \\
    & = -\Tr(\Hat{G}_{\text{sp}}[j=0]\Hat{u}_2^{\nu\mu}\Hat{G}_{\text{sp}}[j=0]\Hat{u}_1^{\mu\nu}\Hat{G}_{\text{sp}}[j=0]\Hat{v}_{\alpha}\Hat{G}_{\text{sp}}[j=0]\Hat{v}_{\alpha'}) \nonumber\\
    & -\Tr(\Hat{G}_{\text{sp}}[j=0]\Hat{u}_1^{\mu\nu}\Hat{G}_{\text{sp}}[j=0]\Hat{u}_2^{\nu\mu}\Hat{G}_{\text{sp}}[j=0]\Hat{v}_{\alpha}\Hat{G}_{\text{sp}}[j=0]\Hat{v}_{\alpha'}) \nonumber \\
    & -\Tr(\Hat{G}_{\text{sp}}[j=0]\Hat{u}_1^{\mu\nu}\Hat{G}_{\text{sp}}[j=0]\Hat{v}_{\alpha}\Hat{G}_{\text{sp}}[j=0]\Hat{u}_2^{\nu\mu}\Hat{G}_{\text{sp}}[j=0]\Hat{v}_{\alpha'}) \nonumber \\
    & -\Tr(\Hat{G}_{\text{sp}}[j=0]\Hat{u}_2^{\nu\mu}\Hat{G}_{\text{sp}}[j=0]\Hat{v}_{\alpha}\Hat{G}_{\text{sp}}[j=0]\Hat{u}_1^{\mu\nu}\Hat{G}_{\text{sp}}[j=0]\Hat{v}_{\alpha'}) \nonumber \\
    & -\Tr(\Hat{G}_{\text{sp}}[j=0]\Hat{v}_{\alpha}\Hat{G}_{\text{sp}}[j=0]\Hat{u}_2^{\nu\mu}\Hat{G}_{\text{sp}}[j=0]\Hat{u}_1^{\mu\nu}\Hat{G}_{\text{sp}}[j=0]\Hat{v}_{\alpha'}) \nonumber\\
    & -\Tr(\Hat{G}_{\text{sp}}[j=0]\Hat{v}_{\alpha}\Hat{G}_{\text{sp}}[j=0]\Hat{u}_1^{\mu\nu}\Hat{G}_{\text{sp}}[j=0]\Hat{u}_2^{\nu\mu}\Hat{G}_{\text{sp}}[j=0]\Hat{v}_{\alpha'})
\end{align}
\end_inset


\end_layout

\begin_layout Standard
The forms of the three correlation functions 
\begin_inset CommandInset ref
LatexCommand eqref
reference "eq:321correlationfunctions"
plural "false"
caps "false"
noprefix "false"

\end_inset

 are standard 
\series bold
Gaussian-like
\series default
 integrals. The property of Gaussian-like integrals is that an integral over an even number of fields can be written as a sum over pairwise "
\shape italic
contractions
\shape default
":
\end_layout

\begin_layout Standard

\begin_inset Formula \begin{equation}
    \overbrace{r_1...r_{2k}} = \sum_{i_1...i_k}\overbrace{r_1r_{i_1}}...\overbrace{r_kr_{i_k}}
    \label{eq:gaussiancontraction}
\end{equation}
\end_inset


\end_layout

\begin_layout Standard
where the contraction of two variable yields
\end_layout

\begin_layout Standard

\begin_inset Formula \begin{align}
    \overbrace{\phi_1\phi_2} & \equiv Z^{-1}\int\mathcal{D}\phi\phi_1\phi_2\exp[-N\mathcal{S}_{\text{RPA}}] \nonumber \\
                             &  =  Z^{-1}\int\mathcal{D}\phi\phi_1\phi_2\exp\bigg(-N\sum_{\alpha\alpha'}\phi_{\alpha}D^{-1}_{\alpha\alpha'}\phi_{\alpha'}\bigg) \nonumber \\
                             &  = \frac{1}{N}D_{1,2}
                             \label{eq:contractionofRPA}
\end{align}
\end_inset


\end_layout

\begin_layout Standard
The arguments of the Gaussian-like integral 
\begin_inset CommandInset ref
LatexCommand eqref
reference "eq:321correlationfunctions"
plural "false"
caps "false"
noprefix "false"

\end_inset

 can be categorized diagrammatically by: 
\series bold
external loops
\series default
 (the 
\begin_inset Formula $j$
\end_inset

-derivative of the action), and the 
\series bold
internal loops
\series default
 (the expansion of 
\begin_inset Formula $\mathcal{S}_{\text{int}}^{n\geq3}$
\end_inset

). Because algebraically, both of them can be represented by traces of product of operators.
\end_layout

\begin_layout Subsubsection
Diagrammatic method
\end_layout

\begin_layout Standard
The building blocks of the Schwinger boson theory are established in previous sections. They are: 1. the spinon propagator at the saddle point 
\begin_inset Formula $\Hat{G}^{-1}_{\text{sp}}$
\end_inset

; 2. the RPA propagator 
\begin_inset Formula $D_{\alpha\alpha'}$
\end_inset

 of the fluctuating auxiliary fields; 3. internal vertices (loops); 4. external vertices (loops). 
\begin_inset Newline newline
\end_inset


\end_layout

\begin_layout Standard
According to above discussions, the recipe to construct a diagram contributing to 
\begin_inset Formula $R_{\text{\rom{1}}}$
\end_inset

 is as following: 1. pick up 
\series bold
one
\series default
 external loop with 
\series bold
two
\series default
 external vertices and 
\begin_inset Formula $n (\geq1)$
\end_inset

 internal vertices; 2. pick up 
\begin_inset Formula $L$
\end_inset

 internal loops (with 
\begin_inset Formula $n\geq3$
\end_inset

 vertices); 3. evaluate the Gaussian-like integrals, 
\shape italic
i.e.
\shape default
 Wick contract the fluctuating auxiliary fields in pairs to make RPA propagators. 
\begin_inset Newline newline
\end_inset


\end_layout

\begin_layout Standard
The recipe to construct a diagram contributing to 
\begin_inset Formula $R_{\text{\rom{2}}}$
\end_inset

 is as following: 1. pick up 
\series bold
two
\series default
 external loops with 
\series bold
one
\series default
 external vertex for each and 
\begin_inset Formula $n \geq 1$
\end_inset

 internal vertices; step 2 and 3 are the same as those of 
\begin_inset Formula $R_{\text{\rom{1}}}$
\end_inset

.
\end_layout

\begin_layout Standard
Each internal loop is proportional to 
\begin_inset Formula $N$
\end_inset

, because of the prefactor 
\begin_inset Formula $(-N)^L$
\end_inset

 for each internal loop given in 
\begin_inset CommandInset ref
LatexCommand eqref
reference "eq:321correlationfunctions"
plural "false"
caps "false"
noprefix "false"

\end_inset

. While the external loop does not have flavor symmetry, thus proportional to 
\begin_inset Formula $N^0$
\end_inset

. Also according to 
\begin_inset CommandInset ref
LatexCommand eqref
reference "eq:contractionofRPA"
plural "false"
caps "false"
noprefix "false"

\end_inset

, each RPA propagator is proportional to 
\begin_inset Formula $1/N$
\end_inset

. In summary: 
\begin_inset Newline newline
\end_inset


\end_layout

\begin_layout Standard

\begin_inset Box Boxed
position "c"
hor_pos "l"
has_inner_box 1
inner_pos "c"
use_parbox 0
use_makebox 0
width "30em"
special "none"
height "1in"
height_special "totalheight"
thickness "0.4pt"
separation "3pt"
shadowsize "4pt"
framecolor "black"
backgroundcolor "none"
status open


\begin_layout Plain Layout
 The order of a diagram with 
\begin_inset Formula $L$
\end_inset

 internal loops and 
\begin_inset Formula $P$
\end_inset

 contractions 
\begin_inset Formula $=\bigg(\frac{1}{N}\bigg)^p$
\end_inset

, 
\begin_inset Formula $p = P-L$
\end_inset

 
\end_layout

\end_inset


\end_layout

\begin_layout Standard
We will evaluate the lowest-order (
\begin_inset Formula $p=1$
\end_inset

) corrections to the dynamical correlation function. Before counting diagrams explicitly, we ask what's the minimum number of contractions needed when 
\begin_inset Formula $L$
\end_inset

 is given. Refer to the recipes to construct the diagrams, the number of contraction is minimized if the total number of internal vertices is minimized,
\end_layout

\begin_layout Standard

\begin_inset Formula \begin{equation}
    P_{\text{min}} = 
    \begin{cases}
    \frac{1}{2}(3L+1) & \text{for a diagram contributing to} \quad R_{\text{\rom{1}}} \\
    \frac{1}{2}(3L+2) & \text{for a diagram contributing to}  \quad R_{\text{\rom{2}}}
    \end{cases}
\end{equation}
\end_inset


\end_layout

\begin_layout Standard
Since 
\begin_inset Formula $P = L + p \geq P_{\text{min}}$
\end_inset

, at a given order 
\begin_inset Formula $p$
\end_inset

, we find
\end_layout

\begin_layout Standard

\begin_inset Formula \begin{equation}
    L \leq 
    \begin{cases}
    2p-1 & \text{for a diagram contributing to}   \quad R_{\text{\rom{1}}} \\
    2p-2 & \text{for a diagram contributing to}  \quad R_{\text{\rom{2}}}
        \end{cases}
\end{equation}
\end_inset


\end_layout

\begin_layout Standard
Thus, at the lowest order 
\begin_inset Formula $p=1$
\end_inset

 in 
\begin_inset Formula $(1/N)$
\end_inset

, the diagrams contributing to 
\begin_inset Formula $R_{\text{\rom{1}}}$
\end_inset

 are: (
\begin_inset Formula $P$
\end_inset

, 
\begin_inset Formula $L$
\end_inset

) = (2,1) and (1,0); and for 
\begin_inset Formula $R_{\text{\rom{2}}}$
\end_inset

, (
\begin_inset Formula $P$
\end_inset

, 
\begin_inset Formula $L$
\end_inset

) = (1,0).
\end_layout

\begin_layout Standard

\begin_inset Newpage newpage
\end_inset


\end_layout

\begin_layout Standard
\start_of_appendix

\begin_inset ERT
status collapsed

\begin_layout Plain Layout
%dummy comment inserted by tex2lyx to ensure that this paragraph is not empty
\end_layout

\begin_layout Plain Layout

\end_layout

\end_inset


\end_layout

\begin_layout Section
Method of steepest descents
\end_layout

\begin_layout Standard
In this part, I will review the 
\shape italic
method of of steepest descents
\shape default
 (MSD), which gives the asymptotic behavior of some contour integrals. The method is very useful in path integral approach. Consider contour integral of a single complex variable:
\end_layout

\begin_layout Standard

\begin_inset Formula \begin{equation}
    I(g) = \int_C dz e^{-gf(z)} 
    \label{eq:I(g)}
\end{equation}
\end_inset


\end_layout

\begin_layout Standard

\begin_inset Formula $f(z) = u + iv$
\end_inset

 is analytic in the complex plane, and 
\begin_inset Formula $C$
\end_inset

 is a contour which connects two points where 
\begin_inset Formula $\Re f \rightarrow \infty$
\end_inset

. 
\begin_inset Formula $g$
\end_inset

 is a real parameter. According to the Cauchy-Riemann conditions
\end_layout

\begin_layout Standard

\begin_inset Formula \begin{align}
    \partial_x \Re f & = \partial_y \Im f \nonumber \\
    \partial_y \Re f & = - \partial_x \Im f
\end{align}
\end_inset


\end_layout

\begin_layout Standard
which further implies that neither the real nor the imaginary part of 
\begin_inset Formula $f(z)$
\end_inset

 can possess absolute minimum since
\end_layout

\begin_layout Standard

\begin_inset Formula \begin{equation}
    \nabla^2 \Re f = (\partial_x^2 + \partial_y^2)\Re f = (\partial_x\partial_y-\partial_y\partial_x)\Im f = 0
\end{equation}
\end_inset


\end_layout

\begin_layout Standard
same for 
\begin_inset Formula $\Im f$
\end_inset

. Although 
\begin_inset Formula $f(z)$
\end_inset

 cannot have extrema, it can have a saddle point. According to the Cauchy theorem
\end_layout

\begin_layout Standard

\begin_inset Formula \begin{equation}
    \oint dz f(z) = 0
\end{equation}
\end_inset


\end_layout

\begin_layout Standard
for 
\begin_inset Formula $\forall f(z)$
\end_inset

 that is analytic. In another words, the integral does not depend on the exact path, but the end points. The motivation of MSD is to deform the contour 
\begin_inset Formula $C$
\end_inset

 to another contour 
\begin_inset Formula $C'$
\end_inset

, such that 
\begin_inset Formula $\Re f =u$
\end_inset

 has a saddle point along the new contour. 
\begin_inset Formula $\Re f$
\end_inset

 can be expanded near the saddle point. Furthermore, we can find the direction, in which the 
\begin_inset Formula $\Re f$
\end_inset

 decrease "steepest" while keeping 
\begin_inset Formula $\Im f$
\end_inset

 a constant. We assume the leading contributions of the integral 
\begin_inset CommandInset ref
LatexCommand eqref
reference "eq:I(g)"
plural "false"
caps "false"
noprefix "false"

\end_inset

 are from a small range along the steepest direction near the saddle point. Since the function decreases very fast (to zero), we can extend the integration limit from the small range to infinity without causing big problems. If we expand 
\begin_inset Formula $\Re f$
\end_inset

 up to the second order, we just need to do Gaussian integrals–which is familiar to all physicists. Now let's turn words into math.
\end_layout

\begin_layout Standard
Define
\end_layout

\begin_layout Standard

\begin_inset Formula \begin{equation}
    m(z,g) = -g f(z) = u(z,g) + iv(z,g) 
\end{equation}
\end_inset


\end_layout

\begin_layout Standard
We can expand the function near the saddle point 
\begin_inset Formula $z_0$
\end_inset


\end_layout

\begin_layout Standard

\begin_inset Formula \begin{equation}
    m(z,g) = m(z_0,g) + \frac{m''(z_0,g)}{2!}(z-z_0)^2 + ...
\end{equation}
\end_inset


\end_layout

\begin_layout Standard
Introduce compact notations 
\begin_inset Formula $m_0 = m(z_0,g)$
\end_inset

 and 
\begin_inset Formula $m''_0=m''(z_0,g)$
\end_inset

 and use polar representation
\end_layout

\begin_layout Standard

\begin_inset Formula \begin{equation}
    m''_0 = \abs{m''_0}e^{i\alpha} \quad z-z_0 = re^{i\theta}
\end{equation}
\end_inset


\end_layout

\begin_layout Standard

\begin_inset Formula \begin{align}
   m(z,g) & \simeq m(z_0,g) + \frac{1}{2} \abs{m''_0}r^2e^{i(\alpha+2\theta)} \nonumber \\
   & = m(z_0,g) + \frac{1}{2} \abs{m''_0}r^2 (\cos(\alpha+2\theta) + i\sin(\alpha+2\theta))
\end{align}
\end_inset


\end_layout

\begin_layout Standard
Up to the second order, the real part 
\begin_inset Formula $u(z,g)$
\end_inset

 increases fastest in the directions 
\begin_inset Formula $\alpha+2\theta = 2n\pi$
\end_inset

, i.e. 
\begin_inset Formula $\theta = -\alpha/2$
\end_inset

 and 
\begin_inset Formula $\theta = -\alpha/2 + \pi$
\end_inset

. While it decreases fastest in the directions 
\begin_inset Formula $\alpha+2\theta = (2n+1)\pi$
\end_inset

, i.e. 
\begin_inset Formula $\theta = -\alpha/2 + (\pi/2, 3\pi/2)$
\end_inset

. The constant lines (so called 
\series bold
level lines
\series default
) are along 
\begin_inset Formula $\theta = -\alpha/2 + (\pi/4,3\pi/4,5\pi/4,7\pi/4)$
\end_inset

. We can also find the above directions for 
\begin_inset Formula $v(z,g)$
\end_inset

. It's important to notice that the steepest descent directions of 
\begin_inset Formula $u(z,g)$
\end_inset

 are actually the level lines of 
\begin_inset Formula $v(z,g)$
\end_inset

. Deform the contour to 
\begin_inset Formula $C'$
\end_inset

 to let it pass the saddle point 
\begin_inset Formula $z_0$
\end_inset

, we can make following approximation
\end_layout

\begin_layout Standard

\begin_inset Formula \begin{align}
    I(g) & \simeq 2 e^{m_0 + i\theta} \int_0^a dr e^{-\abs{m_0''}r^2/2} \nonumber \\
         & = e^{m_0 + i\theta} \sqrt{\frac{2\pi}{\abs{m_0''}}} \nonumber \\
         & = e^{-gf(z_0)}\sqrt{\frac{2\pi}{g\abs{f^{(2)}}}}
\end{align}
\end_inset


\end_layout

\begin_layout Standard
where "2" comes from the two directions of the steepest descent, and 
\begin_inset Formula $[0,a]$
\end_inset

 is the small range in each of the two directions along the path. The phase integration (imaginary part) is pulled out of the integral by 
\begin_inset Formula $dz = e^{i\theta} dr$
\end_inset

 since the imaginary part is a constant. The higher order corrections of 
\begin_inset Formula $I(g)$
\end_inset

, multiple saddle points and generalization of the method to multi-dimensions can be found in Appendix E of 
\shape italic
Interacting Electrons and Quantum Magnetism, A. Auerbach, Springer (1998)
\shape default
.
\end_layout

\begin_layout Section
Elliptic integrals
\end_layout

\begin_layout Standard
For the most general theory of elliptic integrals, refer to 
\shape italic
Elliptic integrals
\shape default
 by Hancock, Harris. There are three types of elliptic integrals (indefinite integrals), called 
\shape italic
incomplete elliptic integral
\shape default
 of the first, second, and third kind, respectively.
\end_layout

\begin_layout Standard

\begin_inset Formula \begin{equation}
    I_1 = \int dx \frac{1}{s}
\end{equation}
\end_inset


\end_layout

\begin_layout Standard

\begin_inset Formula \begin{equation}
    I_2 = \int dx \frac{x^2}{s}
\end{equation}
\end_inset


\end_layout

\begin_layout Standard

\begin_inset Formula \begin{equation}
    I_3 = \int dx \frac{1}{s(x-b)}
\end{equation}
\end_inset


\end_layout

\begin_layout Standard
where 
\begin_inset Formula $s$
\end_inset

 is a polynomial of 
\begin_inset Formula $x$
\end_inset

 up to the fourth power.
\end_layout

\begin_layout Standard

\begin_inset Formula \begin{equation}
    s^2 = a_0x^4 + a_1 x^3 + a_2 x^2 + a_3 x + a_4
\end{equation}
\end_inset


\end_layout

\begin_layout Standard
The integral cannot be expressed as elementary functions, but the asymptotic behaviors of those functions are well studied. The definite elliptic integrals are called 
\shape italic
complete elliptic integral
\shape default
, which integrates over 
\begin_inset Formula $(0,1)$
\end_inset

 for the integration variable. For instance, the 
\shape italic
complete elliptic integral of the first kind
\shape default
 is defined as
\end_layout

\begin_layout Standard

\begin_inset Formula \begin{equation}
    K(m) = \int_0^1 \frac{1}{\sqrt{(1-t^2)(1-mt^2)}}
\end{equation}
\end_inset


\end_layout

\begin_layout Standard
The general reduction method to elliptic integral of the first kind will be reviewed. For simplicity, set 
\begin_inset Formula $a_0 = 1$
\end_inset


\end_layout

\begin_layout Standard

\begin_inset Formula \begin{equation}
    I_1 = \int dx \frac{1}{\sqrt{x^4 + a_1 x^3 + a_2 x^2 + a_3 x + a_4}} = \int dx \frac{1}{\sqrt{X}}
\end{equation}
\end_inset


\end_layout

\begin_layout Standard
where
\end_layout

\begin_layout Standard

\begin_inset Formula \begin{equation}
    X = \pm (x-\alpha)(x-\beta)(x-\gamma)(x-\delta)
\end{equation}
\end_inset


\end_layout

\begin_layout Standard
where 
\begin_inset Formula $\alpha > \beta > \gamma > \delta$
\end_inset

 are roots of the polynomial. For simplicity, assume all of them are real and different. Using the transform given by Legendre
\end_layout

\begin_layout Standard

\begin_inset Formula \begin{equation}
    x = \frac{p+qy}{1+y}
\end{equation}
\end_inset


\end_layout

\begin_layout Standard
the polynomial can be made to contain only even powers in 
\begin_inset Formula $y$
\end_inset

 with appropriate choice of 
\begin_inset Formula $p$
\end_inset

 and 
\begin_inset Formula $q$
\end_inset

. It's easy to find that
\end_layout

\begin_layout Standard

\begin_inset Formula \begin{equation}
    \int \frac{dx}{\sqrt{X}} = \int (q-p)\frac{dy}{\sqrt{\pm Y}}
\end{equation}
\end_inset


\end_layout

\begin_layout Standard
where
\end_layout

\begin_layout Standard

\begin_inset Formula \begin{equation}
    Y = (p - \alpha + (q - \alpha)y)(p - \beta + (q - \beta)y)(p - \gamma + (q - \gamma)y)(p - \delta + (q - \delta)y)
\end{equation}
\end_inset


\end_layout

\begin_layout Standard
The values of 
\begin_inset Formula $p$
\end_inset

 and 
\begin_inset Formula $q$
\end_inset

 are determined as
\end_layout

\begin_layout Standard

\begin_inset Formula \begin{equation}
    (p - \alpha)(q - \beta) + (p - \beta)(q - \alpha) = 0
\end{equation}
\end_inset


\end_layout

\begin_layout Standard
and
\end_layout

\begin_layout Standard

\begin_inset Formula \begin{equation}
    (p - \gamma)(q - \delta) + (p - \delta)(q - \gamma) = 0
\end{equation}
\end_inset


\end_layout

\begin_layout Standard
combine the two equations, which yields
\end_layout

\begin_layout Standard

\begin_inset Formula \begin{equation}
    \frac{p+q}{2} = \frac{\alpha\beta - \gamma\delta}{\alpha + \beta -\gamma - \delta}
    \label{eq:p+q}
\end{equation}
\end_inset


\end_layout

\begin_layout Standard

\begin_inset Formula \begin{equation}
    pq = \frac{\alpha\beta(\gamma+\delta) - \gamma\delta(\alpha+\beta)}{\alpha + \beta -\gamma - \delta}
\end{equation}
\end_inset


\end_layout

\begin_layout Standard

\begin_inset Formula \begin{equation}
    \bigg(\frac{q-p}{2}\bigg)^2 = \frac{(\alpha-\gamma)(\alpha-\delta)(\beta-\gamma)(\beta-\delta)}{(\alpha+\beta-\gamma-\delta)^2}
    \label{eq:q-p}
\end{equation}
\end_inset


\end_layout

\begin_layout Standard
The exception is 
\begin_inset Formula $\alpha + \beta = \gamma + \delta$
\end_inset

. In this situation, the transform should be
\end_layout

\begin_layout Standard

\begin_inset Formula \begin{equation}
    x = y + \frac{\alpha + \beta}{2} = y + \frac{\gamma + \delta}{2}
\end{equation}
\end_inset


\end_layout

\begin_layout Standard
The denominator can now be expressed as
\end_layout

\begin_layout Standard

\begin_inset Formula \begin{equation}
    Y = (\pm m^2  \pm n^2 y^2)(\pm r^2 \pm l^2 y^2)
\end{equation}
\end_inset


\end_layout

\begin_layout Standard
where
\end_layout

\begin_layout Standard

\begin_inset Formula \begin{equation}
   \pm m^2 = (p - \alpha)(p - \beta)
\end{equation}
\end_inset


\end_layout

\begin_layout Standard

\begin_inset Formula \begin{equation}
   \pm n^2 = (q - \alpha) (q - \beta)
\end{equation}
\end_inset


\end_layout

\begin_layout Standard

\begin_inset Formula \begin{equation}
    \pm r^2 = (p - \gamma)(p - \delta)
\end{equation}
\end_inset


\end_layout

\begin_layout Standard

\begin_inset Formula \begin{equation}
   \pm l^2 = (q - \gamma)(q - \delta)
\end{equation}
\end_inset


\end_layout

\begin_layout Standard
With further simplification
\end_layout

\begin_layout Standard

\begin_inset Formula \begin{equation}
    \int \frac{dx}{\sqrt{X}} = \int (q-p)\frac{dy}{\sqrt{\pm Y}} = \int \frac{dy}{f\sqrt{\pm(1\pm g^2y^2)(1\pm h^2y^2)}}
\end{equation}
\end_inset


\end_layout

\begin_layout Standard

\begin_inset Formula \begin{equation}
    \frac{1}{f} = \frac{q-p}{\pm m\cdot r}
\end{equation}
\end_inset


\end_layout

\begin_layout Standard

\begin_inset Formula \begin{equation}
    \pm g^2 = \frac{\pm n^2}{\pm m^2} 
\end{equation}
\end_inset


\end_layout

\begin_layout Standard

\begin_inset Formula \begin{equation}
    \pm h^2 = \frac{\pm l^2}{\pm r^2}
\end{equation}
\end_inset


\end_layout

\begin_layout Standard
suppose 
\begin_inset Formula $h > g$
\end_inset

, define 
\begin_inset Formula $hy = t$
\end_inset

.
\end_layout

\begin_layout Standard

\begin_inset Formula \begin{equation}
    \int \frac{dx}{\sqrt{X}} = \int \frac{dt}{fh\sqrt{\pm(1\pm t^2)(1\pm c^2t^2)}}
\end{equation}
\end_inset


\end_layout

\begin_layout Standard
where
\end_layout

\begin_layout Standard

\begin_inset Formula \begin{equation}
    c = \frac{g}{h} < 1
\end{equation}
\end_inset


\end_layout

\begin_layout Standard
There are 8 combinations of signs in the denominator. The reference (given in the beginning) summarized the right choice of further change of variable to reach the final result for each combination. For most physical system, all the roots are 
\series bold
real
\series default
. I will write down an easier transform without proof
\end_layout

\begin_layout Standard

\begin_inset Formula \begin{equation}
    x = \frac{\gamma(\beta-\delta) - \delta (\beta-\gamma)\sin^2\phi}{(\beta-\delta) - (\beta-\gamma)\sin^2\phi}
    \label{eq:realtransform}
\end{equation}
\end_inset


\end_layout

\begin_layout Standard
transforms
\end_layout

\begin_layout Standard

\begin_inset Formula \begin{equation}
    \frac{dx}{\sqrt{X}} \rightarrow \frac{2}{\sqrt{(\alpha-\gamma)(\beta-\delta)}}\frac{d\phi}{\sqrt{1-k^2\sin^2\phi}}
\end{equation}
\end_inset


\end_layout

\begin_layout Standard
where
\end_layout

\begin_layout Standard

\begin_inset Formula \begin{equation}
    k^2 = \frac{(\beta-\gamma)(\alpha-\delta)}{(\alpha-\gamma)(\beta-\delta)} \quad \gamma < x < \beta
\end{equation}
\end_inset


\end_layout

\begin_layout Subsection
Examples
\end_layout

\begin_layout Standard
The density of states of 2d tight-binding model on square lattice is given by
\end_layout

\begin_layout Standard

\begin_inset Formula \begin{equation}
    \rho(m) = \mathcal{N}^{-1}\sum_{\bm{k}}\delta(a-\gamma_{\bm{k}})
    \label{eq:startingintegral}
\end{equation}
\end_inset


\end_layout

\begin_layout Standard
where
\end_layout

\begin_layout Standard

\begin_inset Formula \begin{equation}
    \gamma_{\bm{k}} = \frac{1}{2}(\cos k_x + \cos k_y)
\end{equation}
\end_inset


\end_layout

\begin_layout Standard
Notice that 
\begin_inset CommandInset ref
LatexCommand eqref
reference "eq:startingintegral"
plural "false"
caps "false"
noprefix "false"

\end_inset

 can be written as
\end_layout

\begin_layout Standard

\begin_inset Formula \begin{align}
    \rho(m) & = \int_{-\pi}^{\pi}\int_{-\pi}^{\pi}\frac{d k_x}{2\pi}\frac{d k_y}{2\pi}\delta(a-\gamma_{\bm{k}}) \nonumber \\
                 & = 2\int_{0}^{\pi}\int_{0}^{\pi}\frac{d k_x}{2\pi}\frac{d k_y}{2\pi}\delta(a-\gamma_{\bm{k}}) \nonumber \\
                 & = \frac{1}{2\pi^2}\int_{-1}^{+1}\int_{-1}^{+1}dx dy \frac{\delta(a-\frac{1}{2}x - \frac{1}{2}y)}{\sqrt{(1-x^2)(1-y^2)}} \nonumber \\
                 & = \frac{1}{\pi^2}\int_{-1}^{+1}dx \frac{1}{\sqrt{(1-x)^2(1-(2a-x)^2})}
\end{align}
\end_inset


\end_layout

\begin_layout Standard
where I have used the fact that the integrand is an even function with respect to 
\begin_inset Formula $k_x$
\end_inset

 and 
\begin_inset Formula $k_y$
\end_inset

. I have also assumed that the unit length of square lattice constant. In the last step, I have used the properties of Dirac delta
\end_layout

\begin_layout Standard

\begin_inset Formula \begin{equation}
    \delta(f(x)) = \sum_{x_i} \frac{\delta(x-x_i)}{\abs{f'(x_i)}}
\end{equation}
\end_inset


\end_layout

\begin_layout Standard
where 
\begin_inset Formula $x_i$
\end_inset

 are the zeroes of 
\begin_inset Formula $f(x)$
\end_inset

. The integral can be reduced to an elliptic integral since the denominator is a square root of polynomial of 
\begin_inset Formula $x$
\end_inset

 up to the fourth power. For the function of density of states, we are interested in 
\begin_inset Formula $m\in(-1,1)$
\end_inset

, i.e. the first BZ. Let's first discuss the case 
\begin_inset Formula $0<m<1$
\end_inset

. The four roots are
\end_layout

\begin_layout Standard

\begin_inset Formula \begin{equation}
    \alpha = 2a+1 \quad \beta = 1 \quad \gamma = 2a-1 \quad \delta = -1
\end{equation}
\end_inset


\end_layout

\begin_layout Standard
Since all roots are real, we can directly use the transform 
\begin_inset CommandInset ref
LatexCommand eqref
reference "eq:realtransform"
plural "false"
caps "false"
noprefix "false"

\end_inset

. The integral now changes into
\end_layout

\begin_layout Standard

\begin_inset Formula \begin{align}
           & = \frac{1}{\pi^2} \int \frac{2}{\sqrt{(\alpha-\gamma)(\beta-\delta)}}\frac{d\phi}{\sqrt{1-k^2\sin^2\phi}} \nonumber \\
           & = \frac{1}{\pi^2} \int \frac{2}{\sqrt{2\cdot 2}}\frac{d\phi}{\sqrt{1-k^2\sin^2\phi}} \nonumber \\
\end{align}
\end_inset


\end_layout

\begin_layout Standard
where
\end_layout

\begin_layout Standard

\begin_inset Formula \begin{equation}
    k^2 = \frac{2(a-1)2(a+1)}{2\cdot 2} = 1-a^2
\end{equation}
\end_inset


\end_layout

\begin_layout Standard
the integration limit is 
\begin_inset Formula $2a-1 < x < 1$
\end_inset

. Consider the case 
\begin_inset Formula $-1<m<0$
\end_inset

,
\end_layout

\begin_layout Standard

\begin_inset Formula \begin{equation}
    \alpha = 1 \quad \beta = 2a+1  \quad \gamma = -1 \quad \delta = 2a-1 
\end{equation}
\end_inset


\end_layout

\begin_layout Standard
the transform gives the same result, the only difference is the integration limit, which reads 
\begin_inset Formula $-1<x<2a+1$
\end_inset

. Thus we find that the combined integration limit is the same as the original integration limit. We conclude that the integral now transforms into a complete integral of the first kind, with 
\begin_inset Formula $m = 1-a^2$
\end_inset

.
\end_layout

\begin_layout Standard

\begin_inset Formula \begin{equation}
   \rho(m) = 2 \cdot \frac{1}{\pi^2} K(1-a^2)
\end{equation}
\end_inset


\end_layout

\begin_layout Standard
where I have added the results of integration in the two region.
\end_layout

\begin_layout Section
Linked cluster theorem
\end_layout

\begin_layout Standard
The linked cluster theorem states that: the contribution to a correlation function 
\begin_inset Formula $C^{2m}(\bm{x}_1,...,\bm{x}_{2m})$
\end_inset

 at the 
\begin_inset Formula $l$
\end_inset

th order of perturbation theory is given by the sum of all diagrams, excluding the vacuum bubbles (without external legs). The 
\begin_inset Formula $\phi^4$
\end_inset

-theory can be used to prove this theorem without loss of generality, whose Lagrangian density is
\end_layout

\begin_layout Standard

\begin_inset Formula \begin{equation}
    \mathcal{L} = \frac{1}{2}\partial_{\mu}\phi\partial^{\mu}\phi + \frac{1}{2}m\phi^2 + \frac{\lambda}{4!}\phi^4 = \mathcal{L}_0 + \frac{\lambda}{4!}\phi^4
\end{equation}
\end_inset


\end_layout

\begin_layout Standard
We introduce 
\begin_inset Formula $\expval{X[\phi]}$
\end_inset

 as short-handed notation for 
\begin_inset Formula $C^{2m}(\bm{x}_1,...,\bm{x}_{2m})$
\end_inset

. The path integral formalism tells us that
\end_layout

\begin_layout Standard

\begin_inset Formula \begin{align}
    \expval{X[\phi]} & = \frac{\int \mathcal{D}\phi e^{iS}X[\phi]}{\int \mathcal{D}\phi e^{iS}} =  \frac{\int \mathcal{D}\phi e^{i\int d^4x \mathcal{L}_0 + \frac{\lambda}{4!}\phi^4}X[\phi]}{\int \mathcal{D}\phi e^{iS}}\nonumber \\
                     & = \frac{\int \mathcal{D}\phi e^{i\int d^4x\mathcal{L}_0}\bigg(1 + \frac{\lambda}{4!}\phi^4 + ...+ \frac{1}{n!}\bigg[\frac{\lambda}{4!}\phi^4 \bigg]^n...+\bigg)X[\phi]}{\int \mathcal{D}\phi e^{i\int d^4x\mathcal{L}_0}\bigg(1 + \frac{\lambda}{4!}\phi^4 + ...\bigg)} \nonumber \\
\end{align}
\end_inset


\end_layout

\begin_layout Standard
Both the denominator and numerator are Gaussian-like integrals. Consider the 
\begin_inset Formula $n$
\end_inset

th order expansion, according to 
\begin_inset CommandInset ref
LatexCommand eqref
reference "eq:gaussiancontraction"
plural "false"
caps "false"
noprefix "false"

\end_inset

, there are
\end_layout

\begin_layout Standard

\begin_inset Formula \[\begin{pmatrix} 2m + 4n \\ 2 \end{pmatrix}\]
\end_inset


\end_layout

\begin_layout Standard
ways of contraction. We can categorize the contractions into two types: 1. contraction involving external sources (i.e. fields in 
\begin_inset Formula $X[\phi]$
\end_inset

); 2. vacuum bubbles (i.e. fields from the action). The number of vacuum bubbles runs from 0 to 
\begin_inset Formula $n$
\end_inset

, thus the contribution from the 
\begin_inset Formula $n$
\end_inset

th order expansion is given by
\end_layout

\begin_layout Standard

\begin_inset Formula \begin{equation}
    \sum_{p=0}^n \frac{1}{n!}\begin{pmatrix} n \\ p \end{pmatrix}\expval{X[\phi]\bigg(\int d^4x \frac{\lambda}{4!}\phi^4\bigg)^{n-p}}_0^{\text{n.v.}}\expval{\bigg(\int d^4x \frac{\lambda}{4!}\phi^4\bigg)^{p}}_0
\end{equation}
\end_inset


\end_layout

\begin_layout Standard
where n.v. stands for non-vacuum contractions and "0" reminds us the expectation value is evaluated in the case of the free action (Gaussian). Sum over contributions from all orders of expansion, the numerator split intro vacuum and non-vacuum contributions,
\end_layout

\begin_layout Standard

\begin_inset Formula \begin{equation}
    \sum_{n=0}^{\infty}\sum_{p=0}^n\frac{1}{(n-p)!p!}\expval{X[\phi]\bigg(\int d^4x \frac{\lambda}{4!}\phi^4\bigg)^{n-p}}_0^{\text{n.v.}}\expval{\bigg(\int d^4x \frac{\lambda}{4!}\phi^4\bigg)^{p}}_0
\end{equation}
\end_inset


\end_layout

\begin_layout Standard
By straightforward rearrangement of the summations, we have
\end_layout

\begin_layout Standard

\begin_inset Formula \begin{equation}
     \sum_{n=0}^{\infty}\frac{1}{n!}\expval{X[\phi]\bigg(\int d^4x \frac{\lambda}{4!}\phi^4\bigg)^p}_0^{\text{n.v.}}\sum_{p=0}^{\infty}\frac{1}{p!}\expval{\bigg(\int d^4x \frac{\lambda}{4!}\phi^4\bigg)^{p}}_0
\end{equation}
\end_inset


\end_layout

\begin_layout Standard
where the second term cancels the denominator, which is the summation of all vacuum bubbles.
\end_layout

\end_body
\end_document
