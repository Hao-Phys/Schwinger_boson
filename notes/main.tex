%% LyX 2.3.0 created this file.  For more info, see http://www.lyx.org/.
%% Do not edit unless you really know what you are doing.
\documentclass{article}
\usepackage[utf8]{inputenc}
\usepackage{geometry}
\geometry{verbose,tmargin=2cm,bmargin=2cm,headheight=1cm,headsep=4cm,footskip=1cm}
\usepackage{verbatim}
\usepackage{mathtools}
\usepackage{bm}
\usepackage{amsmath}
\usepackage{amssymb}

\makeatletter

%%%%%%%%%%%%%%%%%%%%%%%%%%%%%% LyX specific LaTeX commands.
\DeclareTextSymbolDefault{\textquotedbl}{T1}

%%%%%%%%%%%%%%%%%%%%%%%%%%%%%% User specified LaTeX commands.
\usepackage{amsthm}\usepackage{graphicx}
\usepackage{physics}
\usepackage{bm}
\usepackage{appendix}
\usepackage{comment}
\usepackage{lscape}

\numberwithin{equation}{section}

\newcommand*{\rom}[1]{\expandafter\@slowromancap\romannumeral #1@}

\title{Dynamical structure factor: $J_1$-$J_2$ triangular lattice antiferromagnetic model}
\date{July 2018}
\author{Hao Zhang}

\makeatother

\begin{document}
\maketitle

\section{Introduction}

We consider the triangular lattice $S=1/2$ Heisenberg model in the
presence of nearest neighbor (NN) $J_{1}$ and next nearest neighbor
(NNN) $J_{2}$ antiferromagnetic couplings,

\begin{equation}
\mathcal{H}=J_{1}\sum_{\bm{i},\bm{\delta}_{1}>0}\bm{S}_{\bm{i}}\cdot\bm{S}_{\bm{i}+\bm{\delta}_{1}}+J_{2}\sum_{\bm{i},\bm{\delta}_{2}>0}\bm{S}_{\bm{i}}\cdot\bm{S}_{\bm{i}+\bm{\delta}_{2}}\label{eq:hham}
\end{equation}

Consider the classical limit, (i.e. $S\rightarrow\infty$), we can
treat the spin operators in \eqref{eq:hham} as classical vectors.
Introducing Fourier transform of classical spin vectors:

\begin{equation}
\bm{S}_{\bm{i}}=\frac{1}{\sqrt{\mathcal{N}}}\sum_{\bm{q}}\bm{S}_{\bm{q}}e^{i\bm{q}\cdot\bm{r}}
\end{equation}

where $\mathcal{N}$ is the number of sites of the system and $\bm{q}$
is a wave vector of the Brillouin zone (BZ). The ordering wave vector
$\bm{Q}$ can be found by minimizing the Hamiltonian \eqref{eq:hham}
with respect to $\bm{q}$. The classical analysis predicts that for
$J_{2}/J_{1}<1/8$, the system has the $2\pi/3$ 3-sublattice noncollinear
order, while for $1/8<J_{2}/J_{1}<1$, the ground state is characterized
by a 4-sublattice order of spin vectors which sum to zero around two
neighbouring elementary triangles. Spin wave analysis shows that the
leading $1/S$ quantum corrections lift the infinite classical degeneracy
for $1/8<J_{2}/J_{1}<1$ and favor a state with collinear order with
2-sublattice. \\

We will use the Schwinger boson method including Gaussian fluctuations
to calculate the dynamical structure factor of this model. To investigate
the possibility of observing magnon peaks, we introduce a small symmetry
breaking field in $x-y$ plane.

\begin{equation}
\bm{h}_{\bm{i}}=(h\cos(\bm{Q}\cdot\bm{r}_{\bm{i}}),h\sin(\bm{Q}\cdot\bm{r}_{\bm{i}}),0)
\end{equation}


\section{Path integral formalism}

\subsection{Spin representation}

We introduce the Schwinger boson representation of $\bm{S}_{\bm{i}}$:

\begin{align}
S_{\bm{i}}^{+} & =b_{\bm{i}\uparrow}^{\dagger}b_{\bm{i}\downarrow}\nonumber \\
S_{\bm{i}}^{-} & =b_{\bm{i}\downarrow}^{\dagger}b_{\bm{i}\uparrow}\nonumber \\
S_{\bm{i}}^{z} & =\frac{1}{2}(b_{\bm{i}\uparrow}^{\dagger}b_{\bm{i}\uparrow}-b_{\bm{i}\downarrow}^{\dagger}b_{\bm{i}\downarrow})
\end{align}

The \textbf{physical} subspace of Schwinger bosons must satisfy

\begin{equation}
\sum_{\sigma}b_{\bm{i}\sigma}^{\dagger}b_{\bm{i}\sigma}=2S=1\quad\sigma=\uparrow,\downarrow\label{eq:constraint}
\end{equation}

The Heisenberg interactions can be expressed in terms of six pairs
of \textit{bond operators}:

\begin{equation}
\bm{S}_{\bm{i}}\cdot\bm{S}_{\bm{i}+\bm{\delta}}=-A_{\bm{i},\bm{\delta}}^{\dagger}A_{\bm{i},\bm{\delta}}+:B_{\bm{i},\bm{\delta}}^{\dagger}B_{\bm{i},\bm{\delta}}:
\end{equation}

where $\bm{\delta}=\bm{\delta}_{1},\bm{\delta}_{2}$, and

\begin{align}
A_{\bm{i},\bm{\delta}} & =\frac{1}{2}(b_{\bm{i}\uparrow}b_{\bm{i}+\bm{\delta}\downarrow}-b_{\bm{i}\downarrow}b_{\bm{i}+\bm{\delta}\uparrow})\nonumber \\
B_{\bm{i},\bm{\delta}} & =\frac{1}{2}(b_{\bm{i}\uparrow}b_{\bm{i}+\bm{\delta}\uparrow}^{\dagger}+b_{\bm{i}\downarrow}b_{\bm{i}+\bm{\delta}\downarrow}^{\dagger})
\end{align}


\subsection{The generating Hamiltonian}

The generating Hamiltonian is defined by

\begin{equation}
\mathcal{H}[j,h]=\mathcal{H}-\sum_{\bm{i}}\sum_{\sigma\sigma'}j_{\bm{i}}^{\tau\mu}b_{\bm{i}\sigma}^{\tau\dagger}\sigma_{\sigma\sigma'}^{\mu}b_{\bm{i}\sigma'}^{\tau}+\mathcal{H}_{h}
\end{equation}

where $j_{\bm{i}}^{\tau\mu}$ represents the external field at space
$\bm{i}$ time $\tau$, and $\sigma^{\mu}$ is a Pauli matrix. $\mathcal{H}_{s}$
is the Zeeman coupling of the spin operators and the symmetry breaking
field,

\begin{align}
\mathcal{H}_{h} & =-\sum_{i}\bm{h}_{\bm{i}}\cdot\bm{\sigma}_{\bm{i}}\nonumber \\
 & =-h\sum_{\bm{i}}\cos(\bm{Q}\cdot\bm{r}_{\bm{i}})\sigma_{\bm{i}}^{x}+\sin(\bm{Q}\cdot\bm{r}_{\bm{i}})\sigma_{\bm{i}}^{y}\nonumber \\
 & =-h\sum_{\bm{i}}e^{-i\bm{Q}\cdot\bm{r}_{\bm{i}}}b_{{\bm{i}}\uparrow}^{\dagger}b_{{\bm{i}}\downarrow}+e^{i\bm{Q}\cdot\bm{r}_{\bm{i}}}b_{{\bm{i}}\downarrow}^{\dagger}b_{{\bm{i}}\uparrow}
\end{align}

Refer to \eqref{eq:constraint}, we find that the Schwinger boson
representation introduces an extra local $U(1)$ \textit{gauge} symmetry
in the model, which can be expressed as

\begin{equation}
b_{\bm{i}\sigma}^{\tau}\rightarrow b_{\bm{i}\sigma}^{\tau}e^{i\theta_{\bm{i}}^{\tau}}
\end{equation}

Thus we are free to perform local gauge transforms on boson operators,

\begin{align}
b_{\bm{i}\uparrow}^{\tau} & \rightarrow e^{-i\frac{\bm{Q\cdot\bm{r}_{i}}}{2}}b_{\bm{i}\uparrow}^{\tau}\nonumber \\
b_{\bm{i}\downarrow}^{\tau} & \rightarrow e^{i\frac{\bm{Q\cdot\bm{r}_{i}}}{2}}b_{\bm{i}\downarrow}^{\tau}\label{eq:gaugetrans}
\end{align}

which simplifies the Zeeman energy

\begin{equation}
\mathcal{H}_{h}\rightarrow-\frac{h}{2}\sum_{\bm{i}}b_{{\bm{i}}\uparrow}^{\dagger}b_{{\bm{i}}\downarrow}+b_{{\bm{i}}\downarrow}^{\dagger}b_{{\bm{i}}\uparrow}
\end{equation}


\subsection{The generating functional}

The generating functional of boson coherent states path integral is
given by

\begin{align}
Z[j,h] & =\Tr P_{S}T_{\tau}\bigg[\exp\bigg(-\int_{0}^{\beta}d\tau\mathcal{H}[j,h]\bigg)\bigg]\nonumber \\
 & =\lim_{\epsilon\rightarrow0}\Tr T_{\tau}\prod_{n=0}^{N_{\epsilon}-1}\big[P_{S}^{\tau_{n}}\exp(-\epsilon\mathcal{H}[j(\tau_{n}),h])\big]
\end{align}

where $\epsilon=\beta/N_{\epsilon}$ is the imaginary time step, and
$\tau_{n}=n\epsilon$ is the discrete imaginary time. $P_{S}$ is
the projector to enforce the constraint \eqref{eq:constraint}, which
can be written as

\begin{equation}
P_{S}^{\tau_{n}}=\lim_{\epsilon\rightarrow0}\prod_{\bm{i}}\epsilon\int\frac{d\lambda_{i}}{2\pi}\exp\bigg(-i\epsilon\sum_{\bm{i}}\lambda_{\bm{i}}^{\tau_{n}}(\sum_{\sigma}b_{\bm{i}\sigma}^{\tau_{n}\dagger}b_{\bm{i}\sigma}^{\tau_{n}}-2S)\bigg)
\end{equation}

Recall that the resolution of identity of boson coherent states is
given by

\begin{equation}
\mathbb{I}=\int d^{2}\bm{b}e^{-\bar{\bm{b}}\bm{b}}\ket{\bm{b}}\bra{\bm{b}}\label{eq:roi}
\end{equation}

where $\bm{b}=(b_{1\sigma},...,b_{\mathcal{N}\sigma})$ is a $2\mathcal{N}$-dimensional
($\sigma=\uparrow,\downarrow)$ vector living in the parameter space
of boson coherent states. The integration measure is defined as

\begin{equation}
\int d^{2}\bm{b}\equiv\int_{-\infty}^{\infty}\prod_{\bm{i}\sigma}\frac{d(\Re b_{\bm{i}\sigma})d(\Im b_{\bm{i}\sigma})}{\pi}=\int_{-\infty}^{\infty}\prod_{\bm{i}\sigma}\frac{\text{d}\bar{b}_{\bm{i}\sigma}\text{d}b_{\bm{i}\sigma}}{2\pi i}
\end{equation}

The trace of any operator can be calculated by using the boson coherent
states,

\begin{equation}
\Tr\mathcal{O}=\int d^{2}\bm{b}e^{-\bar{\bm{b}}\bm{b}}\bra{\bm{b}}\mathcal{O}\ket{\bm{b}}
\end{equation}

Thus the generating functional can be written as

\begin{align}
Z[j,h] & =T_{\tau}\int d^{2}\bm{b}^{0}e^{-\Bar{\bm{b}}^{0}\bm{b}^{0}}\nonumber \\
 & \bra{\bm{b}^{0}}\lim_{\epsilon\rightarrow0}\prod_{n=0}^{N_{\epsilon}-1}\prod_{i}\epsilon\int d\lambda_{i}\exp\bigg[-\epsilon\bigg(\mathcal{H}[j(\tau_{n}),h]+\sum_{i}\lambda_{i}^{\tau_{n}}(\sum_{\sigma}b_{\bm{i}\sigma}^{\tau_{n}\dagger}b_{\bm{i}\sigma}^{\tau_{n}}-1)\bigg)\bigg]\ket{\bm{b}^{0}}\nonumber \\
 & =T_{\tau}\int d^{2}\bm{b}^{0}e^{-\Bar{\bm{b}}^{0}\bm{b}^{0}}\nonumber \\
 & \bra{\bm{b}^{0}}\lim_{\epsilon\rightarrow0}\prod_{n=0}^{N_{\epsilon}-1}\prod_{i}\epsilon\int d\lambda_{i}\bigg[1-\epsilon\bigg(\mathcal{H}[j(\tau_{n}),h]+\sum_{i}\lambda_{i}^{\tau_{n}}(\sum_{\sigma}b_{\bm{i}\sigma}^{\tau_{n}\dagger}b_{\bm{i}\sigma}^{\tau_{n}}-1)\bigg)+\mathcal{O}(\epsilon)\bigg]\ket{\bm{b}^{0}}
\end{align}

We can insert $N_{\epsilon}-1$ resolution of identity \eqref{eq:roi},
and use periodic boundary condition

\begin{equation}
\bm{b}^{0}=\bm{b}^{\beta}
\end{equation}

Note that the time-ordering is taken care automatically by the process
of path integral,

\begin{align}
Z[j] & =\lim_{\epsilon\rightarrow0}\prod_{\tau}\int d^{2}\bm{b}^{\tau}e^{-\bar{\bm{b}}^{\tau}\bm{b}^{\tau}}\prod_{\tau=0}^{\beta-\epsilon}\bra{\bm{b}^{\tau+\epsilon}}\epsilon\prod_{i}\int d\lambda_{i}\nonumber \\
 & \bigg[1-\epsilon\bigg(\mathcal{H}[j(\tau),h]+\sum_{i}\lambda_{i}^{\tau}(\sum_{\sigma}b_{\bm{i}\sigma}^{\tau\dagger}b_{\bm{i}\sigma}^{\tau}-1)\bigg)+\mathcal{O}(\epsilon)\bigg]\ket{\bm{b}^{\tau}}\label{eq:generating1}
\end{align}

Note that boson coherent states are the eigenstates of boson annihilation
operators

\begin{align}
b_{i\sigma}^{\tau}\ket{\bm{b}} & =b_{i\sigma}^{\tau}\ket{\bm{b}}\nonumber \\
\bra{\bm{b}}b_{i\sigma}^{\tau\dagger} & =\bra{\bm{b}}\bar{b}_{i\sigma}^{\tau}
\end{align}

and the overlap of two coherent states is given by

\begin{equation}
\bra{\bm{b}}\ket{\bm{b}'}=e^{\bar{\bm{b}}\bm{b}'}
\end{equation}

Thus

\begin{align}
Z[j] & \simeq\lim_{\epsilon\rightarrow0}\prod_{\tau}\int d^{2}\bm{b}^{\tau}e^{-\bar{\bm{b}}^{\tau}\bm{b}^{\tau}}\prod_{\tau=0}^{\beta-\epsilon}\epsilon\prod_{\bm{i}}\int d\lambda_{\bm{i}}^{\tau}\nonumber \\
 & \bigg[\bra{\bm{b}^{\tau+\epsilon}}\ket{\bm{b}^{\tau}}-\epsilon\bigg(H[j(\tau),h]+\sum_{\bm{i}}\lambda_{\bm{i}}^{\tau}(\sum_{\sigma}\bar{b}_{\bm{i}\sigma}^{\tau}b_{\bm{i}\sigma}^{\tau}-1)\bigg)\bra{\bm{b}^{\tau}}\ket{\bm{b}^{\tau}}\bigg]\nonumber \\
 & =\lim_{\epsilon\rightarrow0}\prod_{\tau}\int d^{2}\bm{b}^{\tau}e^{-\bar{\bm{b}}^{\tau}\bm{b}^{\tau}}\prod_{\tau=0}^{\beta-\epsilon}\epsilon\prod_{\bm{i}}\int d\lambda_{\bm{i}}^{\tau}\nonumber \\
 & \bigg[e^{\bar{\bm{b}}^{\tau+\epsilon}(\bm{b}^{\tau+\epsilon}-\epsilon\partial_{\tau}\bm{b}^{\tau+\epsilon})}-\epsilon\bigg(H[j(\tau),h]+\sum_{\bm{i}}\lambda_{\bm{i}}^{\tau}(\sum_{\sigma}\bar{b}_{\bm{i}\sigma}^{\tau}b_{\bm{i}\sigma}^{\tau}-1)\bigg)e^{\bar{\bm{b}^{\tau}}\bm{b}^{\tau}}\bigg]\nonumber \\
 & =\int\mathcal{D}^{2}\bm{b}\int\mathcal{D}\lambda\exp\bigg(-\int_{0}^{\beta}d\tau\sum_{{\bm{i}}\sigma}\bar{b}_{{\bm{i}}\sigma}^{\tau}\partial_{\tau}b_{{\bm{i}}\sigma}^{\tau}+H[j(\tau),h]+i\sum_{\bm{i}}\lambda_{\bm{i}}^{\tau}(\sum_{\sigma}\bar{b}_{{\bm{i}}\sigma}^{\tau}b_{{\bm{i}}\sigma}^{\tau}-1)\bigg)\label{eq:partition}
\end{align}

where we have assumed the operators in $(...)$ of \eqref{eq:generating1}
are evaluated at equal time, which is of course a shaky assumption.
The time-ordering will be taken into account when we introduce the
Fourier transform of the boson operators. We have also introduced
short-handed notations for the integration measures.

\begin{align}
\int\mathcal{D}^{2}\bm{b} & =\lim_{\epsilon\rightarrow0}\prod_{\tau}\int d^{2}\bm{b}^{\tau}\nonumber \\
\int\mathcal{D}\lambda & =\lim_{\epsilon\rightarrow0}\prod_{\bm{i}\tau}\epsilon\int d^{2}\lambda_{\bm{i}}^{\tau}
\end{align}

The \textquotedbl classical generating Hamiltonian\textquotedbl{}
$H[j(\tau),h]$ can be written as

\begin{align}
H[j(\tau),h] & =J_{1}\sum_{\bm{i},\bm{\delta}_{1}>0}(-\bar{A}_{\bm{i},\bm{\delta}_{1}}^{\tau}A_{\bm{i},\bm{\delta}_{1}}^{\tau}+\bar{B}_{\bm{i},\bm{\delta}_{1}}^{\tau}B_{\bm{i},\bm{\delta}_{1}}^{\tau})\nonumber \\
 & +J_{2}\sum_{\bm{i},\bm{\delta}_{2}>0}(-\bar{A}_{\bm{i},\bm{\delta}_{2}}^{\tau}A_{\bm{i},\bm{\delta}_{2}}^{\tau}+\bar{B}_{\bm{i},\bm{\delta}_{2}}^{\tau}B_{\bm{i},\bm{\delta}_{2}}^{\tau})\nonumber \\
 & -\sum_{\bm{i}}\sum_{\sigma\sigma'}j_{\bm{i}}^{\tau\mu}\bar{b}_{\bm{i}\sigma}^{\tau}b_{\bm{i}\sigma'}^{\tau}\sigma_{\sigma\sigma'}^{\mu}+\mathcal{H}_{h}\label{eq:clham}
\end{align}

where

\begin{align}
A_{\bm{i},\bm{\delta}}^{\tau} & =\frac{1}{2}(b_{\bm{i}\uparrow}^{\tau}b_{\bm{i}+\bm{\delta}\downarrow}^{\tau}-b_{\bm{i}\downarrow}^{\tau}b_{\bm{i}+\bm{\delta}\uparrow}^{\tau})\nonumber \\
B_{\bm{i},\bm{\delta}}^{\tau} & =\frac{1}{2}(b_{\bm{i}\uparrow}^{\tau}\bar{b}_{\bm{i}+\bm{\delta}\uparrow}^{\tau}+b_{\bm{i}\downarrow}^{\tau}\bar{b}_{\bm{i}+\bm{\delta}\downarrow}^{\tau})
\end{align}


\subsection{Hubbard-Stratonovich transform}

We note that the classical Hamiltonian \eqref{eq:clham} is quartic
in boson variables. It can be decoupled into quadratic terms via Hubbard-Stratonovich
(HS) transform. We introduce pairs of auxiliary fields $(\bar{W}_{\bm{i},\bm{\delta}}^{r},W_{\bm{i},\bm{\delta}}^{r})$,
where $\bm{\delta}=\bm{\delta}_{1},\bm{\delta}_{2}$ and $r=A,B$.
The HS identities are

\begin{equation}
e^{-J_{\bm{\delta}}\bar{B}_{\bm{i}\bm{\delta}}^{\tau}B_{\bm{i},\bm{\delta}}^{\tau}}=J_{\bm{\delta}}\int\frac{d\bar{W}_{\bm{i},\bm{\delta}}^{(B)\tau}dW_{\bm{i},\bm{\delta}}^{(B)\tau}}{2\pi i}e^{-J_{\bm{\delta}}\bar{W}_{\bm{i},\bm{\delta}}^{(B)\tau}W_{\bm{i},\bm{\delta}}^{(B)\tau}+J_{\bm{\delta}}(-\bar{W}_{\bm{i},\bm{\delta}}^{(B)\tau}B_{\bm{i},\bm{\delta}}^{\tau}+W_{\bm{i},\bm{\delta}}^{(B)\tau}\bar{B}_{\bm{i},\bm{\delta}}^{\tau})}
\end{equation}

\begin{equation}
e^{J_{\bm{\delta}}\bar{A}_{\bm{i}\bm{\delta}}^{\tau}A_{\bm{i},\bm{\delta}}^{\tau}}=J_{\bm{\delta}}\int\frac{d\bar{W}_{\bm{i},\bm{\delta}}^{(A)\tau}dW_{\bm{i},\bm{\delta}}^{(A)\tau}}{2\pi i}e^{-J_{\bm{\delta}}\bar{W}_{\bm{i},\bm{\delta}}^{(A)\tau}W_{\bm{i},\bm{\delta}}^{(A)\tau}+J_{\bm{\delta}}(\bar{W}_{\bm{i},\bm{\delta}}^{(A)\tau}A_{\bm{i},\bm{\delta}}^{\tau}+W_{\bm{i},\bm{\delta}}^{(A)\tau}\bar{A}_{\bm{i},\bm{\delta}}^{\tau})}
\end{equation}

\begin{comment}
Let's check the second identity,

\begin{align}
 & \int d\Re Wd\Im We^{-J(\bar{W}W+\bar{W}A-W\bar{A})}\nonumber \\
 & =\int d\Re Wd\Im We^{-J[(\Re W-i\Im W)(\Re W+i\Im W)+(\Re W-i\Im W)A-(\Re W+i\Im W)\bar{A}]}\nonumber \\
 & =\int d\Re Wd\Im We^{(\Re W+\frac{1}{2}(A-\bar{A}))^{2}+(\Im W-\frac{i}{2}(A+\bar{A}))^{2}}e^{\frac{1}{4}J[(A+\bar{A})^{2}-(A-\bar{A})^{2}]}\nonumber \\
 & =\frac{\pi}{J}e^{J\bar{A}A}
\end{align}

We also note that

\begin{equation}
\int d\bar{W}dW=(2i)\int d\Re Wd\Im W
\end{equation}
\end{comment}


\section{The effective action}

The partition function \eqref{eq:partition} can now be written as

\begin{equation}
Z[j,h]=\int\mathcal{D}\bar{W}\mathcal{D}W\mathcal{D}\lambda e^{-\mathcal{S}_{\text{eff}}(\bar{W},W,\lambda,j,h)}\label{eq:pathintegral}
\end{equation}

where the integration measure is defined as

\begin{equation}
\mathcal{D}\bar{W}\mathcal{D}W=\prod_{\bm{i},\bm{\delta}>0}\prod_{\tau}\prod_{r}J_{\bm{\delta}}\frac{d\bar{W}_{\bm{i},\bm{\delta}}^{(r)\tau}dW_{\bm{i},\bm{\delta}}^{(r)\tau}}{2\pi i}
\end{equation}

The effective action can be split into two terms:

\begin{equation}
\mathcal{S}_{\text{eff}}(\bar{W},W,\lambda,j,h)=\mathcal{S}_{0}(\bar{W},W,\lambda)+\mathcal{S}_{\text{bos}}(\bar{W},W,\lambda,j,h)
\end{equation}

with

\begin{equation}
\mathcal{S}_{0}(\bar{W},W,\lambda)=\int_{0}^{\beta}d\tau\bigg(\sum_{\bm{i},\bm{\delta}>0}\sum_{r}J_{\bm{\delta}}\bar{W}_{\bm{i}\bm{\delta}}^{(r)\tau}W_{\bm{i}\bm{\delta}}^{(r)\tau}-i2S\sum_{\bm{i}}\lambda_{\bm{i}}^{\tau}\bigg)
\end{equation}

The bosonic action is defined as the integration over bosonic variables,

\begin{align}
e^{-\mathcal{S}_{\text{bos}}} & =\int\mathcal{D}^{2}\bm{b}\exp\bigg(-\int_{0}^{\beta}d\tau\sum_{\bm{i}\sigma}\bar{b}_{\bm{i}\sigma}^{\tau}\partial_{\tau}b_{\bm{i}\sigma}+i\sum_{\bm{i}}\lambda_{\bm{i}}^{\tau}\sum_{\sigma}\bar{b}_{\bm{i}\sigma}^{\tau}b_{\bm{i}\sigma}^{\tau}\nonumber \\
 & +\sum_{\bm{i},\bm{\delta}>0}J_{\bm{\delta}}\big(\bar{W}_{\bm{i},\bm{\delta}}^{(B)\tau}B_{\bm{i},\bm{\delta}}^{\tau}-W_{\bm{i},\bm{\delta}}^{(B)\tau}\bar{B}_{\bm{i},\bm{\delta}}^{\tau}-\bar{W}_{\bm{i},\bm{\delta}}^{(A)\tau}A_{\bm{i},\bm{\delta}}^{\tau}-W_{\bm{i},\bm{\delta}}^{(A)\tau}\bar{A}_{\bm{i},\bm{\delta}}^{\tau}\big)\nonumber \\
 & -\sum_{\bm{i}}\sum_{\sigma\sigma'}j_{\bm{i}}^{\tau\mu}\bar{b}_{\bm{i}\sigma}^{\tau}b_{\bm{i}\sigma'}^{\tau}\sigma_{\sigma\sigma'}^{\mu}+\mathcal{H}_{h}\bigg)\nonumber \\
 & =\int\mathcal{D}^{2}\bm{b}e^{-\bm{b}^{\dagger}\mathcal{G}^{-1}(\bar{W},W,\lambda,j,h)\bm{b}}=\det[\mathcal{G}(\bar{W},W,\lambda,j,h)]\label{eq:defofaction1}
\end{align}

where in the last two steps, we have used integration by parts to
define the \textbf{dynamical matrix}. The dimension of $\mathcal{G}$
is $[(\beta/\epsilon)\mathcal{N}N]\times[(\beta/\epsilon)\mathcal{N}N]$
($N$ is the group index of $SU(N)$, in this calculation $N=2$).

\subsection{Momentum-Matsubara representation}

To calculate the dynamical structure factor, we should work in momentum
and Matsubara space. Introduce following Fourier transform

\begin{align}
b_{\bm{i}\sigma}^{\tau} & =\frac{1}{\sqrt{\mathcal{N}\beta}}\sum_{\bm{k},n}b_{\bm{k}\sigma}^{\omega_{n}}e^{-i(\bm{k}\cdot\bm{r}_{i}-\omega_{n}\tau)}\\
\bar{b}_{\bm{i}\sigma}^{\tau} & =\frac{1}{\sqrt{\mathcal{N}\beta}}\sum_{\bm{k},n}\bar{b}_{\bm{k}\sigma}^{\omega_{n}}e^{i(\bm{k}\cdot\bm{r}_{i}-\omega_{n}\tau)}e^{-i\omega_{n}0^{+}}\label{eq:Fourier}
\end{align}

where the convergence factor $e^{i\omega_{n}0^{+}}$ is introduced
to account for the time order of boson coherent states as discussed
in the paragraph after \eqref{eq:partition}. We can also introduce
Fourier transform on the auxiliary fields,

\begin{align}
W_{\bm{i},\bm{\delta}}^{(r)\tau} & =\frac{1}{\sqrt{\mathcal{N}\beta}}\sum_{\bm{k}}\sum_{n}W_{\bm{k},\bm{\delta}}^{(r)\omega_{n}}e^{-i(\bm{k}\cdot\bm{x}_{i}-\omega_{n}\tau)}\nonumber \\
\bar{W}_{\bm{i},\bm{\delta}}^{(r)\tau} & =\frac{1}{\sqrt{\mathcal{N}\beta}}\sum_{\bm{k}}\sum_{n}\bar{W}_{\bm{k},\bm{\delta}}^{(r)\omega_{n}}e^{i(\bm{k}\cdot\bm{x}_{i}-\omega_{n}\tau)}\nonumber \\
\lambda_{\bm{i}}^{\tau} & =\frac{1}{\sqrt{\mathcal{N}\beta}}\sum_{\bm{k}}\sum_{n}\lambda_{\bm{k}}^{\omega_{n}}e^{-i(\bm{k}\cdot\bm{x}_{i}-\omega_{n}\tau)}
\end{align}

As discussed in \eqref{eq:gaugetrans}, we will perform a local gauge
transform on the boson operators to simplify the Zeeman energy. In
momentum space, this is equivalent to

\begin{align}
b_{\bm{k}\uparrow}^{\omega} & \rightarrow b_{\bm{k}+\frac{\bm{Q}}{2}\uparrow}^{\omega}\nonumber \\
b_{\bm{k}\downarrow}^{\omega} & \rightarrow b_{\bm{k}-\frac{\bm{Q}}{2}\downarrow}^{\omega}
\end{align}

We will work in the representation $\bm{b}_{\bm{k},\omega}^{\dagger}=(\bar{b}_{\bm{k}\uparrow}^{\omega},b_{-\bm{k}\downarrow}^{-\omega},\bar{b}_{\bm{k}\downarrow}^{\omega},b_{-\bm{k}\uparrow}^{-\omega})$.
We can work out the result term by term,

\begin{align}
 & \int_{0}^{\beta}d\tau\sum_{\bm{i}\sigma}\bar{b}_{\bm{i}\sigma}^{\tau}\partial_{\tau}b_{\bm{i}\sigma}\nonumber \\
 & =\frac{1}{\mathcal{N}\beta}\int_{0}^{\beta}d\tau\sum_{\bm{i}\sigma}\sum_{\bm{k}\bm{k}'}\sum_{\omega\omega'}(i\omega')\bar{b}_{\bm{k}\sigma}^{\omega}b_{\bm{k}'\sigma}^{\omega'}e^{i(\bm{k}-\bm{k}')\cdot\bm{r}_{i}}e^{-i(\omega-\omega')\tau}e^{-i\omega0^{+}}\nonumber \\
 & =\sum_{\bm{k}\bm{k}'}\sum_{\omega\omega'}(i\omega)\bar{b}_{\bm{k}\sigma}^{\omega}b_{\bm{k}'\sigma}^{\omega'}\delta_{\bm{k},\bm{k}'}\delta_{\omega,\omega'}e^{-i\omega0^{+}}\nonumber \\
 & =\frac{1}{2}\big(\sum_{\bm{k}\bm{k}'}\sum_{\omega\omega'}(i\omega)\bar{b}_{\bm{k}\sigma}^{\omega}b_{\bm{k}'\sigma}^{\omega'}\delta_{\bm{k},\bm{k}'}\delta_{\omega,\omega'}e^{-i\omega0^{+}}+\sum_{\bm{k}\bm{k}'}\sum_{\omega\omega'}(-i\omega)b_{-\bm{k}'\sigma}^{-\omega'}\bar{b}_{-\bm{k}\sigma}^{-\omega}\delta_{\bm{k},\bm{k}'}\delta_{\omega,\omega'}e^{i\omega0^{+}}\big)
\end{align}

The calculations for the $\lambda$-field are similar, which are omitted
here. The coupling with $W^{(B)}$-fields contributes to the diagonal
elements of the dynamical matrix.

\begin{align}
 & -\int_{0}^{\beta}d\tau\sum_{\bm{i}}\sum_{\bm{\delta}>0}\frac{J_{\bm{\delta}}}{2}W_{\bm{i},\bm{\delta}}^{(B)\tau}(b_{\bm{i}+\bm{\delta}\uparrow}^{\tau}\bar{b}_{\bm{i}\uparrow}^{\tau}+b_{\bm{i}+\bm{\delta}\downarrow}^{\tau}\bar{b}_{\bm{i}\downarrow}^{\tau})\nonumber \\
 & \rightarrow-\int_{0}^{\beta}d\tau\sum_{\bm{i}}\sum_{\bm{\delta}>0}\frac{J_{\bm{\delta}}}{2}W_{\bm{i},\bm{\delta}}^{(B)\tau}(b_{\bm{i}+\bm{\delta}\uparrow}^{\tau}\bar{b}_{\bm{i}\uparrow}^{\tau}e^{-i\frac{\bm{Q}\cdot\bm{\delta}}{2}}+b_{\bm{i}+\bm{\delta}\downarrow}^{\tau}\bar{b}_{\bm{i}\downarrow}^{\tau}e^{i\frac{\bm{Q}\cdot\bm{\delta}}{2}})\nonumber \\
 & =-\frac{1}{(\mathcal{N}\beta)^{3/2}}\int_{0}^{\beta}d\tau\sum_{\bm{i}}\sum_{\bm{\delta}>0}\frac{J_{\bm{\delta}}}{2}\sum_{\bm{q}\bm{k}\bm{k}'}\sum_{\nu\omega\omega'}W_{\bm{q},\bm{\delta}}^{(B)\nu}(\bar{b}_{\bm{k}\uparrow}^{\omega}b_{\bm{k}'\uparrow}^{\omega'}e^{i(\bm{k}-\bm{k}'-\bm{q})\cdot\bm{r}_{i}}e^{-i(\omega-\omega'-\nu)\tau}e^{-i(\bm{k}'+\frac{\bm{Q}}{2})\cdot\bm{\delta}}e^{-i\omega0^{+}}\nonumber \\
 & +\bar{b}_{\bm{k}\downarrow}^{\omega}b_{\bm{k}'\downarrow}^{\omega'}e^{i(\bm{k}-\bm{k}'-\bm{q})\cdot\bm{r}_{i}}e^{-i(\omega-\omega'-\nu)\tau}e^{-i(\bm{k}'-\frac{\bm{Q}}{2})\cdot\bm{\delta}}e^{-i\omega0^{+}})\nonumber \\
 & =-\frac{1}{\sqrt{\mathcal{N}\beta}}\sum_{\bm{\delta}>0}\frac{J_{\bm{\delta}}}{2}\sum_{\bm{k}\bm{k}'}\sum_{\omega\omega'}W_{\bm{k}-\bm{k}',\bm{\delta}}^{(B)\omega-\omega'}(\bar{b}_{\bm{k}\uparrow}^{\omega}b_{\bm{k}'\uparrow}^{\omega'}e^{-i(\bm{k}'+\frac{\bm{Q}}{2})\cdot\bm{\delta}}e^{-i\omega0^{+}}+\bar{b}_{\bm{k}\downarrow}^{\omega}b_{\bm{k}'\downarrow}^{\omega'}e^{-i(\bm{k}'-\frac{\bm{Q}}{2})\cdot\bm{\delta}}e^{-i\omega0^{+}})\nonumber \\
 & =-\frac{1}{\sqrt{\mathcal{N}\beta}}\sum_{\bm{\delta}>0}\frac{J_{\bm{\delta}}}{2}\sum_{\bm{k}\bm{k}'}\sum_{\omega\omega'}W_{\bm{k}-\bm{k}',\bm{\delta}}^{(B)\omega-\omega'}(b_{-\bm{k}\uparrow}^{-\omega}\bar{b}_{-\bm{k}'\uparrow}^{-\omega'}e^{i(\bm{k}-\frac{\bm{Q}}{2})\cdot\bm{\delta}}e^{i\omega'0^{+}}+b_{-\bm{k}\downarrow}^{-\omega}\bar{b}_{-\bm{k}'\downarrow}^{-\omega'}e^{i(\bm{k}+\frac{\bm{Q}}{2})\cdot\bm{\delta}}e^{i\omega'0^{+}})
\end{align}

\begin{align}
 & \int_{0}^{\beta}d\tau\sum_{\bm{i}}\sum_{\bm{\delta}>0}\frac{J_{\bm{\delta}}}{2}\bar{W}_{\bm{i},\bm{\delta}}^{(B)\tau}(b_{\bm{i}\uparrow}^{\tau}\bar{b}_{\bm{i}+\bm{\delta}\uparrow}^{\tau}+b_{\bm{i}\downarrow}^{\tau}\bar{b}_{\bm{i}+\bm{\delta}\downarrow}^{\tau})\nonumber \\
 & \rightarrow\int_{0}^{\beta}d\tau\sum_{\bm{i}}\sum_{\bm{\delta}>0}\frac{J_{\bm{\delta}}}{2}\bar{W}_{\bm{i},\bm{\delta}}^{(B)\tau}(b_{\bm{i}\uparrow}^{\tau}\bar{b}_{\bm{i}+\bm{\delta}\uparrow}^{\tau}e^{i\frac{\bm{Q}\cdot\bm{\delta}}{2}}+b_{\bm{i}\downarrow}^{\tau}\bar{b}_{\bm{i}+\bm{\delta}\downarrow}^{\tau}e^{-i\frac{\bm{Q}\cdot\bm{\delta}}{2}})\nonumber \\
 & =\frac{1}{(\mathcal{N}\beta)^{3/2}}\int_{0}^{\beta}d\tau\sum_{\bm{i}}\sum_{\bm{\delta}>0}\frac{J_{\bm{\delta}}}{2}\sum_{\bm{q}\bm{k}\bm{k}'}\sum_{\nu\omega\omega'}\bar{W}_{\bm{q},\bm{\delta}}^{(B)\nu}(\bar{b}_{\bm{k}\uparrow}^{\omega}b_{\bm{k}'\uparrow}^{\omega'}e^{i(\bm{k}-\bm{k}'+\bm{q})\cdot\bm{r}_{i}}e^{-i(\omega-\omega'+\nu)\tau}e^{i(\bm{k}+\frac{\bm{Q}}{2})\cdot\bm{\delta}}e^{-i\omega0^{+}}\nonumber \\
 & +\bar{b}_{\bm{k}\downarrow}^{\omega}b_{\bm{k}'\downarrow}^{\omega'}e^{i(\bm{k}-\bm{k}'+\bm{q})\cdot\bm{r}_{i}}e^{-i(\omega-\omega'+\nu)\tau}e^{i(\bm{k}-\frac{\bm{Q}}{2})\cdot\bm{\delta}}e^{-i\omega0^{+}})\nonumber \\
 & =\frac{1}{\sqrt{\mathcal{N}\beta}}\sum_{\bm{\delta}>0}\frac{J_{\bm{\delta}}}{2}\sum_{\bm{k}\bm{k}'}\sum_{\omega\omega'}\bar{W}_{\bm{k}'-\bm{k},\bm{\delta}}^{(B)\omega'-\omega}(\bar{b}_{\bm{k}\uparrow}^{\omega}b_{\bm{k}'\uparrow}^{\omega'}e^{i(\bm{k}+\frac{\bm{Q}}{2})\cdot\bm{\delta}}e^{-i\omega0^{+}}+\bar{b}_{\bm{k}\downarrow}^{\omega}b_{\bm{k}'\downarrow}^{\omega'}e^{i(\bm{k}-\frac{\bm{Q}}{2})\cdot\bm{\delta}}e^{-i\omega0^{+}})\nonumber \\
 & =\frac{1}{\sqrt{\mathcal{N}\beta}}\sum_{\bm{\delta}>0}\frac{J_{\bm{\delta}}}{2}\sum_{\bm{k}\bm{k}'}\sum_{\omega\omega'}\bar{W}_{\bm{k}'-\bm{k},\bm{\delta}}^{(B)\omega'-\omega}(b_{-\bm{k}\uparrow}^{-\omega}\bar{b}_{-\bm{k}'\uparrow}^{-\omega'}e^{-i(\bm{k}'-\frac{\bm{Q}}{2})\cdot\bm{\delta}}e^{i\omega'0^{+}}+b_{-\bm{k}\downarrow}^{-\omega}\bar{b}_{-\bm{k}'\downarrow}^{-\omega'}e^{-i(\bm{k}'+\frac{\bm{Q}}{2})\cdot\bm{\delta}}e^{i\omega'0^{+}})
\end{align}

The coupling with the $W^{(A)}$-fields connects two boson variables
with opposite spin, momentum, and frequency.

\begin{align}
 & -\int_{0}^{\beta}d\tau\sum_{\bm{i}}\sum_{\bm{\delta}>0}\frac{J_{\bm{\delta}}}{2}\bar{W}_{\bm{i},\bm{\delta}}^{(A)\tau}(b_{\bm{i}\uparrow}^{\tau}b_{\bm{i}+\bm{\delta}\downarrow}^{\tau}-b_{\bm{i}\downarrow}^{\tau}b_{\bm{i}+\bm{\delta}\uparrow}^{\tau})\nonumber \\
 & -\rightarrow\int_{0}^{\beta}d\tau\sum_{\bm{i}}\sum_{\bm{\delta}>0}\frac{J_{\bm{\delta}}}{2}\bar{W}_{\bm{i},\bm{\delta}}^{(A)\tau}(b_{\bm{i}\uparrow}^{\tau}b_{\bm{i}+\bm{\delta}\downarrow}^{\tau}e^{i\frac{\bm{Q}\cdot\bm{\delta}}{2}}-b_{\bm{i}\downarrow}^{\tau}b_{\bm{i}+\bm{\delta}\uparrow}^{\tau}e^{-i\frac{\bm{Q}\cdot\bm{\delta}}{2}})\nonumber \\
 & =-\frac{1}{(\mathcal{N}\beta)^{3/2}}\int_{0}^{\beta}d\tau\sum_{\bm{i}}\sum_{\bm{\delta}>0}\frac{J_{\bm{\delta}}}{2}\sum_{\bm{q}\bm{k}\bm{k}'}\sum_{\nu\omega\omega'}\bar{W}_{\bm{q},\bm{\delta}}^{(A)\nu}(b_{-\bm{k}\uparrow}^{-\omega}b_{\bm{k}'\downarrow}^{\omega'}e^{-i(-\bm{k}+\bm{k}'-\bm{q})\cdot\bm{r}_{i}}e^{-i(\omega-\omega'+\nu)\tau}e^{-i(\bm{k}'-\frac{\bm{Q}}{2})\cdot\bm{\delta}}\nonumber \\
 & -b_{-\bm{k}\downarrow}^{-\omega}b_{\bm{k}'\uparrow}^{\omega'}e^{-i(-\bm{k}+\bm{k}'-\bm{q})\cdot\bm{r}_{i}}e^{-i(\omega-\omega'+\nu)\tau}e^{-i(\bm{k}'+\frac{\bm{Q}}{2})\cdot\bm{\delta}})\nonumber \\
 & =-\frac{1}{\sqrt{\mathcal{N}\beta}}\sum_{\bm{\delta}>0}\frac{J_{\bm{\delta}}}{2}\sum_{\bm{k}\bm{k}'}\sum_{\omega\omega'}\bar{W}_{\bm{k}'-\bm{k},\bm{\delta}}^{(A)\omega'-\omega}(b_{-\bm{k}\uparrow}^{-\omega}b_{\bm{k}'\downarrow}^{\omega'}e^{-i(\bm{k}'-\frac{\bm{Q}}{2})\cdot\bm{\delta}}-b_{-\bm{k}\downarrow}^{-\omega}b_{\bm{k}'\uparrow}^{\omega'}e^{-i(\bm{k}'+\frac{\bm{Q}}{2})\cdot\bm{\delta}})\nonumber \\
 & =-\frac{1}{\sqrt{\mathcal{N}\beta}}\sum_{\bm{\delta}>0}\frac{J_{\bm{\delta}}}{2}\sum_{\bm{k}\bm{k}'}\sum_{\omega\omega'}\bar{W}_{\bm{k}'-\bm{k},\bm{\delta}}^{(A)\omega'-\omega}(b_{-\bm{k}\downarrow}^{-\omega}b_{\bm{k}'\uparrow}^{\omega'}e^{i(\bm{k}+\frac{\bm{Q}}{2})\cdot\bm{\delta}}-b_{-\bm{k}\uparrow}^{-\omega}b_{\bm{k}'\downarrow}^{\omega'}e^{i(\bm{k}-\frac{\bm{Q}}{2})\cdot\bm{\delta}})
\end{align}

\begin{align}
 & -\int_{0}^{\beta}d\tau\sum_{\bm{i}}\sum_{\bm{\delta}>0}\frac{J_{\bm{\delta}}}{2}W_{\bm{i},\bm{\delta}}^{(A)\tau}(\bar{b}_{\bm{i}+\bm{\delta}\downarrow}^{\tau}\bar{b}_{\bm{i}\uparrow}^{\tau}-\bar{b}_{\bm{i}+\bm{\delta}\uparrow}^{\tau}\bar{b}_{\bm{i}\downarrow}^{\tau})\nonumber \\
 & \rightarrow-\int_{0}^{\beta}d\tau\sum_{\bm{i}}\sum_{\bm{\delta}>0}\frac{J_{\bm{\delta}}}{2}W_{\bm{i},\bm{\delta}}^{(A)\tau}(\bar{b}_{\bm{i}+\bm{\delta}\downarrow}^{\tau}\bar{b}_{\bm{i}\uparrow}^{\tau}e^{-i\frac{\bm{Q}\cdot\bm{\delta}}{2}}-\bar{b}_{\bm{i}+\bm{\delta}\uparrow}^{\tau}\bar{b}_{\bm{i}\downarrow}^{\tau}e^{i\frac{\bm{Q}\cdot\bm{\delta}}{2}})\nonumber \\
 & =-\frac{1}{(\mathcal{N}\beta)^{3/2}}\int_{0}^{\beta}d\tau\sum_{\bm{i}}\sum_{\bm{\delta}>0}\frac{J_{\bm{\delta}}}{2}\sum_{\bm{q}\bm{k}\bm{k}'}\sum_{\nu\omega\omega'}W_{\bm{q},\bm{\delta}}^{(A)\nu}(\bar{b}_{\bm{k}\downarrow}\bar{b}_{-\bm{k}'\uparrow}^{-\omega'}e^{i(\bm{k}-\bm{k}'-\bm{q})\cdot\bm{r}_{i}}e^{-i(\omega-\omega'-\nu)\tau}e^{i(\bm{k}-\frac{\bm{Q}}{2})\cdot\bm{\delta}}e^{-i(\omega-\omega')0^{+}}\nonumber \\
 & -\bar{b}_{\bm{k}\uparrow}\bar{b}_{-\bm{k}'\downarrow}^{-\omega'}e^{i(\bm{k}-\bm{k}'-\bm{q})\cdot\bm{r}_{i}}e^{-i(\omega-\omega'-\nu)\tau}e^{i(\bm{k}+\frac{\bm{Q}}{2})\cdot\bm{\delta}}e^{-i(\omega-\omega')0^{+}})\nonumber \\
 & =-\frac{1}{\sqrt{\mathcal{N}\beta}}\sum_{\bm{\delta}>0}\frac{J_{\bm{\delta}}}{2}\sum_{\bm{k}\bm{k}'}\sum_{\omega\omega'}W_{\bm{k}-\bm{k}',\bm{\delta}}^{(A)\omega-\omega'}(\bar{b}_{\bm{k}\downarrow}^{\omega}\bar{b}_{-\bm{k}'\uparrow}^{-\omega'}e^{i(\bm{k}-\frac{\bm{Q}}{2})\cdot\bm{\delta}}e^{-i(\omega-\omega')0^{+}}-\bar{b}_{\bm{k}\uparrow}^{\omega}\bar{b}_{-\bm{k}'\downarrow}^{-\omega'}e^{i(\bm{k}+\frac{\bm{Q}}{2})\cdot\bm{\delta}}e^{-i(\omega-\omega')0^{+}})\nonumber \\
 & =-\frac{1}{\sqrt{\mathcal{N}\beta}}\sum_{\bm{\delta}>0}\frac{J_{\bm{\delta}}}{2}\sum_{\bm{k}\bm{k}'}\sum_{\omega\omega'}W_{\bm{k}-\bm{k}',\bm{\delta}}^{(A)\omega-\omega'}(\bar{b}_{\bm{k}\uparrow}^{\omega}\bar{b}_{-\bm{k}'\downarrow}^{-\omega'}e^{-i(\bm{k}'+\frac{\bm{Q}}{2})\cdot\bm{\delta}}e^{-i(\omega-\omega')0^{+}}-\bar{b}_{\bm{k}\downarrow}^{\omega}\bar{b}_{-\bm{k}'\uparrow}^{-\omega'}e^{-i(\bm{k}'-\frac{\bm{Q}}{2})\cdot\bm{\delta}}e^{-i(\omega-\omega')0^{+}})
\end{align}

The elements of the dynamical matrix $\mathcal{G}^{-1}[j=0]=\mathcal{M}$
are listed below

\begin{align}
\mathcal{M}_{11} & =\bigg(\frac{1}{2}i\omega\delta_{\bm{k},\bm{k}'}\delta_{\omega,\omega'}+\frac{i}{2}\frac{\lambda_{\bm{k}-\bm{k}'}^{\omega-\omega'}}{\sqrt{\mathcal{N}\beta}}\nonumber \\
 & -\sum_{\bm{\delta}=\bm{\delta}_{1},\bm{\delta_{2}}}\frac{J_{\bm{\delta}}}{4\sqrt{\mathcal{N}\beta}}(W_{\bm{k}-\bm{k}'}^{(B)\omega-\omega'}e^{-i(\bm{k}'+\frac{\bm{Q}}{2})\cdot\bm{\delta}}-\bar{W}_{\bm{k}'-\bm{k}}^{(B)\omega'-\omega}e^{i(\bm{k}+\frac{\bm{Q}}{2})\cdot\bm{\delta}})\bigg)e^{-i\omega0^{+}}\\
\mathcal{M}_{22} & =\bigg(\frac{1}{2}(-i\omega)\delta_{\bm{k},\bm{k}'}\delta_{\omega,\omega'}+\frac{i}{2}\frac{\lambda_{\bm{k}'-\bm{k}}^{\omega'-\omega}}{\sqrt{\mathcal{N}\beta}}\nonumber \\
 & -\sum_{\bm{\delta}=\bm{\delta}_{1},\bm{\delta_{2}}}\frac{J_{\bm{\delta}}}{4\sqrt{\mathcal{N}\beta}}(W_{\bm{k}-\bm{k}'}^{(B)\omega-\omega'}e^{i(\bm{k}+\frac{\bm{Q}}{2})\cdot\bm{\delta}}-\bar{W}_{\bm{k}'-\bm{k}}^{(B)\omega'-\omega}e^{-i(\bm{k}'+\frac{\bm{Q}}{2})\cdot\bm{\delta}})\bigg)e^{i\omega'0^{+}}\\
\mathcal{M}_{33} & =\bigg(\frac{1}{2}i\omega\delta_{\bm{k},\bm{k}'}\delta_{\omega,\omega'}+\frac{i}{2}\frac{\lambda_{\bm{k}-\bm{k}'}^{\omega-\omega'}}{\sqrt{\mathcal{N}\beta}}\nonumber \\
 & -\sum_{\bm{\delta}=\bm{\delta}_{1},\bm{\delta_{2}}}\frac{J_{\bm{\delta}}}{4\sqrt{\mathcal{N}\beta}}(W_{\bm{k}-\bm{k}'}^{(B)\omega-\omega'}e^{-i(\bm{k}'-\frac{\bm{Q}}{2})\cdot\bm{\delta}}-\bar{W}_{\bm{k}'-\bm{k}}^{(B)\omega'-\omega}e^{i(\bm{k}-\frac{\bm{Q}}{2})\cdot\bm{\delta}})\bigg)e^{-i\omega0^{+}}\\
\mathcal{M}_{44} & =\bigg(\frac{1}{2}(-i\omega)\delta_{\bm{k},\bm{k}'}\delta_{\omega,\omega'}+\frac{i}{2}\frac{\lambda_{\bm{k}'-\bm{k}}^{\omega'-\omega}}{\sqrt{\mathcal{N}\beta}}\nonumber \\
 & -\sum_{\bm{\delta}=\bm{\delta}_{1},\bm{\delta_{2}}}\frac{J_{\bm{\delta}}}{4\sqrt{\mathcal{N}\beta}}(W_{\bm{k}-\bm{k}'}^{(B)\omega-\omega'}e^{i(\bm{k}-\frac{\bm{Q}}{2})\cdot\bm{\delta}}-\bar{W}_{\bm{k}'-\bm{k}}^{(B)\omega'-\omega}e^{-i(\bm{k}'-\frac{\bm{Q}}{2})\cdot\bm{\delta}})\bigg)e^{i\omega'0^{+}}\\
\mathcal{M}_{12} & =\bigg(\sum_{\bm{\delta}=\bm{\delta}_{1},\bm{\delta_{2}}}\frac{J_{\bm{\delta}}}{4\sqrt{\mathcal{N}\beta}}W_{\bm{k}-\bm{k}'}^{(A)\omega-\omega'}e^{i(\bm{k}+\frac{\bm{Q}}{2})\cdot\bm{\delta}}-e^{-i(\bm{k}'+\frac{\bm{Q}}{2})\cdot\bm{\delta}}\bigg)e^{-i(\omega-\omega')0^{+}}\\
\mathcal{M}_{21} & =\sum_{\bm{\delta}=\bm{\delta}_{1},\bm{\delta_{2}}}\frac{J_{\bm{\delta}}}{4\sqrt{\mathcal{N}\beta}}\bar{W}_{\bm{k}'-\bm{k}}^{(A)\omega'-\omega}(e^{-i(\bm{k}'+\frac{\bm{Q}}{2})\cdot\bm{\delta}}-e^{i(\bm{k}+\frac{\bm{Q}}{2})\cdot\bm{\delta}})\\
\mathcal{M}_{34} & =\sum_{\bm{\delta}=\bm{\delta}_{1},\bm{\delta_{2}}}\frac{J_{\bm{\delta}}}{4\sqrt{\mathcal{N}\beta}}W_{\bm{k}-\bm{k}'}^{(A)\omega-\omega'}(e^{-i(\bm{k}'-\frac{\bm{Q}}{2})\cdot\bm{\delta}}-e^{i(\bm{k}-\frac{\bm{Q}}{2})\cdot\bm{\delta}})\\
\mathcal{M}_{43} & =\bigg(\sum_{\bm{\delta}=\bm{\delta}_{1},\bm{\delta_{2}}}\frac{J_{\bm{\delta}}}{4\sqrt{\mathcal{N}\beta}}\bar{W}_{\bm{k}'-\bm{k}}^{(A)\omega'-\omega}(e^{i(\bm{k}-\frac{\bm{Q}}{2})\cdot\bm{\delta}}-e^{-i(\bm{k}'-\frac{\bm{Q}}{2})\cdot\bm{\delta}})\bigg)e^{-i(\omega-\omega')0^{+}}\\
\mathcal{M}_{13} & =\mathcal{M}_{24}=\mathcal{M}_{31}=\mathcal{M}_{42}=-\frac{h}{2}\delta_{\bm{k},\bm{k}'}\\
\mathcal{M}_{14} & =\mathcal{M}_{23}=\mathcal{M}_{32}=\mathcal{M}_{41}=0
\end{align}


\section{Saddle point expansion}

The path integral \eqref{eq:pathintegral} can be evaluated by using
\textit{saddle point approximation}. The effective action is expanded
around its saddle point

\begin{equation}
\mathcal{S}_{\text{eff}}=\mathcal{S}_{0}+\sum_{\alpha}\mathcal{S}_{\alpha}^{(1)}\Delta\phi_{\alpha}+\sum_{\alpha_{1}\alpha_{2}}\mathcal{S}_{\alpha_{1}\alpha_{2}}^{(2)}\Delta\phi_{\alpha_{1}}\Delta\phi_{\alpha_{2}}+\mathcal{S}_{\text{int}}
\end{equation}

where

\begin{equation}
\mathcal{S}_{\text{int}}=\sum_{n=3}^{\infty}\mathcal{S}_{\alpha_{1}..\alpha_{n}}^{(n)}\Delta\phi_{\alpha_{1}}...\Delta\phi_{\alpha_{n}}
\end{equation}

includes third and higher order terms. And

\begin{equation}
\Delta\phi_{\alpha}=\phi_{\alpha}-\phi_{\alpha}^{\text{sp}}
\end{equation}

are the fluctuations about the saddle point with

\begin{equation}
\phi_{\alpha}=(\bar{W}_{\bm{k},\bm{\delta}}^{(r)\omega},W_{\bm{k},\bm{\delta}}^{(r)\omega},\lambda_{\bm{k}}^{\omega})
\end{equation}

where $\alpha$ includes field, momentum and frequency indices. The
coefficients are

\begin{equation}
\mathcal{S}_{\alpha_{1},...\alpha_{n}}^{(n)}=\frac{1}{n!}\frac{\partial^{n}\mathcal{S}_{\text{eff}}}{\partial\phi_{\alpha_{1}}...\partial\phi_{\alpha_{n}}}
\end{equation}


\subsection{Saddle point dynamical matrix}

The saddle point requires $\mathcal{S}_{\alpha}^{(1)}=0$, \textit{i.e.}

\begin{align}
\left.\frac{\partial\mathcal{S}_{\text{eff}}}{\partial\phi_{\alpha}}\right\vert _{\text{sp}} & =\left.\frac{\partial\mathcal{S}_{0}}{\partial\phi_{\alpha}}\right\vert _{\text{sp}}+\left.\frac{\partial\mathcal{S}_{\text{bos}}}{\partial\phi_{\alpha}}\right\vert _{\text{sp}}\nonumber \\
 & =\left.\frac{\partial\mathcal{S}_{0}}{\partial\phi_{\alpha}}\right\vert _{\text{sp}}+\frac{1}{2}\Tr\bigg[\mathcal{G}^{\text{sp}}\frac{\partial\mathcal{G}^{-1}}{\partial\phi_{\alpha}}\bigg]=0\label{eq:saddlepointeGreenself}
\end{align}

The trace goes over all $\alpha$ indices (auxiliary field component,
flavors, momentum, and frequency). We can also write the results explicitly,
according to \eqref{eq:defofaction1}

\begin{align}
\left.\bar{W}_{\bm{i},\bm{\delta}}^{(B)\tau}\right\vert _{\text{sp}} & =-\left.\expval{B_{\bm{i},\bm{\delta}}^{\tau}}\right\vert _{\text{sp}}\nonumber \\
\left.W_{\bm{i},\bm{\delta}}^{(B)\tau}\right\vert _{\text{sp}} & =+\left.\expval{\bar{B}_{\bm{i},\bm{\delta}}^{\tau}}\right\vert _{\text{sp}}\nonumber \\
\left.\bar{W}_{\bm{i},\bm{\delta}}^{(A)\tau}\right\vert _{\text{sp}} & =+\left.\expval{A_{\bm{i},\bm{\delta}}^{\tau}}\right\vert _{\text{sp}}\nonumber \\
\left.W_{\bm{i},\bm{\delta}}^{(A)\tau}\right\vert _{\text{sp}} & =+\left.\expval{\bar{A}_{\bm{i},\bm{\delta}}^{\tau}}\right\vert _{\text{sp}}
\end{align}

where

\begin{equation}
\left.\expval{\mathcal{O}}\right\vert _{\text{sp}}=Z_{\text{bos}}^{-1}\int\mathcal{D}^{2}\bm{b}\mathcal{O}(\bm{r},\tau)e^{-\mathcal{S}_{\text{bos}}}
\end{equation}

From now on, we assume that the auxiliary fields are \textbf{static}
and \textbf{homogeneous}, whose Fourier transforms are Kronecker delta
both in momentum and frequency. Consider the following ansatz,

\begin{align}
\left.\bar{W}_{\bm{\delta}}^{(B)}\right\vert _{\text{sp}} & =-B_{\bm{\delta}} & \left.W_{\bm{\delta}}^{(B)}\right\vert _{\text{sp}} & =B_{\bm{\delta}} & \left.\bar{W}_{\bm{\delta}}^{(A)}\right\vert _{\text{sp}} & =-i{A_{\bm{\delta}}} & \left.W_{\bm{\delta}}^{(A)}\right\vert _{\text{sp}} & =iA_{\bm{\delta}} & \left.\lambda\right\vert _{\text{sp}} & =i\lambda
\end{align}

where $A_{\bm{\delta}}$, $B_{\bm{\delta}}$, and $\lambda$ are real.
The effective action at saddle point now describes non-interacting
bosons. The corresponding dynamical matrix $\mathcal{M}_{\bm{k},\omega}^{\text{sp}}$
takes the form

\begin{equation}
\begin{pmatrix}(i\omega+\lambda+\gamma_{\bm{k}+\frac{\bm{Q}}{2}}^{B})e^{-i\omega0^{+}} & -\gamma_{\bm{k}+\frac{\bm{Q}}{2}}^{A} & -\frac{h}{2} & 0\\[1em]
-\gamma_{\bm{k}+\frac{\bm{Q}}{2}}^{A} & (-i\omega+\lambda+\gamma_{\bm{k}+\frac{\bm{Q}}{2}}^{B})e^{i\omega0^{+}} & 0 & -\frac{h}{2}\\[1em]
-\frac{h}{2} & 0 & (i\omega+\lambda+\gamma_{-\bm{k}+\frac{\bm{Q}}{2}}^{B})e^{-i\omega0^{+}} & -\gamma_{-\bm{k}+\frac{\bm{Q}}{2}}^{A}\\[1em]
0 & -\frac{h}{2} & -\gamma_{-\bm{k}+\frac{\bm{Q}}{2}}^{A} & (-i\omega+\lambda+\gamma_{-\bm{k}+\frac{\bm{Q}}{2}}^{B})e^{i\omega0^{+}}
\end{pmatrix}
\end{equation}

where

\begin{align}
\gamma_{\bm{k}}^{A} & =\sum_{\bm{\delta}=\bm{\delta}_{1},\bm{\delta}_{2}>0}J_{\bm{\delta}}A_{\bm{\delta}}\sin(\bm{k}\cdot\bm{\delta})\\
\gamma_{\bm{k}}^{B} & =\sum_{\bm{\delta}=\bm{\delta}_{1},\bm{\delta}_{2}>0}J_{\bm{\delta}}B_{\bm{\delta}}\cos(\bm{k}\cdot\bm{\delta})
\end{align}

The action $\mathcal{S}_{0}$ at the saddle point is written as

\begin{equation}
\mathcal{S}_{0}=\beta\mathcal{N}\bigg(\sum_{\bm{\delta}>0}J_{\bm{\delta}}(A_{\bm{\delta}}^{2}-B_{\bm{\delta}}^{2})+2\lambda S\bigg)
\end{equation}


\subsection{Saddle point Green function}

The dispersion relations of \textit{free} Schwinger bosons (spinons)
are given by the \textbf{poles} of (Matsubara) Green function, which
is the inverse of the dynamical matrix at the saddle point.

\begin{equation}
\mathcal{G}_{\bm{k},\omega}^{\text{sp}}=(\mathcal{M}_{\bm{k},\omega}^{\text{sp}})^{-1}
\end{equation}

We expect two bands of dispersion relations, because the symmetry
breaking field lifts the degeneracy. Also, since the spinons propagate
both forwards and backwards in imaginary time, the single spinon Green
function should take the form

\begin{equation}
\mathcal{G}_{\bm{k},\omega}^{\text{sp}}(\bm{k},i\omega)=\frac{g_{\bm{k}}^{-+}}{i\omega-\epsilon_{\bm{k}}^{+}}+\frac{g_{\bm{k}}^{++}}{i\omega+\epsilon_{\bm{k}}^{+}}+\frac{g_{\bm{k}}^{--}}{i\omega-\epsilon_{\bm{k}}^{-}}+\frac{g_{\bm{k}}^{+-}}{i\omega+\epsilon_{\bm{k}}^{-}}
\end{equation}

where $\epsilon_{\bm{k}}^{\pm}$ are the dispersion relations, and
$g_{\bm{k}}^{\sigma\sigma'}$ are $4\times4$ matrices yet to be determined.
\\

The inverse of a general matrix can be calculated as

\begin{equation}
\mathcal{G}=\mathcal{M}^{-1}=\frac{1}{\det(\mathcal{M})}\mathcal{C}^{T}
\end{equation}

where $\mathcal{C}$ is the co-factor matrix of $\mathcal{M}$. Thus
the poles can be determined by

%dispersion relations are determined by the roots of ($i\omega$) of the equation 

\begin{equation}
\det\mathcal{M}_{\bm{k},\omega}^{\text{sp}}(i\omega)=0
\end{equation}

The dispersion relations take the form (see A.1)

\begin{equation}
\epsilon_{\bm{k}}^{\pm}=\sqrt{\frac{1}{2}\bigg[\big(\alpha_{1}^{2}+\alpha_{2}^{2}+\frac{h^{2}}{2}\big)\pm\Delta_{\bm{k}}^{2}\bigg]}
\end{equation}

where

\begin{equation}
\Delta_{\bm{k}}^{2}=\sqrt{\big(\alpha_{1}^{2}-\alpha_{2}^{2})^{2}+\bigg[\bigg((\lambda+\gamma_{1}^{B})+(\lambda+\gamma_{2}^{B})\bigg)^{2}-(\gamma_{1}^{A}-\gamma_{2}^{A})^{2}\bigg]h^{2}}
\end{equation}

The $4\times4$ matrices $g_{\bm{k}}^{\pm\sigma}$ ($\sigma=\pm)$
are (see A.2)

\begin{equation}
g_{\bm{k}}^{-\sigma}=\begin{pmatrix}A_{\bm{k}}^{\sigma} & C_{\bm{k}}^{\sigma} & B_{\bm{k}}^{\sigma} & D_{\bm{k}}^{\sigma}\\
C_{\bm{k}}^{\sigma} & E_{\bm{k}}^{\sigma} & D_{\bm{-k}}^{\sigma} & F_{\bm{k}}^{\sigma}\\
B_{\bm{k}}^{\sigma} & D_{\bm{-k}}^{\sigma} & A_{\bm{-k}}^{\sigma} & C_{\bm{-k}}^{\sigma}\\
D_{\bm{k}}^{\sigma} & F_{\bm{k}}^{\sigma} & C_{\bm{-k}}^{\sigma} & E_{\bm{-k}}^{\sigma}
\end{pmatrix}
\end{equation}

\begin{equation}
g_{\bm{k}}^{+\sigma}=-\begin{pmatrix}E_{\bm{k}}^{\sigma} & C_{\bm{k}}^{\sigma} & F_{\bm{k}}^{\sigma} & D_{\bm{-k}}^{\sigma}\\
C_{\bm{k}}^{\sigma} & A_{\bm{k}}^{\sigma} & D_{\bm{k}}^{\sigma} & B_{\bm{k}}^{\sigma}\\
F_{\bm{k}}^{\sigma} & D_{\bm{k}}^{\sigma} & E_{\bm{-k}}^{\sigma} & C_{\bm{-k}}^{\sigma}\\
D_{\bm{-k}}^{\sigma} & B_{\bm{k}}^{\sigma} & C_{\bm{-k}}^{\sigma} & A_{\bm{-k}}^{\sigma}
\end{pmatrix}
\end{equation}
The form of matrix elements are the same as those given in the appendix
of (arXiv:1802.06878v3 {[}cond-mat.str-el{]}) . Notice that the convergence
factors are ``loaded'' into the dynamical matrix, not into the Green
function.

\subsection{Saddle point equations}

We will use (\ref{eq:saddlepointeGreenself}) to write down the saddle
point equations explicitly. 

First consider the $\lambda$-field, 

\begin{equation}
\frac{\partial\mathcal{G}^{-1}}{\partial\lambda}=\begin{pmatrix}e^{-i\omega0^{+}}\\
 & e^{i\omega0^{+}}\\
 &  & e^{-i\omega0^{+}}\\
 &  &  & e^{i\omega0^{+}}
\end{pmatrix}
\end{equation}

The second term of (\ref{eq:saddlepointeGreenself}) can be calculated
as following,

\begin{align}
\frac{1}{2}\Tr\bigg[\mathcal{G}^{\text{sp}}\frac{\partial\mathcal{G}^{-1}}{\partial\lambda}\bigg] & =\frac{1}{2}\sum_{\bm{k}}\sum_{\omega}\bigg[\frac{1}{i\omega-\epsilon_{\bm{k}}^{+}}\bigg(A_{\bm{k}}^{+}e^{-i\omega0^{+}}+E_{\bm{k}}^{+}e^{i\omega0^{+}}+A_{-\bm{k}}^{+}e^{-i\omega0+}+E_{\bm{-k}}^{+}e^{i\omega0^{+}}\bigg)\nonumber \\
 & -\frac{1}{i\omega+\epsilon_{\bm{k}}^{+}}\bigg(E_{\bm{k}}^{+}e^{-i\omega0^{+}}+A_{\bm{k}}^{+}e^{i\omega0^{+}}+E_{-\bm{k}}^{+}e^{-i\omega0+}+A_{\bm{-k}}^{+}e^{i\omega0^{+}}\bigg)\nonumber \\
 & +\frac{1}{i\omega-\epsilon_{\bm{k}}^{-}}\bigg(A_{\bm{k}}^{-}e^{-i\omega0^{+}}+E_{\bm{k}}^{-}e^{i\omega0^{+}}+A_{-\bm{k}}^{-}e^{-i\omega0+}+E_{\bm{-k}}^{-}e^{i\omega0^{+}}\bigg)\nonumber \\
 & -\frac{1}{i\omega+\epsilon_{\bm{k}}^{-}}\bigg(E_{\bm{k}}^{-}e^{-i\omega0^{+}}+A_{\bm{k}}^{-}e^{i\omega0^{+}}+E_{-\bm{k}}^{-}e^{-i\omega0+}+A_{\bm{-k}}^{-}e^{i\omega0^{+}}\bigg)\bigg]\label{eq:lambdasum}
\end{align}

There are four types of Matsubara sums in (\ref{eq:lambdasum}). This
step elucidates the reason to keep the convergence factor: 1. the
sum diverges without the factor; 2. the value of the summation depends
on the sign (in the exponential) of the factor.

\begin{align*}
\sum_{\omega}\frac{e^{i\omega0+}}{i\omega-\epsilon} & =-\beta\frac{1}{e^{\beta\epsilon}-1}=-\beta n(\epsilon)\\
\sum_{\omega}\frac{e^{i\omega0+}}{i\omega+\epsilon} & =\beta\big(n(\epsilon)+1\big)\\
\sum_{\omega}\frac{e^{-i\omega0^{+}}}{i\omega-\epsilon} & =-\beta\big(n(\epsilon)+1\big)\\
\sum_{\omega}\frac{e^{-i\omega0+}}{i\omega+\epsilon} & =\beta n(\epsilon)
\end{align*}

Thus (\ref{eq:lambdasum}) after Matsubara sum is given by

\begin{align*}
\frac{1}{2}\Tr\bigg[\mathcal{G}^{\text{sp}}\frac{\partial\mathcal{G}^{-1}}{\partial\lambda}\bigg] & =-\frac{\beta}{2}\sum_{\bm{k}}\bigg((A_{\bm{k}}^{+}+A_{\bm{-k}}^{+})\big(n(\epsilon_{\bm{k}}^{+})+1\big)+(E_{\bm{k}}^{+}+E_{\bm{-k}}^{+})n(\epsilon_{\bm{k}}^{+})\\
 & +(A_{\bm{k}}^{-}+A_{\bm{-k}}^{-})\big(n(\epsilon_{\bm{k}}^{-})+1\big)+(E_{\bm{k}}^{-}+E_{\bm{-k}}^{-})n(\epsilon_{\bm{k}}^{-})\bigg)
\end{align*}

Refer to (arXiv:1802.06878v3 {[}cond-mat.str-el{]}) for the expressions
of $A_{\bm{k}}, E_{\bm{k}}$. We will focus on the physics at $T=0$,
note that

\begin{equation}
\lim_{\beta\rightarrow\infty}\beta n(\epsilon)=0
\end{equation}

Thus

\begin{equation}
\lim_{\beta\rightarrow\infty}\frac{1}{2}\Tr\bigg[\mathcal{G}^{\text{sp}}\frac{\partial\mathcal{G}^{-1}}{\partial\lambda}\bigg]=-\frac{\beta}{2}\sum_{\bm{k}}\bigg((A_{\bm{k}}^{+}+A_{\bm{-k}}^{+}+A_{\bm{k}}^{-}+A_{\bm{-k}}^{-}\bigg)
\end{equation}

While 

\begin{equation}
\left.\frac{\partial\mathcal{S}_{0}}{\partial\lambda}\right\vert _{\text{sp}}=2\mathcal{N\beta}S
\end{equation}

Thus the saddle point equation for the $\lambda$-field takes the
form

\begin{equation}
\boxed{{S=\frac{1}{\mathcal{N}}\sum_{\bm{k}}\bigg[u_{\bm{k}}^{+^{2}}\frac{\Delta_{-\bm{k}}^{+^{2}}}{\Delta_{\bm{k}}^{2}}+u_{\bm{k}}^{-^{2}}\frac{\Delta_{-\bm{k}}^{-^{2}}}{\Delta_{\bm{k}}^{2}}+(v_{-\bm{k}}^{+^{2}}-v_{-\bm{k}}^{-^{2}})\frac{h^{2}}{4\Delta_{\bm{k}}^{2}}\bigg]}}
\end{equation}

Next, we focus on the calculation of the saddle point equation for
$A_{\bm{\bm{\delta}}}$. The derivative of the dynamical matrix with
respect to $A_{\bm{\bm{\delta}}}$ is given by 

\begin{equation}
\frac{\partial\mathcal{G}^{-1}}{\partial A_{\bm{\delta}}}=\begin{pmatrix}0 & -J_{\bm{\delta}}\sin\bigg(\bm{(k+\frac{\bm{Q}}{2})}\cdot\bm{\delta}\bigg) & 0 & 0\\
-J_{\bm{\delta}}\sin\bigg(\bm{(k+\frac{\bm{Q}}{2})}\cdot\bm{\delta}\bigg) & 0 & 0 & 0\\
0 & 0 & 0 & -J_{\bm{\delta}}\sin\bigg(\bm{(-k+\frac{\bm{Q}}{2})}\cdot\bm{\delta}\bigg)\\
0 & 0 & -J_{\bm{\delta}}\sin\bigg(\bm{(-k+\frac{\bm{Q}}{2})}\cdot\bm{\delta}\bigg) & 0
\end{pmatrix}
\end{equation}

The second term of (\ref{eq:saddlepointeGreenself}) can be calculated
as following,

\begin{align}
\frac{1}{2}\Tr\bigg[\mathcal{G}^{\text{sp}}\frac{\partial\mathcal{G}^{-1}}{\partial A_{\bm{\delta}}}\bigg] & =-\sum_{\bm{k}}\sum_{\bm{\omega}}\bigg\{\frac{1}{i\omega-\epsilon_{\bm{k}}^{+}}\bigg[J_{\bm{\delta}}\sin\bigg(\bm{(k+\frac{\bm{Q}}{2})}\cdot\bm{\delta}\bigg)C_{\bm{k}}^{+}+J_{\bm{\delta}}\sin\bigg(\bm{(-k+\frac{\bm{Q}}{2})}\cdot\bm{\delta}\bigg)C_{-\bm{k}}^{+}\bigg]\nonumber \\
 & -\frac{1}{i\omega+\epsilon_{\bm{k}}^{+}}\bigg[J_{\bm{\delta}}\sin\bigg(\bm{(k+\frac{\bm{Q}}{2})}\cdot\bm{\delta}\bigg)C_{\bm{k}}^{+}+J_{\bm{\delta}}\sin\bigg(\bm{(-k+\frac{\bm{Q}}{2})}\cdot\bm{\delta}\bigg)C_{-\bm{k}}^{+}\bigg]\nonumber \\
 & +\frac{1}{i\omega-\epsilon_{\bm{k}}^{-}}\bigg[J_{\bm{\delta}}\sin\bigg(\bm{(k+\frac{\bm{Q}}{2})}\cdot\bm{\delta}\bigg)C_{\bm{k}}^{-}+J_{\bm{\delta}}\sin\bigg(\bm{(-k+\frac{\bm{Q}}{2})}\cdot\bm{\delta}\bigg)C_{-\bm{k}}^{-}\bigg]\nonumber \\
 & -\frac{1}{i\omega+\epsilon_{\bm{k}}^{-}}\bigg[J_{\bm{\delta}}\sin\bigg(\bm{(k+\frac{\bm{Q}}{2})}\cdot\bm{\delta}\bigg)C_{\bm{k}}^{-}+J_{\bm{\delta}}\sin\bigg(\bm{(-k+\frac{\bm{Q}}{2})}\cdot\bm{\delta}\bigg)C_{-\bm{k}}^{-}\bigg]\bigg\}\nonumber \\
 & =-\sum_{\bm{k}}\sum_{\bm{\omega}}\bigg\{\bigg[J_{\bm{\delta}}\sin\bigg(\bm{(k+\frac{\bm{Q}}{2})}\cdot\bm{\delta}\bigg)C_{\bm{k}}^{+}+J_{\bm{\delta}}\sin\bigg(\bm{(-k+\frac{\bm{Q}}{2})}\cdot\bm{\delta}\bigg)C_{-\bm{k}}^{+}\bigg]\bigg(\frac{1}{i\omega-\epsilon_{\bm{k}}^{+}}-\frac{1}{i\omega+\epsilon_{\bm{k}}^{+}}\bigg)\nonumber \\
 & +\bigg[J_{\bm{\delta}}\sin\bigg(\bm{(k+\frac{\bm{Q}}{2})}\cdot\bm{\delta}\bigg)C_{\bm{k}}^{-}+J_{\bm{\delta}}\sin\bigg(\bm{(-k+\frac{\bm{Q}}{2})}\cdot\bm{\delta}\bigg)C_{-\bm{k}}^{-}\bigg]\bigg(\frac{1}{i\omega-\epsilon_{\bm{k}}^{-}}-\frac{1}{i\omega+\epsilon_{\bm{k}}^{-}}\bigg)\bigg\}\label{eq:saddletraceA}
\end{align}

The Matsubara sums in \ref{eq:saddletraceA} converge, 

\begin{equation}
\sum_{\omega}\bigg(\frac{1}{i\omega-\epsilon}-\frac{1}{i\omega+\epsilon}\bigg)=-\beta\big(1+2n(\epsilon)\big)
\end{equation}

At $T=0$,

\begin{equation}
\lim_{\beta\rightarrow\infty}\frac{1}{2}\Tr\bigg[\mathcal{G}^{\text{sp}}\frac{\partial\mathcal{G}^{-1}}{\partial A_{\bm{\delta}}}\bigg]=\beta J_{\bm{\delta}}\sum_{\bm{k}}\bigg[\sin\bigg(\bm{(k+\frac{\bm{Q}}{2})}\cdot\bm{\delta}\bigg)(C_{\bm{k}}^{+}+C_{\bm{k}}^{-})+\sin\bigg(\bm{(-k+\frac{\bm{Q}}{2})}\cdot\bm{\delta}\bigg)(C_{\bm{-k}}^{+}+C_{-\bm{k}}^{-})\bigg]
\end{equation}

While

\begin{equation}
\frac{\partial\mathcal{S}_{0}}{\partial A_{\bm{\delta}}}=2\beta\mathcal{N}J_{\bm{\delta}}A_{\bm{\delta}}
\end{equation}

Thus the saddle point equations for $A_{\bm{\delta}}$-fields are

\begin{equation}
\boxed{A_{\bm{\delta}}=\frac{1}{\mathcal{N}}\sum_{\bm{k}}\bigg[z_{\bm{k}}^{+}\frac{\Delta_{-\bm{k}}^{+^{2}}}{\Delta_{\bm{k}}^{2}}+z_{\bm{k}}^{-}\frac{\Delta_{-\bm{k}}^{-^{2}}}{\Delta_{\bm{k}}^{2}}-(z_{-\bm{k}}^{+}-z_{-\bm{k}}^{-})\frac{h^{2}}{4\Delta_{\bm{k}}^{2}}\bigg]\sin\bigg((\bm{k}+\frac{\bm{Q}}{2})\cdot\bm{\delta}\bigg)}
\end{equation}

Consider the $B_{\bm{\delta}}$ - fields, 

\begin{equation}
\frac{\partial\mathcal{G}^{-1}}{\partial B_{\bm{\delta}}}=\begin{pmatrix}J_{\bm{\delta}}\cos\bigg(\bm{(k+\frac{\bm{Q}}{2})}\cdot\bm{\delta}\bigg)e^{-i\omega0^{+}} & 0 & 0 & 0\\
0 & J_{\bm{\delta}}\cos\bigg(\bm{(k+\frac{\bm{Q}}{2})}\cdot\bm{\delta}\bigg)e^{i\omega0^{+}} & 0 & 0\\
0 & 0 & J_{\bm{\delta}}\cos\bigg(\bm{(-k+\frac{\bm{Q}}{2})}\cdot\bm{\delta}\bigg)e^{-i\omega0^{+}} & 0\\
0 & 0 & 0 & J_{\bm{\delta}}\cos\bigg(\bm{(-k+\frac{\bm{Q}}{2})}\cdot\bm{\delta}\bigg)e^{i\omega0^{+}}
\end{pmatrix}
\end{equation}

The second term of (\ref{eq:saddlepointeGreenself}) can be calculated
as following,

\begin{align*}
\frac{1}{2}\Tr\bigg[\mathcal{G}^{\text{sp}}\frac{\partial\mathcal{G}^{-1}}{\partial B_{\bm{\delta}}}\bigg] & =J_{\bm{\delta}}\cos\bigg(\bm{(k+\frac{\bm{Q}}{2})}\cdot\bm{\delta}\bigg)\sum_{\bm{k}}\sum_{\bm{\omega}}\bigg\{\frac{1}{i\omega-\epsilon_{\bm{k}}^{+}}(A_{\bm{k}}^{+}e^{-i\omega0^{+}}+E_{\bm{k}}^{+}e^{i\omega0^{+}})-\frac{1}{i\omega+\epsilon_{\bm{k}}^{+}}(E_{\bm{k}}^{+}e^{-i\omega0^{+}}+A_{\bm{k}}^{+}e^{i\omega0^{+}})\\
 & +\frac{1}{i\omega-\epsilon_{\bm{k}}^{-}}(A_{\bm{k}}^{-}e^{-i\omega0^{+}}+E_{\bm{k}}^{-}e^{i\omega0^{+}})-\frac{1}{i\omega+\epsilon_{\bm{k}}^{-}}(E_{\bm{k}}^{-}e^{-i\omega0^{+}}+A_{\bm{k}}^{-}e^{i\omega0^{+}})\bigg\}\\
 & =-2\beta J_{\bm{\delta}}\cos\bigg(\bm{(k+\frac{\bm{Q}}{2})}\cdot\bm{\delta}\bigg)\sum_{\bm{k}}\bigg\{ A_{\bm{k}}^{+}\big[n(\epsilon_{\bm{k}}^{+})+1\big]+E_{\bm{k}}^{+}n(\epsilon_{\bm{k}}^{+})+A_{\bm{k}}^{-}\big[n(\epsilon_{\bm{k}}^{-})+1\big]+E_{\bm{k}}^{-}n(\epsilon_{\bm{k}}^{-})\bigg\}
\end{align*}

At $T=0$,

\begin{align}
\lim_{\beta\rightarrow\infty}\frac{1}{2}\Tr\bigg[\mathcal{G}^{\text{sp}}\frac{\partial\mathcal{G}^{-1}}{\partial B_{\bm{\delta}}}\bigg] & =-2\beta J_{\bm{\delta}}\cos\bigg(\bm{(k+\frac{\bm{Q}}{2})}\cdot\bm{\delta}\bigg)\sum_{\bm{k}}(A_{\bm{k}}^{+}+A_{\bm{k}}^{-})\\
 & =2\beta J_{\bm{\delta}}\cos\bigg(\bm{(k+\frac{\bm{Q}}{2})}\cdot\bm{\delta}\bigg)\sum_{\bm{k}}\bigg[u_{\bm{k}}^{+^{2}}\frac{\Delta_{-\bm{k}}^{+^{2}}}{\Delta_{\bm{k}}^{2}}+u_{\bm{k}}^{-^{2}}\frac{\Delta_{-\bm{k}}^{-^{2}}}{\Delta_{\bm{k}}^{2}}+(v_{-\bm{k}}^{+^{2}}-v_{-\bm{k}}^{-^{2}})\frac{h^{2}}{4\Delta_{\bm{k}}^{2}}\bigg]
\end{align}

While

\begin{equation}
\frac{\partial\mathcal{S}_{0}}{\partial B_{\bm{\delta}}}=-2\beta\mathcal{N}J_{\bm{\delta}}B_{\bm{\delta}}
\end{equation}

The saddle point equations for $B_{\bm{\delta}}$-fields are

\begin{equation}
\boxed{B_{\bm{\delta}}=\frac{1}{\mathcal{N}}\sum_{\bm{k}}\sum_{\bm{k}}\bigg[u_{\bm{k}}^{+^{2}}\frac{\Delta_{-\bm{k}}^{+^{2}}}{\Delta_{\bm{k}}^{2}}+u_{\bm{k}}^{-^{2}}\frac{\Delta_{-\bm{k}}^{-^{2}}}{\Delta_{\bm{k}}^{2}}+(v_{-\bm{k}}^{+^{2}}-v_{-\bm{k}}^{-^{2}})\frac{h^{2}}{4\Delta_{\bm{k}}^{2}}\bigg]\cos\bigg(\bm{(k+\frac{\bm{Q}}{2})}\cdot\bm{\delta}\bigg)}
\end{equation}


\appendix
%dummy comment inserted by tex2lyx to ensure that this paragraph is not empty


\section{Detail calculation of Green function matrix}

\subsection{Dispersion relations}

For simplicity, we introduce short-handed notation $1=\bm{k}+\frac{\bm{Q}}{2}$
and $2=-\bm{k}+\frac{\bm{Q}}{2}$. The determinant is calculated as
following

\begin{align}
 & \det\mathcal{M}_{\bm{k},\omega}^{\text{sp}}\nonumber \\
 & =(i\omega+\lambda+\gamma_{1}^{B})e^{-i\omega0^{+}}\begin{vmatrix}(-i\omega+\lambda+\gamma_{1}^{B})e^{i\omega0^{+}} & 0 & -h/2\\
0 & (i\omega+\lambda+\gamma_{2}^{B})e^{-i\omega0^{+}} & -\gamma_{2}^{A}\\
-h/2 & -\gamma_{2}^{A} & (-i\omega+\lambda+\gamma_{2}^{B})e^{i\omega0^{+}}
\end{vmatrix}\nonumber \\
 & +\gamma_{1}^{A}\begin{vmatrix}-\gamma_{1}^{A} & 0 & -h/2\\
-h/2 & (i\omega+\lambda+\gamma_{2}^{B})e^{-i\omega0^{+}} & -\gamma_{2}^{A}\\
0 & -\gamma_{2}^{A} & (-i\omega+\lambda+\gamma_{2}^{B})e^{i\omega0^{+}}
\end{vmatrix}\nonumber \\
 & -\frac{h}{2}\begin{vmatrix}-\gamma_{1}^{A} & (-i\omega+\lambda+\gamma_{1}^{B})e^{i\omega0^{+}} & -h/2\\
-h/2 & 0 & -\gamma_{2}^{A}\\
0 & -h/2 & (-i\omega+\lambda+\gamma_{2}^{B})e^{i\omega0^{+}}
\end{vmatrix}\nonumber \\
 & =[(\lambda+\gamma_{1}^{B})^{2}+\omega^{2}][(\lambda+\gamma_{2}^{B})^{2}+\omega^{2}-(\gamma_{2}^{A})^{2}]-\frac{h^{2}}{4}(i\omega+\lambda+\gamma_{1}^{B})(i\omega+\lambda+\gamma_{2}^{B})e^{-2i\omega0^{+}}\nonumber \\
 & -\gamma_{1}^{A}\big(\gamma_{1}^{A}[(\lambda+\gamma_{2}^{B})^{2}+\omega^{2}-(\gamma_{2}^{A})^{2}]+\gamma_{2}^{A}\frac{h^{2}}{4}\big)\nonumber \\
 & -\frac{h^{2}}{4}\big(\gamma_{1}^{A}\gamma_{2}^{A}+(-i\omega+\lambda+\gamma_{1}^{B})(-i\omega+\lambda+\gamma_{2}^{B})-\frac{h^{2}}{4}\big)=0\label{eq:det}
\end{align}

Define $\alpha_{\bm{k}}^{2}=(\lambda+\gamma_{\bm{k}}^{B})^{2}-(\gamma_{\bm{k}}^{A})^{2}$,
the expression \eqref{eq:det} can be simplified as

\begin{align}
\det\mathcal{M}_{\bm{k},\omega}^{\text{sp}}=\omega^{4}+(\alpha_{1}^{2}+\alpha_{2}^{2}+\frac{h^{2}}{2})\omega^{2}+\alpha_{1}^{2}\alpha_{2}^{2}-\frac{h^{2}}{2}(\lambda+\gamma_{1}^{B})(\lambda+\gamma_{2}^{B})-\gamma_{1}^{A}\gamma_{2}^{A}\frac{h^{2}}{2}+\frac{h^{4}}{16}=0
\end{align}


\subsection{Matrix elements}

The next step is to calculate the adjunct matrix $\mathcal{A}=\mathcal{C}^{T}$,

\begin{align}
\mathcal{A}_{11} & =\begin{vmatrix}(-i\omega+\lambda+\gamma_{\bm{k}+\frac{\bm{Q}}{2}}^{B})e^{i\omega0^{+}} & 0 & -\frac{h}{2}\\
0 & (i\omega+\lambda+\gamma_{-\bm{k}+\frac{\bm{Q}}{2}}^{B})e^{-i\omega0^{+}} & -\gamma_{-\bm{k}+\frac{\bm{Q}}{2}}^{A}\\
-\frac{h}{2} & -\gamma_{-\bm{k}+\frac{\bm{Q}}{2}}^{A} & (-i\omega+\lambda+\gamma_{-\bm{k}+\frac{\bm{Q}}{2}}^{B})e^{i\omega0^{+}}
\end{vmatrix}\nonumber \\
 & =\boxed{(-i\omega+\lambda+\gamma_{1}^{B})e^{i\omega0^{+}}(\omega^{2}+\alpha_{2}^{2})-\frac{h^{2}}{4}(i\omega+\lambda+\gamma_{2}^{B})e^{-i\omega0^{+}}}
\end{align}

\begin{align}
\mathcal{A}_{21} & =-\begin{vmatrix}-\gamma_{\bm{k}+\frac{\bm{Q}}{2}}^{A} & 0 & -\frac{h}{2}\\
-\frac{h}{2} & (i\omega+\lambda+\gamma_{-\bm{k}+\frac{\bm{Q}}{2}}^{B})e^{-i\omega0^{+}} & -\gamma_{-\bm{k}+\frac{\bm{Q}}{2}}^{A}\\[1em]
0 & -\gamma_{-\bm{k}+\frac{\bm{Q}}{2}}^{A} & (-i\omega+\lambda+\gamma_{-\bm{k}+\frac{\bm{Q}}{2}}^{B})e^{i\omega0^{+}}
\end{vmatrix}\nonumber \\
 & =\boxed{\gamma_{1}^{A}(\omega^{2}+\alpha_{2}^{2})+\frac{h^{2}}{4}\gamma_{2}^{A}}
\end{align}

\begin{align}
\mathcal{A}_{31} & =\begin{vmatrix}-\gamma_{\bm{k}+\frac{\bm{Q}}{2}}^{A} & (-i\omega+\lambda+\gamma_{\bm{k}+\frac{\bm{Q}}{2}}^{B})e^{i\omega0^{+}} & -\frac{h}{2}\\
-\frac{h}{2} & 0 & -\gamma_{-\bm{k}+\frac{\bm{Q}}{2}}^{A}\\[1em]
0 & -\frac{h}{2} & (-i\omega+\lambda+\gamma_{-\bm{k}+\frac{\bm{Q}}{2}}^{B})e^{i\omega0^{+}}
\end{vmatrix}\nonumber \\
 & =\boxed{\frac{h}{2}[\gamma_{1}^{A}\gamma_{2}^{A}-(-i\omega+\lambda+\gamma_{1}^{B})(-i\omega+\lambda+\gamma_{2}^{B})e^{2i\omega0^{+}}-\frac{h^{2}}{4}]}
\end{align}

\begin{align}
\mathcal{A}_{41} & =-\begin{vmatrix}-\gamma_{\bm{k}+\frac{\bm{Q}}{2}}^{A} & (-i\omega+\lambda+\gamma_{\bm{k}+\frac{\bm{Q}}{2}}^{B})e^{i\omega0^{+}} & 0\\
-\frac{h}{2} & 0 & (i\omega+\lambda+\gamma_{-\bm{k}+\frac{\bm{Q}}{2}}^{B})e^{-i\omega0^{+}}\\
0 & -\frac{h}{2} & -\gamma_{-\bm{k}+\frac{\bm{Q}}{2}}^{A}
\end{vmatrix}\nonumber \\
 & =\boxed{\frac{h}{2}[\gamma_{1}^{A}(i\omega+\lambda+\gamma_{2}^{B})e^{-i\omega0^{+}}+\gamma_{2}^{A}(-i\omega+\lambda+\gamma_{1}^{B})e^{i\omega0^{+}}]}
\end{align}

\begin{align}
\mathcal{A}_{12} & =-\begin{vmatrix}-\gamma_{\bm{k}+\frac{\bm{Q}}{2}}^{A} & -\frac{h}{2} & 0\\
0 & (i\omega+\lambda+\gamma_{-\bm{k}+\frac{\bm{Q}}{2}}^{B})e^{-i\omega0^{+}} & -\gamma_{-\bm{k}+\frac{\bm{Q}}{2}}^{A}\\
-\frac{h}{2} & -\gamma_{-\bm{k}+\frac{\bm{Q}}{2}}^{A} & (-i\omega+\lambda+\gamma_{-\bm{k}+\frac{\bm{Q}}{2}}^{B})e^{i\omega0^{+}}
\end{vmatrix}\nonumber \\
 & =\boxed{\gamma_{1}^{A}(\omega^{2}+\alpha_{2}^{2})+\frac{h^{2}}{4}\gamma_{2}^{A}}
\end{align}

\begin{align}
\mathcal{A}_{22} & =\begin{vmatrix}(i\omega+\lambda+\gamma_{\bm{k}+\frac{\bm{Q}}{2}}^{B})e^{-i\omega0^{+}} & -\frac{h}{2} & 0\\
-\frac{h}{2} & (i\omega+\lambda+\gamma_{-\bm{k}+\frac{\bm{Q}}{2}}^{B})e^{-i\omega0^{+}} & -\gamma_{-\bm{k}+\frac{\bm{Q}}{2}}^{A}\\
0 & -\gamma_{-\bm{k}+\frac{\bm{Q}}{2}}^{A} & (-i\omega+\lambda+\gamma_{-\bm{k}+\frac{\bm{Q}}{2}}^{B})e^{i\omega0^{+}}
\end{vmatrix}\nonumber \\
 & =\boxed{(i\omega+\lambda+\gamma_{1}^{B})e^{-i\omega0^{+}}(\omega^{2}+\alpha_{2}^{2})-\frac{h^{2}}{4}(-i\omega+\lambda+\gamma_{2}^{B})e^{i\omega0^{+}}}
\end{align}

\begin{align}
\mathcal{A}_{32} & =-\begin{vmatrix}(i\omega+\lambda+\gamma_{\bm{k}+\frac{\bm{Q}}{2}}^{B})e^{-i\omega0^{+}} & -\gamma_{\bm{k}+\frac{\bm{Q}}{2}}^{A} & 0\\
-\frac{h}{2} & 0 & -\gamma_{-\bm{k}+\frac{\bm{Q}}{2}}^{A}\\
0 & -\frac{h}{2} & (-i\omega+\lambda+\gamma_{-\bm{k}+\frac{\bm{Q}}{2}}^{B})e^{i\omega0^{+}}
\end{vmatrix}\nonumber \\
%\boxed{\frac{h}{2}[(i\omega+\lambda+\gamma_{1}^{B})e^{-i\omega0^{+}}\gamma_{2}^{A}+\frac{h}{2}(-i\omega+\lambda+\gamma_{2}^{B})e^{i\omega0^{+}}]}
 & =\boxed{\frac{h}{2}[(i\omega+\lambda+\gamma_{1}^{B})e^{-i\omega0^{+}}\gamma_{2}^{A}+(-i\omega+\lambda+\gamma_{2}^{B})e^{i\omega0^{+}}\gamma_{1}^{A}]}
\end{align}

\begin{align}
\mathcal{A}_{42} & =\begin{vmatrix}(i\omega+\lambda+\gamma_{\bm{k}+\frac{\bm{Q}}{2}}^{B})e^{-i\omega0^{+}} & -\gamma_{\bm{k}+\frac{\bm{Q}}{2}}^{A} & -\frac{h}{2}\\
-\frac{h}{2} & 0 & (i\omega+\lambda+\gamma_{-\bm{k}+\frac{\bm{Q}}{2}}^{B})e^{-i\omega0^{+}}\\
0 & -\frac{h}{2} & -\gamma_{-\bm{k}+\frac{\bm{Q}}{2}}^{A}
\end{vmatrix}\nonumber \\
 & =\boxed{\frac{h}{2}[(i\omega+\lambda+\gamma_{1}^{B})(i\omega+\lambda+\gamma_{2}^{B})e^{-2i\omega0^{+}}+\gamma_{1}^{A}\gamma_{2}^{A}-\frac{h^{2}}{4}]}
\end{align}

\begin{align}
\mathcal{A}_{13} & =\begin{vmatrix}-\gamma_{\bm{k}+\frac{\bm{Q}}{2}}^{A} & -\frac{h}{2} & 0\\
(-i\omega+\lambda+\gamma_{\bm{k}+\frac{\bm{Q}}{2}}^{B})e^{i\omega0^{+}} & 0 & -\frac{h}{2}\\
-\frac{h}{2} & -\gamma_{-\bm{k}+\frac{\bm{Q}}{2}}^{A} & (-i\omega+\lambda+\gamma_{-\bm{k}+\frac{\bm{Q}}{2}}^{B})e^{i\omega0^{+}}
\end{vmatrix}\nonumber \\
 & =\boxed{\frac{h}{2}[\gamma_{1}^{A}\gamma_{2}^{A}+(-i\omega+\lambda+\gamma_{1}^{B})(-i\omega+\lambda+\gamma_{2}^{B})e^{2i\omega0^{+}}-\frac{h^{2}}{4}]}
\end{align}

\begin{align}
\mathcal{A}_{23} & =-\begin{vmatrix}(i\omega+\lambda+\gamma_{\bm{k}+\frac{\bm{Q}}{2}}^{B})e^{-i\omega0^{+}} & -\frac{h}{2} & 0\\
-\gamma_{\bm{k}+\frac{\bm{Q}}{2}}^{A} & 0 & -\frac{h}{2}\\
0 & -\gamma_{-\bm{k}+\frac{\bm{Q}}{2}}^{A} & (-i\omega+\lambda+\gamma_{-\bm{k}+\frac{\bm{Q}}{2}}^{B})e^{i\omega0^{+}}
\end{vmatrix}\nonumber \\
%
 & =\boxed{\frac{h}{2}[(i\omega+\lambda+\gamma_{1}^{B})e^{-i\omega0^{+}}\gamma_{1}^{A}+(-i\omega+\lambda+\gamma_{2}^{B})e^{i\omega0^{+}}\gamma_{1}^{A}]}
\end{align}

\begin{align}
\mathcal{A}_{33} & =\begin{vmatrix}(i\omega+\lambda+\gamma_{\bm{k}+\frac{\bm{Q}}{2}}^{B})e^{-i\omega0^{+}} & -\gamma_{\bm{k}+\frac{\bm{Q}}{2}}^{A} & 0\\
-\gamma_{\bm{k}+\frac{\bm{Q}}{2}}^{A} & (-i\omega+\lambda+\gamma_{\bm{k}+\frac{\bm{Q}}{2}}^{B})e^{i\omega0^{+}} & -\frac{h}{2}\\
0 & -\frac{h}{2} & (-i\omega+\lambda+\gamma_{-\bm{k}+\frac{\bm{Q}}{2}}^{B})e^{i\omega0^{+}}
\end{vmatrix}\nonumber \\
 & =\boxed{(-i\omega+\lambda+\gamma_{2}^{B})e^{i\omega0^{+}}(\omega^{2}+\alpha_{1}^{2})-\frac{h^{2}}{4}(-i\omega+\lambda+\gamma_{1}^{B})e^{-i\omega0^{+}}}
\end{align}

\begin{align}
\mathcal{A}_{43} & =-\begin{vmatrix}(i\omega+\lambda+\gamma_{\bm{k}+\frac{\bm{Q}}{2}}^{B})e^{-i\omega0^{+}} & -\gamma_{\bm{k}+\frac{\bm{Q}}{2}}^{A} & -\frac{h}{2}\\
-\gamma_{\bm{k}+\frac{\bm{Q}}{2}}^{A} & (-i\omega+\lambda+\gamma_{\bm{k}+\frac{\bm{Q}}{2}}^{B})e^{i\omega0^{+}} & 0\\
0 & -\frac{h}{2} & -\gamma_{-\bm{k}+\frac{\bm{Q}}{2}}^{A}
\end{vmatrix}\nonumber \\
 & =\boxed{\gamma_{2}^{A}(\omega^{2}+\alpha_{1}^{2})+\frac{h^{2}}{4}\gamma_{1}^{A}}
\end{align}

\begin{align}
\mathcal{A}_{14} & =-\begin{vmatrix}-\gamma_{\bm{k}+\frac{\bm{Q}}{2}}^{A} & -\frac{h}{2} & 0\\
(-i\omega+\lambda+\gamma_{\bm{k}+\frac{\bm{Q}}{2}}^{B})e^{i\omega0^{+}} & 0 & -\frac{h}{2}\\
0 & (i\omega+\lambda+\gamma_{-\bm{k}+\frac{\bm{Q}}{2}}^{B})e^{-i\omega0^{+}} & -\gamma_{-\bm{k}+\frac{\bm{Q}}{2}}^{A}
\end{vmatrix}\nonumber \\
 & =\boxed{\frac{h}{2}[\gamma_{1}^{A}(i\omega+\lambda+\gamma_{2}^{B})e^{-i\omega0^{+}}+\gamma_{2}^{A}(-i\omega+\lambda+\gamma_{1}^{B})e^{i\omega0^{+}}]}
\end{align}

\begin{align}
\mathcal{A}_{24} & =\begin{vmatrix}(i\omega+\lambda+\gamma_{\bm{k}+\frac{\bm{Q}}{2}}^{B})e^{-i\omega0^{+}} & -\frac{h}{2} & 0\\
-\gamma_{\bm{k}+\frac{\bm{Q}}{2}}^{A} & 0 & -\frac{h}{2}\\
-\frac{h}{2} & (i\omega+\lambda+\gamma_{-\bm{k}+\frac{\bm{Q}}{2}}^{B})e^{-i\omega0^{+}} & -\gamma_{-\bm{k}+\frac{\bm{Q}}{2}}^{A}
\end{vmatrix}\nonumber \\
 & =\boxed{\frac{h}{2}[(i\omega+\lambda+\gamma_{1}^{B})(i\omega+\lambda+\gamma_{2}^{B})e^{-2i\omega0^{+}}+\gamma_{1}^{A}\gamma_{2}^{A}-\frac{h^{2}}{4}]}
\end{align}

\begin{align}
\mathcal{A}_{34} & =-\begin{vmatrix}(i\omega+\lambda+\gamma_{\bm{k}+\frac{\bm{Q}}{2}}^{B})e^{-i\omega0^{+}} & -\gamma_{\bm{k}+\frac{\bm{Q}}{2}}^{A} & 0\\
-\gamma_{\bm{k}+\frac{\bm{Q}}{2}}^{A} & (-i\omega+\lambda+\gamma_{\bm{k}+\frac{\bm{Q}}{2}}^{B})e^{i\omega0^{+}} & -\frac{h}{2}\\
-\frac{h}{2} & 0 & -\gamma_{-\bm{k}+\frac{\bm{Q}}{2}}^{A}
\end{vmatrix}\nonumber \\
 & =\boxed{\gamma_{2}^{A}(\omega^{2}+\alpha_{1}^{2})+\frac{h^{2}}{4}\gamma_{1}^{A}}
\end{align}

\begin{align}
\mathcal{A}_{44} & =\begin{vmatrix}(i\omega+\lambda+\gamma_{\bm{k}+\frac{\bm{Q}}{2}}^{B})e^{-i\omega0^{+}} & -\gamma_{\bm{k}+\frac{\bm{Q}}{2}}^{A} & -\frac{h}{2}\\
-\gamma_{\bm{k}+\frac{\bm{Q}}{2}}^{A} & (-i\omega+\lambda+\gamma_{\bm{k}+\frac{\bm{Q}}{2}}^{B})e^{i\omega0^{+}} & 0\\
-\frac{h}{2} & 0 & (i\omega+\lambda+\gamma_{-\bm{k}+\frac{\bm{Q}}{2}}^{B})e^{-i\omega0^{+}}
\end{vmatrix}\nonumber \\
 & =\boxed{(i\omega+\lambda+\gamma_{2}^{B})e^{-i\omega0^{+}}(\omega^{2}+\alpha_{1}^{2})-\frac{h^{2}}{4}(-i\omega+\lambda+\gamma_{1}^{B})e^{i\omega0^{+}}}
\end{align}

The adjunct matrix $\mathcal{A}$ is a linear combination of $g_{\bm{k}}^{\sigma\sigma'}$
matrices

\begin{align}
\mathcal{A} & =-\bigg(g_{\bm{k}}^{-+}(i\omega+\epsilon_{\bm{k}}^{+})[\omega^{2}+(\epsilon_{\bm{k}}^{-})^{2}]+g_{\bm{k}}^{++}(i\omega-\epsilon_{\bm{k}}^{+})[\omega^{2}+(\epsilon_{\bm{k}}^{-})^{2}]\nonumber \\
 & +g_{\bm{k}}^{--}(i\omega+\epsilon_{\bm{k}}^{-})[\omega^{2}+(\epsilon_{\bm{k}}^{+})^{2}]+g_{\bm{k}}^{+-}(i\omega-\epsilon_{\bm{k}}^{-})[\omega^{2}+(\epsilon_{\bm{k}}^{+})^{2}]\bigg)\label{eq:adjunct}
\end{align}

The matrix elements of $g_{\bm{k}}^{\sigma\sigma'}$ are determined
by solving \eqref{eq:adjunct} element by element.

\begin{equation}
\mathcal{A}_{11}=-i\omega^{3}+(\lambda+\gamma_{1}^{B})\omega^{2}-i(\alpha_{2}^{2}+\frac{h^{2}}{4})\omega+(\lambda+\gamma_{1}^{B})\alpha_{2}^{2}-\frac{h^{2}}{4}(\lambda+\gamma_{2}^{B})
\end{equation}

We have

\begin{equation}
\big(g_{\bm{k}}^{-+}+g_{\bm{k}}^{++}+g_{\bm{k}}^{--}+g_{\bm{k}}^{+-}\big)_{11}=1
\end{equation}

\begin{equation}
\big((\epsilon_{\bm{k}}^{+}g_{\bm{k}}^{-+}-\epsilon_{\bm{k}}^{+}g_{\bm{k}}^{++}+\epsilon_{\bm{k}}^{-}g_{\bm{k}}^{--}-\epsilon_{\bm{k}}^{-}g_{\bm{k}}^{+-}\big)_{11}=-(\lambda+\gamma_{1}^{B})
\end{equation}

\begin{equation}
\big((\epsilon_{\bm{k}}^{-})^{2}g_{\bm{k}}^{-+}+(\epsilon_{\bm{k}}^{-})^{2}g_{\bm{k}}^{++}+(\epsilon_{\bm{k}}^{+})^{2}g_{\bm{k}}^{--}+(\epsilon_{\bm{k}}^{+})^{2}g_{\bm{k}}^{+-}\big)_{11}=\alpha_{2}^{2}+\frac{h^{2}}{4}
\end{equation}

\begin{equation}
\big(\epsilon_{\bm{k}}^{+}(\epsilon_{\bm{k}}^{-})^{2}g_{\bm{k}}^{-+}-\epsilon_{\bm{k}}^{+}(\epsilon_{\bm{k}}^{-})^{2}g_{\bm{k}}^{++}+\epsilon_{\bm{k}}^{-}(\epsilon_{\bm{k}}^{+})^{2}g_{\bm{k}}^{--}-\epsilon_{\bm{k}}^{-}(\epsilon_{\bm{k}}^{+})^{2}g_{\bm{k}}^{+-}\big)_{11}=\frac{h^{2}}{4}(\lambda+\gamma_{2}^{B})-(\lambda+\gamma_{1}^{B})\alpha_{2}^{2}
\end{equation}

Define $X_{1}=\big(g_{\bm{k}}^{-+}-g_{\bm{k}}^{++}\big)_{11}$ , $X_{2}=\big(g_{\bm{k}}^{-+}-g_{\bm{k}}^{+-})_{11}$
, $Y_{1}=\big(g_{\bm{k}}^{-+}+g_{\bm{k}}^{++}\big)_{11}$ , $Y_{2}=\big(g_{\bm{k}}^{-+}+g_{\bm{k}}^{+-})_{11}$
. Thus

\[
Y_{1}+Y_{2}=1
\]

\[
\epsilon_{\bm{k}}^{+}X_{1}+\epsilon_{\bm{k}}^{-}X_{2}=-(\lambda+\gamma_{1}^{B})
\]

\[
(\epsilon_{\bm{k}}^{-})^{2}Y_{1}+(\epsilon_{\bm{k}}^{+})^{2}Y_{2}=\alpha_{2}^{2}+\frac{h^{2}}{4}
\]

\[
\epsilon_{\bm{k}}^{+}(\epsilon_{\bm{k}}^{-})^{2}X_{1}+\epsilon_{\bm{k}}^{-}(\epsilon_{\bm{k}}^{+})^{2}X_{2}=\frac{h^{2}}{4}(\lambda+\gamma_{2}^{B})-(\lambda+\gamma_{1}^{B})\alpha_{2}^{2}
\]

We have

\[
Y_{1}=\frac{1}{\Delta_{\bm{k}}^{2}}[(\epsilon_{\bm{k}}^{+})^{2}-(\alpha_{2}^{2}+\frac{h^{2}}{4})]
\]

\[
Y_{2}=-\frac{1}{\Delta_{\bm{k}}^{2}}[(\epsilon_{\bm{k}}^{-})^{2}-(\alpha_{2}^{2}+\frac{h^{2}}{4})]
\]

\[
X_{1}=\frac{(\lambda+\gamma_{1}^{B})}{\epsilon_{\bm{k}}^{+}\Delta_{\bm{k}}^{2}}\bigg(\alpha_{2}^{2}-(\epsilon_{\bm{k}}^{+})^{2}\bigg)-\frac{h^{2}}{4}\frac{(\lambda+\gamma_{2}^{B})}{\epsilon_{\bm{k}}^{+}\Delta_{\bm{k}}^{2}}
\]

\[
X_{2}=-\frac{(\lambda+\gamma_{1}^{B})}{\epsilon_{\bm{k}}^{-}\Delta_{\bm{k}}^{2}}\bigg(\alpha_{2}^{2}-(\epsilon_{\bm{k}}^{-})^{2}\bigg)+\frac{h^{2}}{4}\frac{(\lambda+\gamma_{2}^{B})}{\epsilon_{\bm{k}}^{-}\Delta_{\bm{k}}^{2}}
\]

The matrix elements are

\begin{align*}
(g_{\bm{k}}^{-+})_{11} & =-\frac{(\Delta_{-\bm{k}}^{+})^{2}}{\Delta_{\bm{k}}^{2}}\bigg[\frac{\lambda+\gamma_{1}^{B}}{2\epsilon_{\bm{k}}^{+}}-\frac{1}{2}\bigg]-\frac{h^{2}}{4\Delta_{\bm{k}}^{2}}\bigg[\frac{\lambda+\gamma_{2}^{B}}{2\epsilon_{\bm{k}}^{+}}+\frac{1}{2}\bigg]
\end{align*}

\end{document}
